%% Generated by Sphinx.
\def\sphinxdocclass{report}
\documentclass[a4paper,11pt,english]{sphinxmanual}
\ifdefined\pdfpxdimen
   \let\sphinxpxdimen\pdfpxdimen\else\newdimen\sphinxpxdimen
\fi \sphinxpxdimen=.75bp\relax
\ifdefined\pdfimageresolution
    \pdfimageresolution= \numexpr \dimexpr1in\relax/\sphinxpxdimen\relax
\fi
%% let collapsible pdf bookmarks panel have high depth per default
\PassOptionsToPackage{bookmarksdepth=5}{hyperref}

\PassOptionsToPackage{booktabs}{sphinx}
\PassOptionsToPackage{colorrows}{sphinx}

\PassOptionsToPackage{warn}{textcomp}
\usepackage[utf8]{inputenc}
\ifdefined\DeclareUnicodeCharacter
% support both utf8 and utf8x syntaxes
  \ifdefined\DeclareUnicodeCharacterAsOptional
    \def\sphinxDUC#1{\DeclareUnicodeCharacter{"#1}}
  \else
    \let\sphinxDUC\DeclareUnicodeCharacter
  \fi
  \sphinxDUC{00A0}{\nobreakspace}
  \sphinxDUC{2500}{\sphinxunichar{2500}}
  \sphinxDUC{2502}{\sphinxunichar{2502}}
  \sphinxDUC{2514}{\sphinxunichar{2514}}
  \sphinxDUC{251C}{\sphinxunichar{251C}}
  \sphinxDUC{2572}{\textbackslash}
\fi
\usepackage{cmap}
\usepackage[T1]{fontenc}
\usepackage{amsmath,amssymb,amstext}
\usepackage{babel}



\usepackage{tgtermes}
\usepackage{tgheros}
\renewcommand{\ttdefault}{txtt}



\usepackage[Bjarne]{fncychap}
\usepackage{sphinx}

\fvset{fontsize=auto}
\usepackage{geometry}


% Include hyperref last.
\usepackage{hyperref}
% Fix anchor placement for figures with captions.
\usepackage{hypcap}% it must be loaded after hyperref.
% Set up styles of URL: it should be placed after hyperref.
\urlstyle{same}


\usepackage{sphinxmessages}


\usepackage[utf8]{inputenc}
\DeclareUnicodeCharacter{2212}{\textendash}


\title{User Guide}
\date{Friday, 27 December 2024 at 13:45}
\release{2.6b1}
\author{Veronica Berglyd Olsen}
\newcommand{\sphinxlogo}{\sphinxincludegraphics{novelwriter-pdf.png}\par}
\renewcommand{\releasename}{Release}
\makeindex
\begin{document}

\ifdefined\shorthandoff
  \ifnum\catcode`\=\string=\active\shorthandoff{=}\fi
  \ifnum\catcode`\"=\active\shorthandoff{"}\fi
\fi

\pagestyle{empty}
\sphinxmaketitle
\pagestyle{plain}
\sphinxtableofcontents
\pagestyle{normal}
\phantomsection\label{\detokenize{index::doc}}


\begin{DUlineblock}{0em}
\item[] \sphinxstylestrong{Release Version:} 2.6b1
\item[] \sphinxstylestrong{Updated:} Friday, 27 December 2024 at 13:45
\end{DUlineblock}

\sphinxAtStartPar
novelWriter is an open source plain text editor designed for writing novels assembled from
individual text documents. It uses a minimal formatting syntax inspired by Markdown, and adds a
meta data syntax for comments, synopsis, and cross\sphinxhyphen{}referencing. It is designed to be a simple text
editor that allows for easy organisation of text and notes, using human readable text files as
storage for robustness.

\begin{figure}[htbp]
\centering

\noindent\sphinxincludegraphics[width=500\sphinxpxdimen]{{screenshot_multi}.png}
\end{figure}

\sphinxAtStartPar
The project storage is suitable for version control software, and also well suited for file
synchronisation tools. All text is saved as plain text files, and your project data as standard
data formats in XML and JSON. See {\hyperref[\detokenize{tech_storage:a-storage}]{\sphinxcrossref{\DUrole{std}{\DUrole{std-ref}{How Data is Stored}}}}} for more details.

\sphinxAtStartPar
Any operating system that has Python 3 and the Qt 5 libraries should be able to run novelWriter.
It runs fine on Linux, Windows and MacOS, and users have tested it on other platforms as well.
See {\hyperref[\detokenize{int_started:a-started}]{\sphinxcrossref{\DUrole{std}{\DUrole{std-ref}{Setup and Installation}}}}} for more details.

\sphinxAtStartPar
\sphinxstylestrong{Useful Links}
\begin{itemize}
\item {} 
\sphinxAtStartPar
Website: \sphinxurl{https://novelwriter.io}

\item {} 
\sphinxAtStartPar
Documentation: \sphinxurl{https://docs.novelwriter.io}

\item {} 
\sphinxAtStartPar
Public Releases: \sphinxurl{https://releases.novelwriter.io}

\item {} 
\sphinxAtStartPar
Internationalisation: \sphinxurl{https://crowdin.com/project/novelwriter}

\item {} 
\sphinxAtStartPar
Source Code: \sphinxurl{https://github.com/vkbo/novelWriter}

\item {} 
\sphinxAtStartPar
Source Releases: \sphinxurl{https://github.com/vkbo/novelWriter/releases}

\item {} 
\sphinxAtStartPar
Issue Tracker: \sphinxurl{https://github.com/vkbo/novelWriter/issues}

\item {} 
\sphinxAtStartPar
Feature Discussions: \sphinxurl{https://github.com/vkbo/novelWriter/discussions}

\item {} 
\sphinxAtStartPar
PyPi Project: \sphinxurl{https://pypi.org/project/novelWriter}

\item {} 
\sphinxAtStartPar
Social Media: \sphinxurl{https://fosstodon.org/@novelwriter}

\end{itemize}

\sphinxstepscope


\chapter{Overview}
\label{\detokenize{int_introduction:overview}}\label{\detokenize{int_introduction:a-intro}}\label{\detokenize{int_introduction::doc}}
\sphinxAtStartPar
At its core, novelWriter is a multi\sphinxhyphen{}document plain text editor. The idea is to let you edit your
text without having to deal with formatting until you generate a draft document or manuscript.
Instead, you can focus on the writing right from the start.

\sphinxAtStartPar
Of course, you probably need \sphinxstyleemphasis{some} formatting for your text. At the very least you need emphasis.
Most people are familiar with adding emphasis using \sphinxcode{\sphinxupquote{\_underscores\_}} and \sphinxcode{\sphinxupquote{**asterisks**}}. This
formatting standard comes from \sphinxhref{https://en.wikipedia.org/wiki/Markdown}{Markdown} and is supported by novelWriter. It also uses Markdown
formatting for defining document headings. If you need more specialised formatting, additional
formatting options are available using a shortcode format. See {\hyperref[\detokenize{usage_format:a-fmt-shortcodes}]{\sphinxcrossref{\DUrole{std}{\DUrole{std-ref}{Formatting with Shortcodes}}}}} for more
details.

\begin{sphinxadmonition}{note}{Limitations}

\sphinxAtStartPar
novelWriter is designed for writing fiction, so the formatting features available are limited to
those relevant for this purpose. It is \sphinxstyleemphasis{not} suitable for technical writing, and it is \sphinxstyleemphasis{not} a
full\sphinxhyphen{}featured Markdown editor.

\sphinxAtStartPar
It is also not intended as a tool for organising research for writing, and therefore lacks
formatting features you may need for this purpose. The notes feature in novelWriter is mainly
intended for character profiles and plot outlines.
\end{sphinxadmonition}

\sphinxAtStartPar
Your novel project in novelWriter is organised as a collection of separate plain text documents
instead of a single, large document. The idea is to make it easier to reorganise your project
structure without having to cut and paste text between chapters and scenes.

\sphinxAtStartPar
There are two kinds of documents in your project: {\hyperref[\detokenize{int_glossary:term-Novel-Documents}]{\sphinxtermref{\DUrole{xref}{\DUrole{std}{\DUrole{std-term}{Novel Documents}}}}}} are documents that are
part of your story. The other kind of documents are {\hyperref[\detokenize{int_glossary:term-Project-Notes}]{\sphinxtermref{\DUrole{xref}{\DUrole{std}{\DUrole{std-term}{Project Notes}}}}}}, which are intended for
your notes about your characters, your world building, and so on.

\sphinxAtStartPar
You can at any point split the individual documents by their headings up into multiple documents,
or merge multiple documents into a single document. This makes it easier to use variations of the
\sphinxhref{https://www.advancedfictionwriting.com/articles/snowflake-method/}{Snowflake} method for writing. You can start by writing larger structure\sphinxhyphen{}focused documents, like
for instance one document per act, and later effortlessly split these up into chapters or scenes.


\section{Key Features}
\label{\detokenize{int_introduction:key-features}}\label{\detokenize{int_introduction:a-intro-features}}
\sphinxAtStartPar
Below are some key features of novelWriter.
\begin{description}
\sphinxlineitem{\sphinxstylestrong{Focus on writing}}
\sphinxAtStartPar
The aim of the user interface is to let you focus on writing instead of spending time formatting
text. Formatting is therefore limited to a small set of formatting tags for simple things like
text emphasis and paragraph alignment. Additional shortcodes are available for special
formatting cases.

\sphinxAtStartPar
When you really want to focus on just writing, you can switch the editor into \sphinxstylestrong{Focus Mode}
where only the text editor panel itself is visible, and the project structure view is hidden
away.

\sphinxlineitem{\sphinxstylestrong{Keep an eye on your notes}}
\sphinxAtStartPar
The main window can optionally show a document viewer to the right of the editor. The viewer
is intended for displaying another scene document, your character notes, plot notes, or any
other document you may need to reference while writing. It is not intended as a preview panel
for the document you’re editing, but if you wish, you can also use it for this purpose.

\sphinxlineitem{\sphinxstylestrong{Organise your documents how you like}}
\sphinxAtStartPar
You can split your novel project up into as many individual documents as you want to. When you
build the project into a manuscript, they are all glued together in the top\sphinxhyphen{}to\sphinxhyphen{}bottom order in
which they appear in the project tree. You can use as few text documents as you like, but
splitting the project up into chapters and scenes means you can easily reorder them using the
drag\sphinxhyphen{}and\sphinxhyphen{}drop feature of the project tree. You can also start out with fewer documents and then
later split them into multiple documents based on chapter and scene headings.

\sphinxlineitem{\sphinxstylestrong{Multi\sphinxhyphen{}novel project support}}
\sphinxAtStartPar
The main parts of your project is split up into top level special folders called “Root” folders.
Your main story text lives in the “Novel” root folder. You can have multiple such folders in a
project, and rename them to whatever you want. This allows you to keep a series of individual
novels with the same characters and world building in the same project, and create manuscripts
for them individually.

\sphinxlineitem{\sphinxstylestrong{Keep track of your story elements}}
\sphinxAtStartPar
All notes in your project can be assigned a {\hyperref[\detokenize{int_glossary:term-Tag}]{\sphinxtermref{\DUrole{xref}{\DUrole{std}{\DUrole{std-term}{tag}}}}}} that you can then {\hyperref[\detokenize{int_glossary:term-Reference}]{\sphinxtermref{\DUrole{xref}{\DUrole{std}{\DUrole{std-term}{reference}}}}}} from
any other document or note. In fact, you can add a new tag under each heading of a note if you
need to be able to reference specific sections of it, or you want to keep several topics in the
same note.

\sphinxlineitem{\sphinxstylestrong{Get an overview of your story}}
\sphinxAtStartPar
It is not the documents themselves that define the chapters and scenes of your story, but the
headings that separate them. In the \sphinxstylestrong{Outline View} on the main window you can see an outline
of all the chapter and scene headings of each novel root folder in your project. If they have
any references in them, like which character is in what chapter and scene, these are listed in
additional columns.

\sphinxAtStartPar
You can also add a synopsis to each chapter or scene, which can be listed here as well. You have
the option to add or remove columns of information from this outline. A subset of the outline
information is also available in the \sphinxstylestrong{Novel View} as an alternative view to the project tree.

\sphinxlineitem{\sphinxstylestrong{Get an overview of your story elements}}
\sphinxAtStartPar
Under the document viewer panel you will find a series of tabs that show the different story
elements you have created tags for. The tabs are sorted into \sphinxstylestrong{Characters}, \sphinxstylestrong{Plots}, etc,
depending on which categories you are using in your story. This panel can be hidden to free up
space when you don’t need it.

\sphinxlineitem{\sphinxstylestrong{Assembling your manuscript}}
\sphinxAtStartPar
Whether you want to assemble a manuscript, or export all your notes, or generate an outline of
your chapters and scenes with a synopsis included, you can use the \sphinxstylestrong{Build Manuscript} tool to
do so. The tool lets you select what information you want to include in the generated document,
and how it is formatted. You can send the result to a printer or PDF, or generate an Open
Document file that can be opened by most office type word processors. You can also generate the
result as HTML, or Markdown, both suitable for further conversion to other formats.

\end{description}


\section{Screenshots}
\label{\detokenize{int_introduction:screenshots}}\label{\detokenize{int_introduction:a-intro-screenshots}}
\begin{figure}[htbp]
\centering
\capstart

\noindent\sphinxincludegraphics{{screenshot_light}.png}
\caption{novelWriter with light colour theme}\label{\detokenize{int_introduction:id1}}\end{figure}

\begin{figure}[htbp]
\centering
\capstart

\noindent\sphinxincludegraphics{{screenshot_dark}.png}
\caption{novelWriter with dark colour theme}\label{\detokenize{int_introduction:id2}}\end{figure}

\sphinxstepscope


\chapter{What to Read First}
\label{\detokenize{int_overview:what-to-read-first}}\label{\detokenize{int_overview:a-reading}}\label{\detokenize{int_overview::doc}}
\sphinxAtStartPar
The documentation of novelWriter is quite extensive. There are a lot of features to get used to,
but you don’t need all of them to get started.

\sphinxAtStartPar
The chapters below labelled “Essential Information” are the ones you need to know to use the
application correctly. By “correctly”, it is meant: in a way so novelWriter understands the basic
structure of your text. It collects a lot of information from your text and uses it to display the
structure of it in various ways to help you get an overview of your writing.

\sphinxAtStartPar
The chapters labelled “Recommended Reading” includes additional information on how the different
parts if the application work and what the features do.

\sphinxAtStartPar
The “Optional” and “Lookup” chapters contain additional information or lookup tables that are not
essential for using the application.


\section{Using novelWriter}
\label{\detokenize{int_overview:using-novelwriter}}
\sphinxAtStartPar
In order to use novelWriter effectively, you need to know the basics of how it works. The following
chapters will explain the main principles. It starts with the basics, and gets more detailed as you
read on.
\begin{description}
\sphinxlineitem{{\hyperref[\detokenize{usage_breakdown:a-breakdown}]{\sphinxcrossref{\DUrole{std}{\DUrole{std-ref}{How it Works}}}}} \textendash{} Essential Information}
\sphinxAtStartPar
This chapter explains the basics of how the application works and what it can and cannot do.

\sphinxlineitem{{\hyperref[\detokenize{usage_project:a-ui-project}]{\sphinxcrossref{\DUrole{std}{\DUrole{std-ref}{Project Views}}}}} \textendash{} Recommended Reading}
\sphinxAtStartPar
This chapter will give you a more detailed explanation of how you can use the user interface
components to organise and view your project work.

\sphinxlineitem{{\hyperref[\detokenize{usage_writing:a-ui-writing}]{\sphinxcrossref{\DUrole{std}{\DUrole{std-ref}{The Editor and Viewer}}}}} \textendash{} Recommended Reading}
\sphinxAtStartPar
This chapter will give you a more detailed explanation of how the text editor and viewer work.

\sphinxlineitem{{\hyperref[\detokenize{usage_format:a-fmt}]{\sphinxcrossref{\DUrole{std}{\DUrole{std-ref}{Formatting Your Text}}}}} \textendash{} Essential Information}
\sphinxAtStartPar
This chapter covers how you should format your text. The editor is plain text, so text
formatting requires some basic markup. The structure of your novel is also inferred from how you
use headings. Tags and references are implemented by special keywords.

\sphinxlineitem{{\hyperref[\detokenize{usage_shortcuts:a-kb}]{\sphinxcrossref{\DUrole{std}{\DUrole{std-ref}{Keyboard Shortcuts}}}}} \textendash{} Optional / Lookup}
\sphinxAtStartPar
This chapter lists all the keyboard shortcuts in novelWriter and what they do. Most of the
shortcuts are also listed next to their menu entries inside the app, or in tool tips. This
chapter is mostly for reference.

\sphinxlineitem{{\hyperref[\detokenize{usage_typography:a-typ}]{\sphinxcrossref{\DUrole{std}{\DUrole{std-ref}{Typographical Notes}}}}} \textendash{} Optional}
\sphinxAtStartPar
This chapter gives you an overview of the special typographical symbols available in
novelWriter. The auto\sphinxhyphen{}replace feature can handle the insertion of standard quote symbols for
your language, and other special characters. If you use any symbols aside from these, their
intended use is explained here.

\end{description}


\section{Organising Your Projects}
\label{\detokenize{int_overview:organising-your-projects}}
\sphinxAtStartPar
In addition to managing a collection of plain text files, novelWriter can interpret and map the
structure of your novel and show you additional information about its flow and content. In order
to take advantage of these features, you must structure your text in a specific way and add some
meta data for it to extract.
\begin{description}
\sphinxlineitem{{\hyperref[\detokenize{project_overview:a-proj}]{\sphinxcrossref{\DUrole{std}{\DUrole{std-ref}{Managing Projects}}}}} \textendash{} Essential Information}
\sphinxAtStartPar
This chapter explains how you organise the content of your project, and how to set up automated
backups of your work.

\sphinxlineitem{{\hyperref[\detokenize{project_structure:a-struct}]{\sphinxcrossref{\DUrole{std}{\DUrole{std-ref}{Novel Structure}}}}} \textendash{} Essential Information}
\sphinxAtStartPar
This chapter covers the way your novel’s structure is encoded into the text documents. It
explains how the different levels of headings are used, and some special formatting for
different kinds of headings.

\sphinxlineitem{{\hyperref[\detokenize{project_references:a-references}]{\sphinxcrossref{\DUrole{std}{\DUrole{std-ref}{Tags and References}}}}} \sphinxhyphen{} Recommended Reading}
\sphinxAtStartPar
This chapter explains how you organise your notes, and how the Tags and References system works.
This system lets you cross\sphinxhyphen{}link your documents in your project, and display these references in
the application interface.

\sphinxlineitem{{\hyperref[\detokenize{project_manuscript:a-manuscript}]{\sphinxcrossref{\DUrole{std}{\DUrole{std-ref}{Building the Manuscript}}}}} \sphinxhyphen{} Recommended Reading}
\sphinxAtStartPar
This chapter explains how the \sphinxstylestrong{Manuscript Build} tool works, how you can control the way
chapter titles are formatted, and how scene and section breaks are handled.

\end{description}


\section{Additional Details \& Technical Topics}
\label{\detokenize{int_overview:additional-details-technical-topics}}
\sphinxAtStartPar
The Additional Details and the Technical Topics sections contain more in\sphinxhyphen{}depth information about
how various bits of novelWriter works. This information is not essential to getting started using
novelWriter.

\sphinxstepscope


\chapter{Setup and Installation}
\label{\detokenize{int_started:setup-and-installation}}\label{\detokenize{int_started:a-started}}\label{\detokenize{int_started::doc}}
\sphinxAtStartPar
Ready\sphinxhyphen{}made packages and installers for novelWriter are available for all major platforms, including
Linux, Windows and MacOS, from the \sphinxhref{https://download.novelwriter.io}{Downloads page}. See below for install additional instructions
for each platform.

\sphinxAtStartPar
You can also install novelWriter from the Python Package Index (\sphinxhref{https://pypi.org/project/novelWriter/}{PyPi}). See {\hyperref[\detokenize{int_started:a-started-pip}]{\sphinxcrossref{\DUrole{std}{\DUrole{std-ref}{Installing from PyPi}}}}}.
Installing from PyPi does not set up icon launchers, so you will either have to do this yourself,
or start novelWriter from the command line.

\sphinxAtStartPar
Spell checking in novelWriter is provided by a third party library called \sphinxhref{https://abiword.github.io/enchant/}{Enchant}. Generally, it
should pull dictionaries from your operating system automatically. However, on Windows they must be
installed manually. See {\hyperref[\detokenize{more_customise:a-custom-dict}]{\sphinxcrossref{\DUrole{std}{\DUrole{std-ref}{Spell Check Dictionaries}}}}} for more details.


\section{Installing on Windows}
\label{\detokenize{int_started:installing-on-windows}}\label{\detokenize{int_started:a-started-windows}}
\sphinxAtStartPar
You can install novelWriter with both Python and library dependencies embedded using the Windows
Installer (setup.exe) file from the \sphinxhref{https://download.novelwriter.io}{Downloads page}, or from the \sphinxhref{https://github.com/vkbo/novelWriter/releases}{Releases} page on \sphinxhref{https://github.com/vkbo/novelWriter}{GitHub}.
Installing it should be straightforward.

\sphinxAtStartPar
If you have any issues, try uninstalling the previous version and making a fresh install. If you
already had a version installed via a different method, you should uninstall that first as having
multiple installations has been known to cause problems.

\begin{sphinxadmonition}{note}{Note:}
\sphinxAtStartPar
The novelWriter installer is not signed because Microsoft doesn’t currently provide a way for
non\sphinxhyphen{}profit open source projects to properly sign their installers. The novelWriter project
doesn’t have the funding to pay for commercial software signing certificates. You will therefore
see an additional warning about this when you download and run the installer.
\end{sphinxadmonition}


\section{Installing on Linux}
\label{\detokenize{int_started:installing-on-linux}}\label{\detokenize{int_started:a-started-linux}}
\sphinxAtStartPar
A Debian package can be downloaded from the \sphinxhref{https://download.novelwriter.io}{Downloads page}, or from the \sphinxhref{https://github.com/vkbo/novelWriter/releases}{Releases} page on
\sphinxhref{https://github.com/vkbo/novelWriter}{GitHub}. This package should work on both Debian, Ubuntu and Linux Mint, at least.

\sphinxAtStartPar
If you prefer, you can also add the novelWriter repository on Launchpad to your package manager.
The Launchpad packages \sphinxhref{https://launchpad.net/~vkbo}{are signed by the author}.


\subsection{Ubuntu}
\label{\detokenize{int_started:ubuntu}}
\sphinxAtStartPar
You can add the Ubuntu \sphinxhref{https://launchpad.net/~vkbo/+archive/ubuntu/novelwriter}{PPA} and install novelWriter with the following commands.

\begin{sphinxVerbatim}[commandchars=\\\{\}]
sudo\PYG{+w}{ }add\PYGZhy{}apt\PYGZhy{}repository\PYG{+w}{ }ppa:vkbo/novelwriter
sudo\PYG{+w}{ }apt\PYG{+w}{ }update
sudo\PYG{+w}{ }apt\PYG{+w}{ }install\PYG{+w}{ }novelwriter
\end{sphinxVerbatim}

\sphinxAtStartPar
If you want the \sphinxhref{https://launchpad.net/~vkbo/+archive/ubuntu/novelwriter-pre}{Pre\sphinxhyphen{}Release PPA} instead, add the \sphinxcode{\sphinxupquote{ppa:vkbo/novelwriter\sphinxhyphen{}pre}} repository.


\subsection{Debian and Mint}
\label{\detokenize{int_started:debian-and-mint}}
\sphinxAtStartPar
Since this is a pure Python package, the Launchpad PPA can in principle also be used on Debian or
Mint. However, the above command will fail to add the signing key, as it is Ubuntu\sphinxhyphen{}specific.

\sphinxAtStartPar
Instead, run the following commands to add the repository and key:

\begin{sphinxVerbatim}[commandchars=\\\{\}]
sudo\PYG{+w}{ }gpg\PYG{+w}{ }\PYGZhy{}\PYGZhy{}no\PYGZhy{}default\PYGZhy{}keyring\PYG{+w}{ }\PYGZhy{}\PYGZhy{}keyring\PYG{+w}{ }/usr/share/keyrings/novelwriter\PYGZhy{}ppa\PYGZhy{}keyring.gpg\PYG{+w}{ }\PYGZhy{}\PYGZhy{}keyserver\PYG{+w}{ }hkp://keyserver.ubuntu.com:80\PYG{+w}{ }\PYGZhy{}\PYGZhy{}recv\PYGZhy{}keys\PYG{+w}{ }F19F1FCE50043114
\PYG{n+nb}{echo}\PYG{+w}{ }\PYG{l+s+s2}{\PYGZdq{}deb [signed\PYGZhy{}by=/usr/share/keyrings/novelwriter\PYGZhy{}ppa\PYGZhy{}keyring.gpg] http://ppa.launchpad.net/vkbo/novelwriter/ubuntu noble main\PYGZdq{}}\PYG{+w}{ }\PYG{p}{|}\PYG{+w}{ }sudo\PYG{+w}{ }tee\PYG{+w}{ }/etc/apt/sources.list.d/novelwriter.list
\end{sphinxVerbatim}

\sphinxAtStartPar
Then run the update and install commands as for Ubuntu:

\begin{sphinxVerbatim}[commandchars=\\\{\}]
sudo\PYG{+w}{ }apt\PYG{+w}{ }update
sudo\PYG{+w}{ }apt\PYG{+w}{ }install\PYG{+w}{ }novelwriter
\end{sphinxVerbatim}

\begin{sphinxadmonition}{note}{Note:}
\sphinxAtStartPar
You may need to use the Ubuntu 20.04 (focal) packages for Debian 11 or earlier. The newer Ubuntu
packages use a different compression algorithm that may not be supported.
\end{sphinxadmonition}

\begin{sphinxadmonition}{tip}{Tip:}
\sphinxAtStartPar
If you get an error message like \sphinxcode{\sphinxupquote{gpg: failed to create temporary file}} when importing the key
from the Ubuntu keyserver, try creating the folder it fails on, and import the key again:

\begin{sphinxVerbatim}[commandchars=\\\{\}]
sudo\PYG{+w}{ }mkdir\PYG{+w}{ }/root/.gnupg/
\end{sphinxVerbatim}
\end{sphinxadmonition}


\subsection{AppImage Releases}
\label{\detokenize{int_started:appimage-releases}}
\sphinxAtStartPar
For other Linux distros than the ones mentioned above, the primary option is \sphinxhref{https://appimage.org/}{AppImage}. These are
completely standalone images for the app that include the necessary environment to run novelWriter.
They can of course be run on any Linux distro, if you prefer this to native packages.

\begin{sphinxadmonition}{note}{Note:}
\sphinxAtStartPar
novelWriter generally doesn’t support Python versions that have reached end of life. If your
Linux distro still uses older Python versions and novelWriter won’t run, you may want to try the
AppImage instead.
\end{sphinxadmonition}


\section{Installing on MacOS}
\label{\detokenize{int_started:installing-on-macos}}\label{\detokenize{int_started:a-started-macos}}
\sphinxAtStartPar
You can install novelWriter with both its Python and library dependencies embedded using the DMG
application image file from the \sphinxhref{https://download.novelwriter.io}{Downloads page}, or from the \sphinxhref{https://github.com/vkbo/novelWriter/releases}{Releases} page on \sphinxhref{https://github.com/vkbo/novelWriter}{GitHub}.
Installing it should be straightforward.
\begin{itemize}
\item {} 
\sphinxAtStartPar
Download the DMG file and open it. Then drag the novelWriter icon to the \sphinxguilabel{Applications}
folder on the right. This will install it into your \sphinxguilabel{Applications}.

\item {} 
\sphinxAtStartPar
The first time you try to launch it, it will say that the bundle cannot be verified, simply press
the \sphinxguilabel{Open} button to add an exception.

\item {} 
\sphinxAtStartPar
If you are not presented with an \sphinxguilabel{Open} button in the dialog, launch the application
again by right clicking on the application in Finder and selecting \sphinxguilabel{Open} from the
context menu.

\end{itemize}

\sphinxAtStartPar
The context menu can also be accessed by option\sphinxhyphen{}clicking if you have a one button mouse. This is
done by holding down the option key on your keyboard and clicking on the application in Finder.

\begin{sphinxadmonition}{note}{Note:}
\sphinxAtStartPar
The novelWriter DMG is not signed because Apple doesn’t currently provide a way for non\sphinxhyphen{}profit
open source projects to properly sign their installers. The novelWriter project doesn’t have the
funding to pay for commercial software signing certificates.
\end{sphinxadmonition}


\section{Installing from PyPi}
\label{\detokenize{int_started:installing-from-pypi}}\label{\detokenize{int_started:a-started-pip}}
\sphinxAtStartPar
novelWriter is also available on the Python Package Index, or \sphinxhref{https://pypi.org/project/novelWriter/}{PyPi}. This install method works on
all supported operating systems with a suitable Python environment.

\sphinxAtStartPar
To install from PyPi you must first have the \sphinxcode{\sphinxupquote{python}} and \sphinxcode{\sphinxupquote{pip}} commands available on your
system. You can download Python from \sphinxhref{https://www.python.org/downloads/}{python.org}. It is recommended that you install the latest
version. If you are on Windows, also make sure to select the “Add Python to PATH” option during
installation.

\sphinxAtStartPar
To install novelWriter from PyPi, use the following command:

\begin{sphinxVerbatim}[commandchars=\\\{\}]
pip\PYG{+w}{ }install\PYG{+w}{ }novelwriter
\end{sphinxVerbatim}

\sphinxAtStartPar
To upgrade an existing installation, use:

\begin{sphinxVerbatim}[commandchars=\\\{\}]
pip\PYG{+w}{ }install\PYG{+w}{ }\PYGZhy{}\PYGZhy{}upgrade\PYG{+w}{ }novelwriter
\end{sphinxVerbatim}

\sphinxAtStartPar
When installing via pip, novelWriter can be launched from command line with:

\begin{sphinxVerbatim}[commandchars=\\\{\}]
novelwriter
\end{sphinxVerbatim}

\sphinxAtStartPar
Make sure the install location for pip is in your PATH variable. This is not always the case by
default, and then you may get a “Not Found” error when running the \sphinxcode{\sphinxupquote{novelwriter}} command.

\sphinxstepscope


\chapter{Tips \& Tricks}
\label{\detokenize{int_howto:tips-tricks}}\label{\detokenize{int_howto:a-howto}}\label{\detokenize{int_howto::doc}}
\sphinxAtStartPar
This is a list of hopefully helpful little tips on how to get the most out of novelWriter.

\begin{sphinxadmonition}{note}{Note:}
\sphinxAtStartPar
This section will be expanded over time. If you would like to have something added, feel free to
contribute, or start a discussion on the project’s \sphinxhref{https://github.com/vkbo/novelWriter/discussions}{Discussions Page}.
\end{sphinxadmonition}


\section{Managing the Project}
\label{\detokenize{int_howto:managing-the-project}}\subsubsection*{Create a Project from a Template}

\sphinxAtStartPar
On the Welcome dialog’s \sphinxstylestrong{Create New Project} form, you can select to “Prefill Project” from
the content of a different project. This feature is most useful if you copy a project you have
dedicated to be a template project. If you have a structure and settings you want to use for
every new project, this is the best solution.
\subsubsection*{Merge Multiple Documents Into One}

\sphinxAtStartPar
If you need to merge a selection of documents in your project into a single document, you can
achieve this by first making a new folder for just that purpose, and drag all the documents you
want merged into this folder. Then you can right click the folder, select \sphinxguilabel{Transform}
and \sphinxguilabel{Merge Documents in Folder}.

\sphinxAtStartPar
In the dialog that pops up, the documents will be in the same order as in the folder, but you
can rearrange them here of you wish. See {\hyperref[\detokenize{usage_project:a-ui-tree-split-merge}]{\sphinxcrossref{\DUrole{std}{\DUrole{std-ref}{Splitting and Merging Documents}}}}} for more details.


\section{Layout Tricks}
\label{\detokenize{int_howto:layout-tricks}}\subsubsection*{Align Paragraphs with Line Breaks}

\sphinxAtStartPar
If you have line breaks in you paragraphs, and also want to apply additional text alignment or
indentation, you must apply the alignment tags to the first line.

\sphinxAtStartPar
For example, this will centre the two lines.

\begin{sphinxVerbatim}[commandchars=\\\{\}]
\PYGZgt{}\PYGZgt{} Line one is centred. \PYGZlt{}\PYGZlt{}
Line two is also centred.

This text is not centred, because it is a new paragraph.
\end{sphinxVerbatim}

\sphinxAtStartPar
See {\hyperref[\detokenize{usage_format:a-fmt-align}]{\sphinxcrossref{\DUrole{std}{\DUrole{std-ref}{Paragraph Alignment and Indentation}}}}} for more details.
\subsubsection*{Create a Simple Table}

\sphinxAtStartPar
The formatting tools available in novelWriter don’t allow for complex structures like tables.
However, the editor does render tabs in a similar way that regular word processors do. You can
set the width of a tab in \sphinxstylestrong{Preferences}.

\sphinxAtStartPar
The tab key should have the same distance in the editor as in the viewer, so you can align text
in columns using the tab key, and it should look the same when viewed next to the editor.

\sphinxAtStartPar
This is most suitable for your notes, as the result in exported documents cannot be guaranteed
to match. Especially if you don’t use the same font in your manuscript as in the editor.
\subsubsection*{Turn Off First Line Indent for a Paragraph}

\sphinxAtStartPar
If you have first line indent enabled, but have a specific paragraph that you don’t want
indented, you can disable the indentation by explicitly add text alignment. Aligned paragraphs
are not indented. For instance by adding \sphinxcode{\sphinxupquote{\textless{}\textless{}}} to the end to left\sphinxhyphen{}align it,

\sphinxAtStartPar
See {\hyperref[\detokenize{usage_format:a-fmt-align}]{\sphinxcrossref{\DUrole{std}{\DUrole{std-ref}{Paragraph Alignment and Indentation}}}}} for more details.


\section{Organising Your Text}
\label{\detokenize{int_howto:organising-your-text}}\subsubsection*{Add Introductory Text to Chapters}

\sphinxAtStartPar
Sometimes chapters have a short preface, like a brief piece of text or a quote to set the stage
before the first scene begins.

\sphinxAtStartPar
If you add separate files for chapters and scenes, the chapter file is the perfect place to add
such text. Separating chapter and scene files also allows you to make scene files child
documents of the chapter.
\subsubsection*{Distinguishing Soft and Hard Scene Breaks}

\sphinxAtStartPar
Depending on your writing style, you may need to separate between soft and hard scene breaks
within chapters. Like for instance if you switch point\sphinxhyphen{}of\sphinxhyphen{}view character often.

\sphinxAtStartPar
In such cases you may want to use different scene headings for hard and soft scene breaks. The
\sphinxstylestrong{Build Manuscript} tool will let you define a different format for scenes using the \sphinxcode{\sphinxupquote{\#\#\#}}
and \sphinxcode{\sphinxupquote{\#\#\#!}} heading codes when you generate your manuscript. You can for instance add the
common “\sphinxcode{\sphinxupquote{* * *}}” for hard breaks and select to soft scene breaks, which will just insert an
empty paragraph in their place. See {\hyperref[\detokenize{project_manuscript:a-manuscript-settings}]{\sphinxcrossref{\DUrole{std}{\DUrole{std-ref}{Build Settings}}}}} for more details.

\sphinxAtStartPar
\DUrole{versionmodified}{\DUrole{added}{Added in version 2.4.}}


\section{Other Tools}
\label{\detokenize{int_howto:other-tools}}\subsubsection*{Convert Project to/from yWriter Format}

\sphinxAtStartPar
There is a tool available that lets you convert a \sphinxhref{http://spacejock.com/yWriter7.html}{yWriter}
project to a novelWriter project, and vice versa.

\sphinxAtStartPar
The tool is available at \sphinxhref{https://peter88213.github.io/yw2nw/}{peter88213.github.io/yw2nw}

\sphinxstepscope


\chapter{Glossary}
\label{\detokenize{int_glossary:glossary}}\label{\detokenize{int_glossary:a-glossary}}\label{\detokenize{int_glossary::doc}}\begin{description}
\sphinxlineitem{Context Menu\index{Context Menu@\spxentry{Context Menu}|spxpagem}\phantomsection\label{\detokenize{int_glossary:term-Context-Menu}}}
\sphinxAtStartPar
A context menu is a menu that pops up when you right click something in the user interface.
In novelWriter, you can often also open a context menu by pressing the keyboard shortcut
\sphinxkeyboard{\sphinxupquote{Ctrl+.}}.

\sphinxlineitem{Headings\index{Headings@\spxentry{Headings}|spxpagem}\phantomsection\label{\detokenize{int_glossary:term-Headings}}}
\sphinxAtStartPar
Each level of headings in {\hyperref[\detokenize{int_glossary:term-Novel-Documents}]{\sphinxtermref{\DUrole{xref}{\DUrole{std}{\DUrole{std-term}{Novel Documents}}}}}} have a specific meaning in terms of the
structure of the story. That is, they determine what novelWriter considers a partition, a
chapter, a scene or a text section. For {\hyperref[\detokenize{int_glossary:term-Project-Notes}]{\sphinxtermref{\DUrole{xref}{\DUrole{std}{\DUrole{std-term}{Project Notes}}}}}}, the heading levels don’t
matter. For more details on headings in novel documents, see {\hyperref[\detokenize{project_structure:a-struct-heads}]{\sphinxcrossref{\DUrole{std}{\DUrole{std-ref}{Importance of Headings}}}}}.

\sphinxlineitem{Keyword\index{Keyword@\spxentry{Keyword}|spxpagem}\phantomsection\label{\detokenize{int_glossary:term-Keyword}}}
\sphinxAtStartPar
A keyword in novelWriter is a special command you put in the text of your documents. They are
not standard Markdown, but are used in novelWriter to add information that is interpreted by
the application. For instance, keywords are used for {\hyperref[\detokenize{int_glossary:term-Tag}]{\sphinxtermref{\DUrole{xref}{\DUrole{std}{\DUrole{std-term}{tags}}}}}} and
{\hyperref[\detokenize{int_glossary:term-Reference}]{\sphinxtermref{\DUrole{xref}{\DUrole{std}{\DUrole{std-term}{references}}}}}}.

\sphinxAtStartPar
Keywords must always be on their own line, and the first character of the line must always be
the \sphinxcode{\sphinxupquote{@}} character. The keyword must also always be followed by a \sphinxcode{\sphinxupquote{:}} character, and the
values passed to the command are added after this, separated by commas.

\sphinxlineitem{Novel Documents\index{Novel Documents@\spxentry{Novel Documents}|spxpagem}\phantomsection\label{\detokenize{int_glossary:term-Novel-Documents}}}
\sphinxAtStartPar
These are documents that are created under a “Novel” {\hyperref[\detokenize{int_glossary:term-Root-Folder}]{\sphinxtermref{\DUrole{xref}{\DUrole{std}{\DUrole{std-term}{root folder}}}}}}. They behave
differently than {\hyperref[\detokenize{int_glossary:term-Project-Notes}]{\sphinxtermref{\DUrole{xref}{\DUrole{std}{\DUrole{std-term}{Project Notes}}}}}}, and have some more restrictions. For instance, they
can not exist in folders intended only for project notes. See the {\hyperref[\detokenize{project_structure:a-struct}]{\sphinxcrossref{\DUrole{std}{\DUrole{std-ref}{Novel Structure}}}}} chapter for
more details.

\sphinxlineitem{Project Index\index{Project Index@\spxentry{Project Index}|spxpagem}\phantomsection\label{\detokenize{int_glossary:term-Project-Index}}}
\sphinxAtStartPar
The project index is a record of all headings in a project, with all their meta data like
synopsis comments, {\hyperref[\detokenize{int_glossary:term-Tag}]{\sphinxtermref{\DUrole{xref}{\DUrole{std}{\DUrole{std-term}{tags}}}}}} and {\hyperref[\detokenize{int_glossary:term-Reference}]{\sphinxtermref{\DUrole{xref}{\DUrole{std}{\DUrole{std-term}{references}}}}}}. The project index is
kept up to date automatically, but can also be regenerated manually from the
\sphinxguilabel{Tools} menu or by pressing \sphinxkeyboard{\sphinxupquote{F9}}.

\sphinxlineitem{Project Notes\index{Project Notes@\spxentry{Project Notes}|spxpagem}\phantomsection\label{\detokenize{int_glossary:term-Project-Notes}}}
\sphinxAtStartPar
Project Notes are unrestricted documents that can be placed anywhere in your project. You
should not use these documents for story elements, only for notes. Project notes are the
source files used by the Tags and References system. See the {\hyperref[\detokenize{project_references:a-references}]{\sphinxcrossref{\DUrole{std}{\DUrole{std-ref}{Tags and References}}}}} chapter for
more details on how to use them.

\sphinxlineitem{Reference\index{Reference@\spxentry{Reference}|spxpagem}\phantomsection\label{\detokenize{int_glossary:term-Reference}}}
\sphinxAtStartPar
A reference is one of a set of {\hyperref[\detokenize{int_glossary:term-Keyword}]{\sphinxtermref{\DUrole{xref}{\DUrole{std}{\DUrole{std-term}{keywords}}}}}} that can be used to link to a
{\hyperref[\detokenize{int_glossary:term-Tag}]{\sphinxtermref{\DUrole{xref}{\DUrole{std}{\DUrole{std-term}{tag}}}}}} in another document. The reference keywords are specific to the different
{\hyperref[\detokenize{int_glossary:term-Root-Folder}]{\sphinxtermref{\DUrole{xref}{\DUrole{std}{\DUrole{std-term}{root folder}}}}}} types. A full overview is available in the {\hyperref[\detokenize{project_references:a-references}]{\sphinxcrossref{\DUrole{std}{\DUrole{std-ref}{Tags and References}}}}} chapter.

\sphinxlineitem{Root Folder\index{Root Folder@\spxentry{Root Folder}|spxpagem}\phantomsection\label{\detokenize{int_glossary:term-Root-Folder}}}
\sphinxAtStartPar
A “Root Folder” is a top level folder of the project tree in novelWriter. Each type of root
folder has a specific icon to identify it. For an overview of available root folder types,
see {\hyperref[\detokenize{project_overview:a-proj-roots}]{\sphinxcrossref{\DUrole{std}{\DUrole{std-ref}{Project Structure}}}}}.

\sphinxlineitem{Tag\index{Tag@\spxentry{Tag}|spxpagem}\phantomsection\label{\detokenize{int_glossary:term-Tag}}}
\sphinxAtStartPar
A tag is a user defined value assigned as a tag to a section of your {\hyperref[\detokenize{int_glossary:term-Project-Notes}]{\sphinxtermref{\DUrole{xref}{\DUrole{std}{\DUrole{std-term}{Project Notes}}}}}}.
It is optional, and can be defined once per heading. It is set using the {\hyperref[\detokenize{int_glossary:term-Keyword}]{\sphinxtermref{\DUrole{xref}{\DUrole{std}{\DUrole{std-term}{keyword}}}}}}
syntax \sphinxcode{\sphinxupquote{@tag: value}}, where \sphinxcode{\sphinxupquote{value}} is the user defined part. Each tag can be referenced
in another file using one of the {\hyperref[\detokenize{int_glossary:term-Reference}]{\sphinxtermref{\DUrole{xref}{\DUrole{std}{\DUrole{std-term}{reference}}}}}} keywords. See {\hyperref[\detokenize{project_references:a-references}]{\sphinxcrossref{\DUrole{std}{\DUrole{std-ref}{Tags and References}}}}} chapter
for more details.

\end{description}

\sphinxstepscope


\chapter{How it Works}
\label{\detokenize{usage_breakdown:how-it-works}}\label{\detokenize{usage_breakdown:a-breakdown}}\label{\detokenize{usage_breakdown::doc}}
\sphinxAtStartPar
The main features of novelWriter are listed in the {\hyperref[\detokenize{int_introduction:a-intro}]{\sphinxcrossref{\DUrole{std}{\DUrole{std-ref}{Overview}}}}} chapter. In this chapter, we go
into some more details on how they are implemented. This is intended as an overview. Later on in
this documentation these features will be covered in more detail.


\section{GUI Layout and Design}
\label{\detokenize{usage_breakdown:gui-layout-and-design}}\label{\detokenize{usage_breakdown:a-breakdown-design}}
\sphinxAtStartPar
The user interface of novelWriter is intended to be as minimalistic as practically possible, while
at the same time provide useful features needed for writing a novel.

\sphinxAtStartPar
The main window does not by default have an editor toolbar like many other applications do. This
reduces clutter, and since the documents are formatted with style tags, it is not needed most of
the time. Still, a small formatting toolbar can be popped out by clicking the left\sphinxhyphen{}most button in
the header of the document editor. It gives quick access to standard formatting codes.

\sphinxAtStartPar
Most formatting features supported are available through convenient keyboard shortcuts. They are
also available in the main menu under \sphinxstylestrong{Format}, so you don’t have to look up formatting codes
every time you need them. For reference, a list of all shortcuts can be found in the {\hyperref[\detokenize{usage_shortcuts:a-kb}]{\sphinxcrossref{\DUrole{std}{\DUrole{std-ref}{Keyboard Shortcuts}}}}}
chapter.

\begin{sphinxadmonition}{note}{Note:}
\sphinxAtStartPar
novelWriter is not intended to be a full office type word processor. It doesn’t support images,
links, tables, and other complex structures and objects often needed for such documents.
Formatting is limited to headings, in\sphinxhyphen{}line basic text formats, text alignment, and a few other
simple features.
\end{sphinxadmonition}

\sphinxAtStartPar
On the left side of the main window, you will find a sidebar. This bar has buttons for the standard
views you can switch between, a quick link to the \sphinxstylestrong{Build Manuscript} tool, and a set of
project\sphinxhyphen{}related tools and quick access to settings at the bottom.

\sphinxAtStartPar
\DUrole{versionmodified}{\DUrole{added}{Added in version 2.2: }}A number of new formatting options were added in 2.2 to allow for some special formatting cases.
At the same time, a small formatting toolbar was added to the editor. It is hidden by default,
but can be opened by pressing the button in the top right corner of the editor header.


\subsection{Project Tree and Editor View}
\label{\detokenize{usage_breakdown:project-tree-and-editor-view}}
\begin{figure}[htbp]
\centering
\capstart

\noindent\sphinxincludegraphics{{fig_project_tree_view}.png}
\caption{A screenshot of the Project Tree and Editor View.}\label{\detokenize{usage_breakdown:id1}}\end{figure}

\sphinxAtStartPar
When the application is in \sphinxstylestrong{Project Tree View} mode, the main work area of the main window is
split in two, or optionally three, panels. The left\sphinxhyphen{}most panel contains the project tree and all
the documents in your project. The second panel is the document editor.

\sphinxAtStartPar
An optional third panel on the right side contains a document viewer which can view any document in
your project independently of what is open in the document editor. This panel is not intended as a
preview window, although you can use it for this purpose if you wish. For instance if you need to
check that the formatting tags behave as you expect. However, the main purpose of the viewer is for
viewing your notes next to your editor while you’re writing.

\sphinxAtStartPar
The editor also has a \sphinxstylestrong{Focus Mode} you can toggle either from the menu, from the icon in the
editor’s header, or by pressing \sphinxkeyboard{\sphinxupquote{F8}}. When \sphinxstylestrong{Focus Mode} is enabled, all the user interface
elements other than the document editor itself are hidden away.


\subsection{Novel View and Editor View}
\label{\detokenize{usage_breakdown:novel-view-and-editor-view}}
\begin{figure}[htbp]
\centering
\capstart

\noindent\sphinxincludegraphics{{fig_novel_tree_view}.png}
\caption{A screenshot of the Novel Tree and Editor View.}\label{\detokenize{usage_breakdown:id2}}\end{figure}

\sphinxAtStartPar
When the application is in \sphinxstylestrong{Novel Tree View} mode, the project tree is replaced by an overview of
your novel structure for a specific Novel {\hyperref[\detokenize{int_glossary:term-Root-Folder}]{\sphinxtermref{\DUrole{xref}{\DUrole{std}{\DUrole{std-term}{root folder}}}}}}. Instead of showing individual
documents, the tree now shows all headings of your novel text. This includes multiple headings
within the same document.

\sphinxAtStartPar
Each heading is indented according to the heading level, not its parent/child relationship to other
elements of your project. You can open and edit your novel documents from this view as well. All
headings contained in the currently open document should be highlighted in the view to indicate
which ones belong together in the same document.

\sphinxAtStartPar
If you have multiple \sphinxstylestrong{Novel} type root folders, the header of the novel view becomes a dropdown
box. You can then switch between them by clicking the \sphinxguilabel{Outline of …} text. You can also
click the novel icon button next to it.

\sphinxAtStartPar
Generally, the novel view should update when you make changes to the novel structure, including
edits of the current document in the editor. The information is only updated when the automatic
save of the document is triggered, or you manually press \sphinxkeyboard{\sphinxupquote{Ctrl+S}} to save changes. (You can
adjust the auto\sphinxhyphen{}save interval in \sphinxstylestrong{Preferences}.) You can also regenerate the whole novel view by
pressing the refresh button in the novel view header.

\sphinxAtStartPar
It is possible to show an optional third column in the novel view. The settings are available from
the menu button in the toolbar.

\sphinxAtStartPar
If you click the triangular icon to the right of each item, a tooltip will pop out showing all the
meta data collected for that heading.


\subsection{Novel Outline View}
\label{\detokenize{usage_breakdown:novel-outline-view}}
\begin{figure}[htbp]
\centering
\capstart

\noindent\sphinxincludegraphics{{fig_outline_view}.png}
\caption{A screenshot of the Novel Outline View.}\label{\detokenize{usage_breakdown:id3}}\end{figure}

\sphinxAtStartPar
When the application is in \sphinxstylestrong{Novel Outline View} mode, the tree, editor and viewer will be
replaced by a large table that shows the entire novel structure with all the tags and references
listed. Pretty much all collected meta data is available here in different columns.

\sphinxAtStartPar
You can select which novel root folder to display from the dropdown box, and you can select which
columns to show or hide from the menu button. You can also rearrange the columns by drag and drop.
The app will remember your column order and size between sessions, and for each individual project.


\subsection{Colour Themes}
\label{\detokenize{usage_breakdown:colour-themes}}
\sphinxAtStartPar
By default, novelWriter will use the colour theme provided by the Qt library, which is determined
by the \sphinxhref{https://doc.qt.io/qt-6/gallery.html}{Fusion} style setting. You can also choose between a standard dark and light theme that have
neutral colours from \sphinxstylestrong{Preferences}.

\sphinxAtStartPar
If you wish, you \sphinxstyleemphasis{can} create your own colour themes, and even have them added to the application.
See {\hyperref[\detokenize{more_customise:a-custom-theme}]{\sphinxcrossref{\DUrole{std}{\DUrole{std-ref}{Syntax and GUI Themes}}}}} for more details.

\sphinxAtStartPar
Switching the GUI colour theme does not affect the colours of the editor and viewer. They have
separate colour selectable from the “Document colour theme” setting in \sphinxstylestrong{Preferences}. They are
separated because there are a lot more options to choose from for the editor and viewer.

\begin{sphinxadmonition}{note}{Note:}
\sphinxAtStartPar
If you switch between light and dark mode on the GUI, you should also switch editor theme to
match, otherwise icons may be hard to see in the editor and viewer.
\end{sphinxadmonition}


\subsection{Project Search}
\label{\detokenize{usage_breakdown:project-search}}
\sphinxAtStartPar
A global search tool is available from the side bar. It allows you to search through your entire
project. The tool does not provide a replace feature. There is a search and replace tool available
in the document editor that acts on the open document.

\sphinxAtStartPar
\DUrole{versionmodified}{\DUrole{added}{Added in version 2.4.}}


\subsection{Switching Focus}
\label{\detokenize{usage_breakdown:switching-focus}}
\sphinxAtStartPar
If the project or novel view does not have focus, pressing \sphinxkeyboard{\sphinxupquote{Ctrl+T}} switches focus to
whichever of the two is visible. If one of them already has focus, the key press will switch
between them instead.

\sphinxAtStartPar
Likewise, pressing \sphinxkeyboard{\sphinxupquote{Ctrl+E}} with switch focus to the document editor or viewer, or if any of
them already have focus, it will switch focus between them,

\sphinxAtStartPar
These two shortcuts makes it possible to jump between all these GUI elements without having to
reach for the mouse or touchpad.


\section{Project Layout}
\label{\detokenize{usage_breakdown:project-layout}}\label{\detokenize{usage_breakdown:a-breakdown-project}}
\sphinxAtStartPar
This is a brief introduction to how you structure your writing projects. All of this will be
covered in more detail later.

\sphinxAtStartPar
The main point of novelWriter is that you are free to organise your project documents as you wish
into sub\sphinxhyphen{}folders or sub\sphinxhyphen{}documents, and split the text between these documents in whatever way suits
you. All that matters to novelWriter is the linear order the documents appear at in the project
tree (top to bottom). The chapters, scenes and sections of the novel are determined by the headings
within those documents.

\begin{figure}[htbp]
\centering
\capstart

\noindent\sphinxincludegraphics{{fig_header_levels}.png}
\caption{An illustration of how heading levels correspond to the novel structure.}\label{\detokenize{usage_breakdown:id4}}\end{figure}

\sphinxAtStartPar
The four heading levels, \sphinxstylestrong{Level 1} to \sphinxstylestrong{Level 4}, are treated as follows:
\begin{itemize}
\item {} 
\sphinxAtStartPar
\sphinxstylestrong{Level 1} is used for the novel title, and for partitions.

\item {} 
\sphinxAtStartPar
\sphinxstylestrong{Level 2} is used for chapter tiles.

\item {} 
\sphinxAtStartPar
\sphinxstylestrong{Level 3} is used for scene titles \textendash{} optionally replaced by separators.

\item {} 
\sphinxAtStartPar
\sphinxstylestrong{Level 4} is for section titles within scenes, if such granularity is needed.

\end{itemize}

\sphinxAtStartPar
The project tree will select an icon for the document based on the first heading in it.

\sphinxAtStartPar
This heading level structure is only taken into account for {\hyperref[\detokenize{int_glossary:term-Novel-Documents}]{\sphinxtermref{\DUrole{xref}{\DUrole{std}{\DUrole{std-term}{novel documents}}}}}}. For
{\hyperref[\detokenize{int_glossary:term-Project-Notes}]{\sphinxtermref{\DUrole{xref}{\DUrole{std}{\DUrole{std-term}{project notes}}}}}}, the heading levels have no structural meaning, and you are free to use them
however you want. See {\hyperref[\detokenize{project_structure:a-struct}]{\sphinxcrossref{\DUrole{std}{\DUrole{std-ref}{Novel Structure}}}}} and {\hyperref[\detokenize{project_references:a-references}]{\sphinxcrossref{\DUrole{std}{\DUrole{std-ref}{Tags and References}}}}} for more details.

\sphinxAtStartPar
\DUrole{versionmodified}{\DUrole{added}{Added in version 2.0: }}You can add documents as child items of other documents. This is often more useful than adding
folders, since you anyway may want to have the chapter heading in a separate document from your
individual scene documents so that you can rearrange scene documents freely without affecting
chapter placement.


\section{Building a Manuscript}
\label{\detokenize{usage_breakdown:building-a-manuscript}}\label{\detokenize{usage_breakdown:a-breakdown-export}}
\sphinxAtStartPar
The project can at any time be assembled into a range of different formats through the
\sphinxstylestrong{Build Manuscript} tool. Natively, novelWriter supports \sphinxhref{https://en.wikipedia.org/wiki/OpenDocument}{Open Document}, HTML5, and
various flavours of Markdown.

\sphinxAtStartPar
The HTML5 format is suitable for conversion by a number of other tools like \sphinxhref{https://pandoc.org/}{Pandoc}, or for
importing into word processors if the Open Document format isn’t suitable. The Open Document format
is supported by most office type applications. In addition, printing is also possible. Print to PDF
is available from the print dialog.

\sphinxAtStartPar
For advanced processing, you can export the content of the project to a JSON file. This is useful
if you want to write your own custom processing script in for instance Python, as the entire novel
can be read into a Python dictionary with a couple of lines of code. The JSON file can be populated
with either HTML formatted text, or with the raw text as typed into the novel documents.

\sphinxAtStartPar
See {\hyperref[\detokenize{project_manuscript:a-manuscript}]{\sphinxcrossref{\DUrole{std}{\DUrole{std-ref}{Building the Manuscript}}}}} for more details.

\sphinxAtStartPar
\DUrole{versionmodified}{\DUrole{added}{Added in version 2.1: }}You can now define multiple build definitions in the \sphinxstylestrong{Build Manuscript} tool. This allows you
to define specific settings for various types of draft documents, outline documents, and
manuscript formats. See {\hyperref[\detokenize{project_manuscript:a-manuscript}]{\sphinxcrossref{\DUrole{std}{\DUrole{std-ref}{Building the Manuscript}}}}} for more details.


\section{Project Storage}
\label{\detokenize{usage_breakdown:project-storage}}\label{\detokenize{usage_breakdown:a-breakdown-storage}}
\sphinxAtStartPar
The files of a novelWriter project are stored in a dedicated project folder. The project structure
is kept in a file at the root of this folder called \sphinxcode{\sphinxupquote{nwProject.nwx}}. All the document files and
associated meta data is stored in other folders below the project folder. For more technical
details about what all the files mean and how they’re organised, see the {\hyperref[\detokenize{tech_storage:a-storage}]{\sphinxcrossref{\DUrole{std}{\DUrole{std-ref}{How Data is Stored}}}}} section.

\sphinxAtStartPar
This way of storing data was chosen for several reasons.

\sphinxAtStartPar
Firstly, all the text you add to your project is saved directly to your project folder in separate
files. Only the project structure and the text you are currently editing is stored in memory at any
given time, which means there is a smaller risk of losing data if the application or your computer
crashes.

\sphinxAtStartPar
Secondly, having multiple small files means it is very easy to synchronise them between computers
with standard file synchronisation tools.

\sphinxAtStartPar
Thirdly, if you use version control software to track the changes to your project, the file formats
used for the files are well suited. All the JSON documents have line breaks and indents as well,
which makes it easier to track them with version control software.

\begin{sphinxadmonition}{note}{Note:}
\sphinxAtStartPar
Since novelWriter has to keep track of a bunch of files and folders when a project is open, it
may not run well on some virtual file systems. A file or folder must be accessible with exactly
the path it was saved or created with. An example where this is not the case is the way Google
Drive is mapped on Linux Gnome desktops using gvfs/gio.
\end{sphinxadmonition}

\begin{sphinxadmonition}{caution}{Caution:}
\sphinxAtStartPar
You should not add additional files to the project folder yourself. Nor should you, as a rule,
manually edit files within it. If you really must manually edit the text files, e.g. with some
automated task you want to perform, you need to rebuild the {\hyperref[\detokenize{int_glossary:term-Project-Index}]{\sphinxtermref{\DUrole{xref}{\DUrole{std}{\DUrole{std-term}{Project Index}}}}}} when you open
the project again.

\sphinxAtStartPar
Editing text files in the \sphinxcode{\sphinxupquote{content}} folder is less risky as these are just plain text. Editing
the main project XML file, however, may make the project file unreadable and you may crash
novelWriter and lose project structure information and project settings.
\end{sphinxadmonition}

\sphinxstepscope


\chapter{Project Views}
\label{\detokenize{usage_project:project-views}}\label{\detokenize{usage_project:a-ui-project}}\label{\detokenize{usage_project::doc}}
\sphinxAtStartPar
This chapter covers in more detail the different project views available in novelWriter.

\begin{figure}[htbp]
\centering
\capstart

\noindent\sphinxincludegraphics{{fig_project_tree_detailed}.png}
\caption{The \sphinxstylestrong{Project Content} tree as it appears when loading a sample project.}\label{\detokenize{usage_project:id1}}\end{figure}


\section{The Project Tree}
\label{\detokenize{usage_project:the-project-tree}}\label{\detokenize{usage_project:a-ui-tree}}
\sphinxAtStartPar
The main window contains a project tree in the left\sphinxhyphen{}most panel. It shows the entire structure of
the project, and has four columns.
\begin{description}
\sphinxlineitem{\sphinxstylestrong{Column 1}}
\sphinxAtStartPar
The first column shows the icon and label of each folder, document, or note in your project. The
label is not the same as the heading title you set inside the document. However, the document’s
label will appear in the header above the document text itself so you know where in the project
an open document belongs. The icon is selected based on the type of item, and for novel
documents, the level of the first heading in the document text.

\sphinxlineitem{\sphinxstylestrong{Column 2}}
\sphinxAtStartPar
The second column shows the word count of the document, or the sum of words of the child items
for folders and documents with sub\sphinxhyphen{}documents. If the counts seem incorrect, they can be updated
by rebuilding the {\hyperref[\detokenize{int_glossary:term-Project-Index}]{\sphinxtermref{\DUrole{xref}{\DUrole{std}{\DUrole{std-term}{project index}}}}}} from the \sphinxstylestrong{Tools} menu, or by pressing \sphinxkeyboard{\sphinxupquote{F9}}.

\sphinxlineitem{\sphinxstylestrong{Column 3}}
\sphinxAtStartPar
The third column indicates whether the document is considered active or inactive in the project.
You can use this flag to indicate that a document is still in the project, but should not be
considered an active part of it. When you run the \sphinxstylestrong{Build Manuscript} tool, you can include or
exclude documents based on this flag. You can change this value from the right\sphinxhyphen{}click
{\hyperref[\detokenize{int_glossary:term-Context-Menu}]{\sphinxtermref{\DUrole{xref}{\DUrole{std}{\DUrole{std-term}{context menu}}}}}}.

\sphinxlineitem{\sphinxstylestrong{Column 4}}
\sphinxAtStartPar
The fourth column shows the user\sphinxhyphen{}defined status or importance labels you’ve assigned to each
project item. See {\hyperref[\detokenize{usage_project:a-ui-tree-status}]{\sphinxcrossref{\DUrole{std}{\DUrole{std-ref}{Document Importance and Status}}}}} for more details on how to uses these labels. You can
select these labels from the {\hyperref[\detokenize{int_glossary:term-Context-Menu}]{\sphinxtermref{\DUrole{xref}{\DUrole{std}{\DUrole{std-term}{context menu}}}}}}, and define them in \sphinxstylestrong{Project Settings}.

\end{description}

\sphinxAtStartPar
Right\sphinxhyphen{}clicking an item in the project tree will open a context menu under the cursor, displaying
a selection of actions that can be performed on the selected item.

\sphinxAtStartPar
At the top of the project tree, you will find a set of buttons.
\begin{itemize}
\item {} 
\sphinxAtStartPar
The first button is a quick links button that will show you a dropdown menu of all the
{\hyperref[\detokenize{int_glossary:term-Root-Folder}]{\sphinxtermref{\DUrole{xref}{\DUrole{std}{\DUrole{std-term}{root folders}}}}}} in your project. Selecting one will move to that position in
the tree. You can also activate this menu by pressing \sphinxkeyboard{\sphinxupquote{Ctrl+L}}.

\item {} 
\sphinxAtStartPar
The next two buttons can be used to move items up and down in the project tree. This is the only
way to move root folders.

\item {} 
\sphinxAtStartPar
The next button opens a dropdown menu for adding new items to the tree. This includes root
folders and template documents. You can also activate this dropdown menu by pressing
\sphinxkeyboard{\sphinxupquote{Ctrl+N}}.

\item {} 
\sphinxAtStartPar
The last button is a menu of further actions you can apply to the project tree.

\end{itemize}

\sphinxAtStartPar
Below the project tree you will find a small details panel showing the full information of the
currently selected item. This panel also includes the latest paragraph and character counts in
addition to the word count.

\begin{sphinxadmonition}{tip}{Tip:}
\sphinxAtStartPar
If you want to set the label of a document to be the same as a header within it, you can
right\sphinxhyphen{}click a header in the document when it is open in the editor and select
\sphinxguilabel{Set as Document Name} from the context menu.
\end{sphinxadmonition}


\subsection{Splitting and Merging Documents}
\label{\detokenize{usage_project:splitting-and-merging-documents}}\label{\detokenize{usage_project:a-ui-tree-split-merge}}
\sphinxAtStartPar
Under the \sphinxstylestrong{Transform} submenu in the context menu of an item in the project tree, you will find
several options on how to change a document or folder. This includes changing between document and
note, but also splitting them into multiple documents, or merging child items into a single
document.


\subsubsection{Splitting Documents}
\label{\detokenize{usage_project:splitting-documents}}
\begin{figure}[htbp]
\centering
\capstart

\noindent\sphinxincludegraphics{{fig_project_split_tool}.png}
\caption{The \sphinxstylestrong{Split Document} dialog.}\label{\detokenize{usage_project:id2}}\end{figure}

\sphinxAtStartPar
The \sphinxstylestrong{Split Document by Headings} option will open a dialog that allows you to split the selected
document into multiple new documents based on the headings it contains. You can select at which
heading level the split is to be performed from the dropdown box. The list box will preview which
headings will be split into new documents.

\sphinxAtStartPar
You are given the option to create a folder for these new documents, and whether or not to create a
hierarchy of documents. That is, put sections under scenes, and scenes under chapters.

\sphinxAtStartPar
The source document \sphinxstylestrong{is not} deleted in the process, but you have the option to let the tool move
the source document to the \sphinxguilabel{Trash} folder.


\subsubsection{Merging Documents}
\label{\detokenize{usage_project:merging-documents}}
\begin{figure}[htbp]
\centering
\capstart

\noindent\sphinxincludegraphics{{fig_project_merge_tool}.png}
\caption{The \sphinxstylestrong{Merge Documents} dialog.}\label{\detokenize{usage_project:id3}}\end{figure}

\sphinxAtStartPar
You have two options for merging documents that are child elements of another document. You can
either \sphinxstylestrong{Merge Child Items into Self} and \sphinxstylestrong{Merge Child Items into New}. The first option will
pull all content of child items and merge them into the parent document, while the second option
will create a new document in the process.

\sphinxAtStartPar
When merging documents in a folder, you only have the latter process is possible, so only the
choice \sphinxstylestrong{Merge Documents in Folder} is available.

\sphinxAtStartPar
In either case, the \sphinxstylestrong{Merge Documents} dialog will let you exclude documents you don’t want to
include, and it also lets you reorder them if you wish.


\subsection{Document Importance and Status}
\label{\detokenize{usage_project:document-importance-and-status}}\label{\detokenize{usage_project:a-ui-tree-status}}
\sphinxAtStartPar
Each document or folder in your project can have either a “Status” or “Importance” flag set. These
are flags that you control and define yourself, and novelWriter doesn’t use them for anything. To
modify the labels, go to their respective tabs in \sphinxstylestrong{Project Settings}.

\sphinxAtStartPar
The “Status” flag is intended to tag a {\hyperref[\detokenize{int_glossary:term-Novel-Documents}]{\sphinxtermref{\DUrole{xref}{\DUrole{std}{\DUrole{std-term}{novel document}}}}}} as for instance a
draft or as completed, and the “Importance” flag is intended to tag character notes, or other
{\hyperref[\detokenize{int_glossary:term-Project-Notes}]{\sphinxtermref{\DUrole{xref}{\DUrole{std}{\DUrole{std-term}{project notes}}}}}}, as for instance a main, major, or minor character or story element.

\sphinxAtStartPar
Whether a document uses a “Status” or “Importance” flag depends on which {\hyperref[\detokenize{int_glossary:term-Root-Folder}]{\sphinxtermref{\DUrole{xref}{\DUrole{std}{\DUrole{std-term}{root folder}}}}}} it
lives in. If it’s in a \sphinxstylestrong{Novel} type folder, it uses the “Status” flag, otherwise it uses an
“Importance” flag.


\subsection{Project Tree Drag \& Drop}
\label{\detokenize{usage_project:project-tree-drag-drop}}\label{\detokenize{usage_project:a-ui-tree-dnd}}
\sphinxAtStartPar
The project tree allows drag \& drop to a certain extent to allow you to reorder your documents and
folders. Moving a document in the project tree will affect the text’s position when you assemble
your manuscript in the \sphinxstylestrong{Manuscript Build} tool.

\sphinxAtStartPar
\DUrole{versionmodified}{\DUrole{added}{Added in version 2.2: }}You can now select multiple items in the project tree by holding down the \sphinxkeyboard{\sphinxupquote{Ctrl}} or
\sphinxkeyboard{\sphinxupquote{Shift}} key while selecting items.

\sphinxAtStartPar
You can drag and drop documents and regular folders, but not root folders. If you select multiple
items, they can only be dragged and dropped if they are siblings. That is, they have the same
parent item in the project. This limitation is due to the way drag and drop is implemented in the
user interface framework novelWriter is built upon.

\sphinxAtStartPar
Documents and their folders can be rearranged freely within their root folders. If you move a
\sphinxstylestrong{Novel Document} out of a \sphinxstylestrong{Novel} folder, it will be converted to a \sphinxstylestrong{Project Note}. Notes can
be moved freely between all root folders, but keep in mind that if you move a note into a \sphinxstylestrong{Novel}
type root folder, its “Importance” setting will be switched with a “Status” setting. See
{\hyperref[\detokenize{usage_project:a-ui-tree-status}]{\sphinxcrossref{\DUrole{std}{\DUrole{std-ref}{Document Importance and Status}}}}}. The old value will not be overwritten though, and should be restored if
you move it back at some point.

\sphinxAtStartPar
Root folders in the project tree cannot be dragged and dropped at all. If you want to reorder them,
you can move them up or down with respect to each other from the arrow buttons at the top of the
project tree, or by pressing \sphinxkeyboard{\sphinxupquote{Ctrl+Up}} or \sphinxkeyboard{\sphinxupquote{Ctrl+Down}} when they are selected.


\section{The Novel Tree View}
\label{\detokenize{usage_project:the-novel-tree-view}}\label{\detokenize{usage_project:a-ui-tree-novel}}
\begin{figure}[htbp]
\centering
\capstart

\noindent\sphinxincludegraphics{{fig_novel_tree_view}.png}
\caption{A screenshot of the Novel Tree View.}\label{\detokenize{usage_project:id4}}\end{figure}

\sphinxAtStartPar
An alternative way to view the project structure is the novel view. You can switch to this view by
selecting the \sphinxguilabel{Novel View} button in the sidebar. This view is a simplified version of
the view in the \sphinxstylestrong{Outline View}. It is convenient when you want to browse the structure of the
story itself rather than the document files.

\begin{sphinxadmonition}{note}{Note:}
\sphinxAtStartPar
You cannot reorganise the entries in the novel view, or add any new documents, as that would
imply restructuring the content of the document files themselves. Any such editing must be done
in the project tree. However, you can add new headings to existing documents, or change
references, which will be updated in this view when the document is saved.
\end{sphinxadmonition}


\section{The Novel Outline View}
\label{\detokenize{usage_project:the-novel-outline-view}}\label{\detokenize{usage_project:a-ui-outline}}
\begin{figure}[htbp]
\centering
\capstart

\noindent\sphinxincludegraphics{{fig_outline_view}.png}
\caption{A screenshot of the Novel Outline View.}\label{\detokenize{usage_project:id5}}\end{figure}

\sphinxAtStartPar
The project’s \sphinxstylestrong{Novel Outline View} is available as another view option from the sidebar. The
outline provides an overview of the novel structure, displaying a tree hierarchy of the elements of
the novel, that is, the level 1 to 4 headings representing partitions, chapters, scenes and
sections.

\sphinxAtStartPar
The document containing the heading can also be displayed as a separate column, as well as the line
number where the heading is defined. Double\sphinxhyphen{}clicking an entry will open the corresponding document
in the editor and switch to \sphinxstylestrong{Project Tree View} mode.

\sphinxAtStartPar
You can select which novel folder to display from the dropdown menu. You can optionally also choose
to show a combination of all novel folders.

\begin{sphinxadmonition}{note}{Note:}
\sphinxAtStartPar
Since the internal structure of the novel does not depend directly on the folder and document
structure of the project tree, this view will not necessarily look the same, depending on how
you choose to organise your documents. See the {\hyperref[\detokenize{project_structure:a-struct}]{\sphinxcrossref{\DUrole{std}{\DUrole{std-ref}{Novel Structure}}}}} page for more details.
\end{sphinxadmonition}

\sphinxAtStartPar
Various meta data and information extracted from {\hyperref[\detokenize{int_glossary:term-Tag}]{\sphinxtermref{\DUrole{xref}{\DUrole{std}{\DUrole{std-term}{tags}}}}}} can be displayed in columns in
the outline. A default set of such columns is visible, but you can turn on or off more columns from
the menu button in the toolbar. The order of the columns can also be rearranged by dragging them to
a different position. You column settings are saved between sessions on a per\sphinxhyphen{}project basis.

\begin{sphinxadmonition}{note}{Note:}
\sphinxAtStartPar
The \sphinxstylestrong{Title} column cannot be disabled or moved.
\end{sphinxadmonition}

\sphinxAtStartPar
The information viewed in the outline is based on the {\hyperref[\detokenize{int_glossary:term-Project-Index}]{\sphinxtermref{\DUrole{xref}{\DUrole{std}{\DUrole{std-term}{project index}}}}}}. While novelWriter does
its best to keep the index up to date when contents change, you can always rebuild it manually by
pressing \sphinxkeyboard{\sphinxupquote{F9}} if something isn’t right.

\sphinxAtStartPar
The outline view itself can be regenerated by pressing the refresh button. By default, the content
is refreshed each time you switch to this view.

\sphinxAtStartPar
The \sphinxstylestrong{Synopsis} column of the outline view takes its information from a specially formatted
comment. See {\hyperref[\detokenize{usage_format:a-fmt-comm}]{\sphinxcrossref{\DUrole{std}{\DUrole{std-ref}{Comments and Synopsis}}}}}.

\sphinxstepscope


\chapter{The Editor and Viewer}
\label{\detokenize{usage_writing:the-editor-and-viewer}}\label{\detokenize{usage_writing:a-ui-writing}}\label{\detokenize{usage_writing::doc}}
\sphinxAtStartPar
This chapter covers in more detail how the document editor and viewer panels work.


\section{Editing a Document}
\label{\detokenize{usage_writing:editing-a-document}}\label{\detokenize{usage_writing:a-ui-edit}}
\begin{figure}[htbp]
\centering
\capstart

\noindent\sphinxincludegraphics{{fig_editor}.png}
\caption{A screenshot of the Document Editor panel.}\label{\detokenize{usage_writing:id1}}\end{figure}

\sphinxAtStartPar
To edit a document, double\sphinxhyphen{}click it in the project tree, or press the \sphinxkeyboard{\sphinxupquote{Return}} key while
having it selected. This will open the document in the document editor. The editor uses a
Markdown\sphinxhyphen{}like syntax for some features, and a novelWriter\sphinxhyphen{}specific syntax for others. The syntax
format is described in the {\hyperref[\detokenize{usage_format:a-fmt}]{\sphinxcrossref{\DUrole{std}{\DUrole{std-ref}{Formatting Your Text}}}}} chapter.

\sphinxAtStartPar
The editor has a maximise button, which toggles the \sphinxstylestrong{Focus Mode}, and a close button in the
top\textendash{}right corner. On the top\textendash{}left side you will find a tools button that opens a toolbar with a
few buttons for applying text formatting, a drop down menu for navigating between headings, and a
search button to open the search dialog.

\sphinxAtStartPar
Both the document editor and viewer will show the label of the currently open document in the
header at the top of the edit or view panel. Optionally, the full project path to the document can
be shown. This can be set in \sphinxstylestrong{Preferences}.

\begin{sphinxadmonition}{tip}{Tip:}
\sphinxAtStartPar
Clicking on the document title bar will select the document in the project tree and thus reveal
its location there, making it easier to find in a large project.
\end{sphinxadmonition}

\sphinxAtStartPar
Any {\hyperref[\detokenize{int_glossary:term-Reference}]{\sphinxtermref{\DUrole{xref}{\DUrole{std}{\DUrole{std-term}{references}}}}}} in the editor can be opened in the viewer by moving the cursor to
the label and pressing \sphinxkeyboard{\sphinxupquote{Ctrl+Return}}. You can also control\sphinxhyphen{}click them with your mouse.


\subsection{Spell Checking}
\label{\detokenize{usage_writing:spell-checking}}
\sphinxAtStartPar
A third party library called Enchant is used for spell checking in the editor. The controls for
spell checking can be found in the \sphinxstylestrong{Tools} menu. You can also set spell checking language in
\sphinxstylestrong{Project Settings}.

\sphinxAtStartPar
This spell checking library comes with support for custom words that you can add by selecting
“Add Word to Dictionary” from the context menu when a word is highlighted by the spell checker as
misspelled. The custom words are managed on a per\sphinxhyphen{}project basis, and can the list of words can be
edited from the \sphinxstylestrong{Project Word List} tool available from the \sphinxstylestrong{Tools} menu.

\begin{sphinxadmonition}{note}{Note:}
\sphinxAtStartPar
Generally, spell checking dictionaries are collected from your operating system, but on Windows
they are not. See {\hyperref[\detokenize{more_customise:a-custom-dict}]{\sphinxcrossref{\DUrole{std}{\DUrole{std-ref}{Spell Check Dictionaries}}}}} for how to add spell checking dictionaries on Windows.
\end{sphinxadmonition}


\subsection{Editor Auto\sphinxhyphen{}Completer}
\label{\detokenize{usage_writing:editor-auto-completer}}
\sphinxAtStartPar
If you type the character \sphinxcode{\sphinxupquote{@}} on a new line, a pop\sphinxhyphen{}up menu will appear showing the different
available keywords. The list will shorten as you type. Once a keyword command has been selected or
typed, the editor may suggest further options based on your project content. See
{\hyperref[\detokenize{project_references:a-references-completer}]{\sphinxcrossref{\DUrole{std}{\DUrole{std-ref}{The References Auto\sphinxhyphen{}Completer}}}}} for more details.

\sphinxAtStartPar
\DUrole{versionmodified}{\DUrole{added}{Added in version 2.2.}}


\section{Viewing a Document}
\label{\detokenize{usage_writing:viewing-a-document}}\label{\detokenize{usage_writing:a-ui-view}}
\begin{figure}[htbp]
\centering
\capstart

\noindent\sphinxincludegraphics{{fig_viewer}.png}
\caption{A screenshot of the Document Viewer panel.}\label{\detokenize{usage_writing:id2}}\end{figure}

\sphinxAtStartPar
Any document in the project tree can also be viewed in parallel in a right hand side document
viewer. To view a document, press \sphinxkeyboard{\sphinxupquote{Ctrl+R}}, or select \sphinxstylestrong{View Document} in the menu or context
menu. If you have a middle mouse button, middle\sphinxhyphen{}clicking on the document will also open it in the
viewer.

\sphinxAtStartPar
The document viewed does not have to be the same document as the one currently being edited.
However, If you \sphinxstyleemphasis{are} viewing the same document, pressing \sphinxkeyboard{\sphinxupquote{Ctrl+R}} again will update the
document with your latest changes. You can also press the reload button in the top\textendash{}right corner of
the viewer panel, next to the close button, to achieve the same thing.

\sphinxAtStartPar
In the viewer {\hyperref[\detokenize{int_glossary:term-Reference}]{\sphinxtermref{\DUrole{xref}{\DUrole{std}{\DUrole{std-term}{references}}}}}} become clickable links. Clicking them will replace the
content of the viewer with the content of the document the reference points to.

\sphinxAtStartPar
The document viewer keeps a history of viewed documents, which you can navigate with the arrow
buttons in the top\textendash{}left corner of the viewer. If your mouse has backward and forward navigation
buttons, these can be used as well. They work just like the backward and forward features in a
browser. The left\sphinxhyphen{}most button is a dropdown menu for quickly navigation between headings in the
document.

\sphinxAtStartPar
At the bottom of the view panel there is a \sphinxstylestrong{References} panel. (If it is hidden, click the button
on the left side of the footer area to reveal it.) This panel contains a References tab with links
to all documents referring back to the one you’re currently viewing, if any has been defined. If
you have created root folders and tags for various story elements like characters and plot points,
these will appear as additional tabs in this panel.

\begin{sphinxadmonition}{note}{Note:}
\sphinxAtStartPar
The \sphinxstylestrong{References} panel relies on an up\sphinxhyphen{}to\sphinxhyphen{}date {\hyperref[\detokenize{int_glossary:term-Project-Index}]{\sphinxtermref{\DUrole{xref}{\DUrole{std}{\DUrole{std-term}{index}}}}}} of the project.
The index is maintained automatically. However, if anything is missing, or seems wrong, the
index can always be rebuilt by selecting \sphinxstylestrong{Rebuild Index} from the \sphinxstylestrong{Tools} menu, or by
pressing \sphinxkeyboard{\sphinxupquote{F9}}.
\end{sphinxadmonition}

\sphinxAtStartPar
\DUrole{versionmodified}{\DUrole{added}{Added in version 2.2: }}The reference panel was redesigned and the additional tabs added.


\section{Search \& Replace}
\label{\detokenize{usage_writing:search-replace}}\label{\detokenize{usage_writing:a-ui-edit-search}}
\begin{figure}[htbp]
\centering
\capstart

\noindent\sphinxincludegraphics{{fig_editor_search}.png}
\caption{A screenshot of the Document Editor search box.}\label{\detokenize{usage_writing:id3}}\end{figure}

\sphinxAtStartPar
The document editor has a search and replace tool that can be activated with \sphinxkeyboard{\sphinxupquote{Ctrl+F}} for
search mode or \sphinxkeyboard{\sphinxupquote{Ctrl+H}} for search and replace mode.

\sphinxAtStartPar
Pressing \sphinxkeyboard{\sphinxupquote{Return}} while in the search box will search for the next occurrence of the word, and
\sphinxkeyboard{\sphinxupquote{Shift+Return}} for the previous. Pressing \sphinxkeyboard{\sphinxupquote{Return}} in the replace box, will replace the
highlighted text and move to the next result.

\sphinxAtStartPar
There are a number of settings for the search tool available as toggle switches above the search
box. They allow you to search for, in order: matched case only, whole word results only, search
using regular expressions, loop search when reaching the end of the document, and move to the next
document when reaching the end. There is also a switch that will try to match the case of the word
when the replacement is made. That is, it will try to keep the word upper, lower, or capitalised to
match the word being replaced.

\sphinxAtStartPar
The regular expression search is somewhat dependant on which version of Qt your system has. If you
have Qt 5.13 or higher, there is better support for Unicode symbols in the search.


\begin{sphinxseealso}{See also:}

\sphinxAtStartPar
For more information on the capabilities of the Regular Expression option, see the Qt
documentation for the \sphinxhref{https://doc.qt.io/qt-5/qregularexpression.html}{QRegularExpression}
class.


\end{sphinxseealso}



\section{Auto\sphinxhyphen{}Replace as You Type}
\label{\detokenize{usage_writing:auto-replace-as-you-type}}\label{\detokenize{usage_writing:a-ui-edit-auto}}
\sphinxAtStartPar
A few auto\sphinxhyphen{}replace features are supported by the editor. You can control every aspect of the
auto\sphinxhyphen{}replace feature from \sphinxstylestrong{Preferences}. You can also disable this feature entirely if you wish.

\begin{sphinxadmonition}{tip}{Tip:}
\sphinxAtStartPar
If you don’t like auto\sphinxhyphen{}replacement, all symbols inserted by this feature are also available in
the \sphinxguilabel{Insert} menu, and via {\hyperref[\detokenize{usage_shortcuts:a-kb-ins}]{\sphinxcrossref{\DUrole{std}{\DUrole{std-ref}{Insert Shortcuts}}}}}. You may also be using a \sphinxhref{https://en.wikipedia.org/wiki/Compose\_key}{Compose Key}
setup, which means you may not need the auto\sphinxhyphen{}replace feature at all.
\end{sphinxadmonition}

\sphinxAtStartPar
The editor is able to replace two and three hyphens with short and long dashes, triple points with
ellipsis, and replace straight single and double quotes with user\sphinxhyphen{}defined quote symbols. It will
also try to determine whether to use the opening or closing symbol, although this feature isn’t
always accurate. Especially distinguishing between closing single quote and apostrophe can be
tricky for languages that use the same symbol for these, like English does.

\begin{sphinxadmonition}{tip}{Tip:}
\sphinxAtStartPar
If the auto\sphinxhyphen{}replace feature changes a symbol when you did not want it to change, pressing
\sphinxkeyboard{\sphinxupquote{Ctrl+Z}} once after the auto\sphinxhyphen{}replacement will undo it without undoing the character
you typed before it.
\end{sphinxadmonition}

\sphinxstepscope


\chapter{Formatting Your Text}
\label{\detokenize{usage_format:formatting-your-text}}\label{\detokenize{usage_format:a-fmt}}\label{\detokenize{usage_format::doc}}
\sphinxAtStartPar
The novelWriter text editor is a plain text editor that uses formatting codes for setting meta data
values and allowing for some text formatting. The syntax is based on Markdown, but novelWriter is
\sphinxstylestrong{not} a Markdown editor. It supports basic formatting like emphasis (italic), strong importance
(bold) and strike through text, as well as four levels of headings. For some further complex
formatting needs, a set of shortcodes can be used.

\sphinxAtStartPar
In addition to formatting codes, novelWriter allows for comments, a synopsis tag, and a number of
keyword and value sets used for {\hyperref[\detokenize{int_glossary:term-Tag}]{\sphinxtermref{\DUrole{xref}{\DUrole{std}{\DUrole{std-term}{tags}}}}}} and {\hyperref[\detokenize{int_glossary:term-Reference}]{\sphinxtermref{\DUrole{xref}{\DUrole{std}{\DUrole{std-term}{references}}}}}}. There are also
some codes that apply to whole paragraphs. See {\hyperref[\detokenize{usage_format:a-fmt-text}]{\sphinxcrossref{\DUrole{std}{\DUrole{std-ref}{Text Paragraphs}}}}} for more details.


\section{Syntax Highlighting}
\label{\detokenize{usage_format:syntax-highlighting}}\label{\detokenize{usage_format:a-fmt-hlight}}
\sphinxAtStartPar
The editor has a syntax highlighter feature that is meant to help you know when you’ve used the
formatting tags or other features correctly. It will change the colour and font size of your
headings, change the text colour of emphasised text, and it can also show you where you have
dialogue in your text.

\begin{figure}[htbp]
\centering
\capstart

\noindent\sphinxincludegraphics{{fig_references}.png}
\caption{An example of the colour highlighting of references. “Bob” is not defined, and “@blabla” is not
a valid reference type.}\label{\detokenize{usage_format:id1}}\end{figure}

\sphinxAtStartPar
When you use the keywords to set tags and references, these also change colour. Correct keywords
have a distinct colour, and the references themselves will get a colour if they are valid. Invalid
references will get a squiggly error line underneath. The same applies to duplicate tags.

\sphinxAtStartPar
There are a number of syntax highlighter colour themes available, both for light and dark GUIs. You
can select them from \sphinxstylestrong{Preferences}.


\section{Headings}
\label{\detokenize{usage_format:headings}}\label{\detokenize{usage_format:a-fmt-head}}
\begin{figure}[htbp]
\centering
\capstart

\noindent\sphinxincludegraphics{{fig_header_levels}.png}
\caption{An illustration of how heading levels correspond to the novel structure.}\label{\detokenize{usage_format:id2}}\end{figure}

\sphinxAtStartPar
Four levels of headings are allowed. For {\hyperref[\detokenize{int_glossary:term-Project-Notes}]{\sphinxtermref{\DUrole{xref}{\DUrole{std}{\DUrole{std-term}{project notes}}}}}}, they are free to be used as you see
fit. That is, novelWriter doesn’t assign the different headings any particular meaning. However,
for {\hyperref[\detokenize{int_glossary:term-Novel-Documents}]{\sphinxtermref{\DUrole{xref}{\DUrole{std}{\DUrole{std-term}{novel documents}}}}}} they indicate the structural level of the novel and must be used
correctly to produce the intended result. See {\hyperref[\detokenize{project_structure:a-struct-heads}]{\sphinxcrossref{\DUrole{std}{\DUrole{std-ref}{Importance of Headings}}}}} for more details.
\begin{description}
\sphinxlineitem{\sphinxcode{\sphinxupquote{\# Title Text}}}
\sphinxAtStartPar
Heading level one. For novel documents, the level indicates the start of a new partition.
Partitions are for when you want to split your story into “Part 1”, “Part 2”, etc. You can also
choose to use them for splitting the text up into acts, and then hide these headings in your
manuscript.

\sphinxlineitem{\sphinxcode{\sphinxupquote{\#\# Title Text}}}
\sphinxAtStartPar
Heading level two. For novel documents, the level indicates the start of a new chapter. Chapter
numbers can be inserted automatically when building the manuscript.

\sphinxlineitem{\sphinxcode{\sphinxupquote{\#\#\# Title Text}}}
\sphinxAtStartPar
Heading level three. For novel documents, the level indicates the start of a new scene. Scene
numbers or scene separators can be inserted automatically when building the manuscript, so you
can use the title field as a working title for your scenes if you wish, but you must provide a
minimal title.

\sphinxlineitem{\sphinxcode{\sphinxupquote{\#\#\#\# Title Text}}}
\sphinxAtStartPar
Heading level four. For novel documents, the level indicates the start of a new section. Section
titles can be replaced by separators or ignored completely when building the manuscript.

\end{description}

\sphinxAtStartPar
For headings level one through three, adding a \sphinxcode{\sphinxupquote{!}} modifies the meaning of the heading:
\begin{description}
\sphinxlineitem{\sphinxcode{\sphinxupquote{\#! Title Text}}}
\sphinxAtStartPar
This tells the build tool that the level one heading is intended to be used for the novel or
notes folder’s main title, like for instance on the front page. When building the manuscript,
this will use a different styling and will exclude the title from, for instance, a Table of
Contents in Libre Office.

\sphinxlineitem{\sphinxcode{\sphinxupquote{\#\#! Title Text}}}
\sphinxAtStartPar
This tells the build tool to not assign a chapter number to this chapter title if automatic
chapter numbers are being used. Such titles are useful for a prologue for instance. See
{\hyperref[\detokenize{project_structure:a-struct-heads-unnum}]{\sphinxcrossref{\DUrole{std}{\DUrole{std-ref}{Unnumbered Chapter Headings}}}}} for more details.

\sphinxlineitem{\sphinxcode{\sphinxupquote{\#\#\#! Title Text}}}
\sphinxAtStartPar
This is an alternative scene heading that can be formatted differently in the \sphinxstylestrong{Manuscript
Build} tool. It is intended for separating “soft” and “hard” scene breaks. Aside from this, it
behaves identically to a regular scene heading. See {\hyperref[\detokenize{project_structure:a-struct-heads-scenes}]{\sphinxcrossref{\DUrole{std}{\DUrole{std-ref}{Hard and Soft Scene Breaks}}}}} for more
details.

\end{description}

\begin{sphinxadmonition}{note}{Note:}
\sphinxAtStartPar
The space after the \sphinxcode{\sphinxupquote{\#}} or \sphinxcode{\sphinxupquote{!}} character is mandatory. The syntax highlighter will change
colour and font size when the heading is correctly formatted.
\end{sphinxadmonition}


\section{Text Paragraphs}
\label{\detokenize{usage_format:text-paragraphs}}\label{\detokenize{usage_format:a-fmt-text}}
\sphinxAtStartPar
A text paragraph is indicated by a blank line. That is, you need two line breaks to separate two
fragments of text into two paragraphs. Single line breaks are treated as line breaks within a
paragraph.

\sphinxAtStartPar
In addition, the editor supports a few additional types of white spaces:
\begin{itemize}
\item {} 
\sphinxAtStartPar
A non\sphinxhyphen{}breaking space can be inserted with \sphinxkeyboard{\sphinxupquote{Ctrl+K}}, \sphinxkeyboard{\sphinxupquote{Space}}.

\item {} 
\sphinxAtStartPar
Thin spaces are also supported, and can be inserted with \sphinxkeyboard{\sphinxupquote{Ctrl+K}}, \sphinxkeyboard{\sphinxupquote{Shift+Space}}.

\item {} 
\sphinxAtStartPar
Non\sphinxhyphen{}breaking thin space can be inserted with \sphinxkeyboard{\sphinxupquote{Ctrl+K}}, \sphinxkeyboard{\sphinxupquote{Ctrl+Space}}.

\end{itemize}

\sphinxAtStartPar
These are all insert features, and the \sphinxstylestrong{Insert} menu has more. The keyboard shortcuts for them
are also listed in {\hyperref[\detokenize{usage_shortcuts:a-kb-ins}]{\sphinxcrossref{\DUrole{std}{\DUrole{std-ref}{Insert Shortcuts}}}}}.

\sphinxAtStartPar
Non\sphinxhyphen{}breaking spaces are highlighted by the syntax highlighter with an alternate coloured
background, depending on the selected theme.

\begin{sphinxadmonition}{tip}{Tip:}
\sphinxAtStartPar
Non\sphinxhyphen{}breaking spaces are for instance the correct type of space to separate a number from its
unit. Generally, non\sphinxhyphen{}breaking spaces are used to prevent line wrapping algorithms from adding
line breaks where they shouldn’t.
\end{sphinxadmonition}


\section{Text Emphasis with Markdown}
\label{\detokenize{usage_format:text-emphasis-with-markdown}}\label{\detokenize{usage_format:a-fmt-emph}}
\sphinxAtStartPar
A minimal set of Markdown text emphasis styles are supported for text paragraphs.
\begin{description}
\sphinxlineitem{\sphinxcode{\sphinxupquote{\_text\_}}}
\sphinxAtStartPar
The text is rendered as emphasised text (italicised).

\sphinxlineitem{\sphinxcode{\sphinxupquote{**text**}}}
\sphinxAtStartPar
The text is rendered as strongly emphasised text (bold).

\sphinxlineitem{\sphinxcode{\sphinxupquote{\textasciitilde{}\textasciitilde{}text\textasciitilde{}\textasciitilde{}}}}
\sphinxAtStartPar
Strike through text.

\end{description}

\sphinxAtStartPar
In Markdown guides it is often recommended to differentiate between strong emphasis and emphasis
by using \sphinxcode{\sphinxupquote{**}} for strong and \sphinxcode{\sphinxupquote{\_}} for emphasis, although Markdown generally also supports \sphinxcode{\sphinxupquote{\_\_}}
for strong and \sphinxcode{\sphinxupquote{*}} for emphasis. However, since the differentiation makes the highlighting and
conversion significantly simpler and faster, in novelWriter this is a rule, not just a
recommendation.

\sphinxAtStartPar
In addition, the following rules apply:
\begin{enumerate}
\sphinxsetlistlabels{\arabic}{enumi}{enumii}{}{.}%
\item {} 
\sphinxAtStartPar
The emphasis and strike through formatting tags do not allow spaces between the words and the
tag itself. That is, \sphinxcode{\sphinxupquote{**text**}} is valid, \sphinxcode{\sphinxupquote{**text **}} is not.

\item {} 
\sphinxAtStartPar
More generally, the delimiters must be on the outer edge of words. That is, \sphinxcode{\sphinxupquote{some **text in
bold** here}} is valid, \sphinxcode{\sphinxupquote{some** text in bold** here}} is not.

\item {} 
\sphinxAtStartPar
If using both \sphinxcode{\sphinxupquote{**}} and \sphinxcode{\sphinxupquote{\_}} to wrap the same text, the underscore must be the \sphinxstylestrong{inner}
wrapper. This is due to the underscore also being a valid word character, so if they are on the
outside, they violate rule 2.

\item {} 
\sphinxAtStartPar
Text emphasis does not span past line breaks. If you need to add emphasis to multiple lines or
paragraphs, you must apply it to each of them in turn.

\item {} 
\sphinxAtStartPar
Text emphasis can only be used in comments and paragraphs. Headings and meta data tags don’t
allow for formatting, and any formatting markup will be rendered as\sphinxhyphen{}is.

\end{enumerate}

\begin{sphinxadmonition}{tip}{Tip:}
\sphinxAtStartPar
novelWriter supports standard escape syntax for the emphasis markup characters in case the
editor misunderstands your intended usage of them. That is, \sphinxcode{\sphinxupquote{\textbackslash{}*}}, \sphinxcode{\sphinxupquote{\textbackslash{}\_}} and \sphinxcode{\sphinxupquote{\textbackslash{}\textasciitilde{}}} will
generate a plain \sphinxcode{\sphinxupquote{*}}, \sphinxcode{\sphinxupquote{\_}} and \sphinxcode{\sphinxupquote{\textasciitilde{}}}, respectively, without interpreting them as part of the
markup.
\end{sphinxadmonition}


\section{Formatting with Shortcodes}
\label{\detokenize{usage_format:formatting-with-shortcodes}}\label{\detokenize{usage_format:a-fmt-shortcodes}}
\sphinxAtStartPar
For additional formatting options, you can use shortcodes. Shortcodes is a form of in\sphinxhyphen{}line codes
that can be used to change the format of the text that follows and opening code, and last until
that formatting region is ended with a closing code.

\sphinxAtStartPar
These shortcodes are intended for special formatting cases, or more complex cases that cannot be
solved with simple Markdown\sphinxhyphen{}like formatting codes. Available shortcodes are listed below.


\begin{savenotes}\sphinxattablestart
\sphinxthistablewithglobalstyle
\centering
\sphinxcapstartof{table}
\sphinxthecaptionisattop
\sphinxcaption{Shortcodes Formats}\label{\detokenize{usage_format:id3}}
\sphinxaftertopcaption
\begin{tabular}[t]{\X{40}{100}\X{60}{100}}
\sphinxtoprule
\sphinxstyletheadfamily 
\sphinxAtStartPar
Syntax
&\sphinxstyletheadfamily 
\sphinxAtStartPar
Description
\\
\sphinxmidrule
\sphinxtableatstartofbodyhook
\sphinxAtStartPar
\sphinxcode{\sphinxupquote{{[}b{]}text{[}/b{]}}}
&
\sphinxAtStartPar
Text is rendered as bold text.
\\
\sphinxhline
\sphinxAtStartPar
\sphinxcode{\sphinxupquote{{[}i{]}text{[}/i{]}}}
&
\sphinxAtStartPar
Text is rendered as italicised text.
\\
\sphinxhline
\sphinxAtStartPar
\sphinxcode{\sphinxupquote{{[}s{]}text{[}/s{]}}}
&
\sphinxAtStartPar
Text is rendered as strike through text.
\\
\sphinxhline
\sphinxAtStartPar
\sphinxcode{\sphinxupquote{{[}u{]}text{[}/u{]}}}
&
\sphinxAtStartPar
Text is rendered as underlined text.
\\
\sphinxhline
\sphinxAtStartPar
\sphinxcode{\sphinxupquote{{[}m{]}text{[}/m{]}}}
&
\sphinxAtStartPar
Text is rendered as highlighted text.
\\
\sphinxhline
\sphinxAtStartPar
\sphinxcode{\sphinxupquote{{[}sup{]}text{[}/sup{]}}}
&
\sphinxAtStartPar
Text is rendered as superscript text.
\\
\sphinxhline
\sphinxAtStartPar
\sphinxcode{\sphinxupquote{{[}sub{]}text{[}/sub{]}}}
&
\sphinxAtStartPar
Text is rendered as subscript text.
\\
\sphinxbottomrule
\end{tabular}
\sphinxtableafterendhook\par
\sphinxattableend\end{savenotes}

\sphinxAtStartPar
Unlike Markdown style codes, these can be used anywhere within a paragraph. Even in the middle of a
word if you need to. You can also freely combine them to form more complex formatting.

\sphinxAtStartPar
The shortcodes are available from the \sphinxstylestrong{Format} menu and in the editor toolbar, which can be
activated by clicking the left\sphinxhyphen{}most icon button in the editor header.

\begin{sphinxadmonition}{note}{Note:}
\sphinxAtStartPar
Shortcodes are not processed until you generate a preview or generate a manuscript document. So
there is no highlighting of the text between the formatting markers. There is also no check that
your markers make sense. You must ensure that you have both the opening and closing formatting
markers where you want them.
\end{sphinxadmonition}

\sphinxAtStartPar
\DUrole{versionmodified}{\DUrole{added}{Added in version 2.2.}}


\section{Comments and Synopsis}
\label{\detokenize{usage_format:comments-and-synopsis}}\label{\detokenize{usage_format:a-fmt-comm}}
\sphinxAtStartPar
In addition to the above formatting features, novelWriter also allows for comments in documents.
The text of a comment is always ignored by the word counter. The text can also be filtered out
when building the manuscript or viewing the document.

\sphinxAtStartPar
The first word of a comment, followed by a colon, can be one of a small set of modifiers that
indicates the comment is intended for a specific purpose. For instance, if the comment starts with
\sphinxcode{\sphinxupquote{Synopsis:}}, the comment is treated in a special manner and will show up in the
{\hyperref[\detokenize{usage_project:a-ui-outline}]{\sphinxcrossref{\DUrole{std}{\DUrole{std-ref}{The Novel Outline View}}}}} in a dedicated column. The word \sphinxcode{\sphinxupquote{synopsis}} is not case sensitive. If it is
correctly formatted, the syntax highlighter will indicate this by altering the colour of the word.

\sphinxAtStartPar
The different styles of comments are as follows:
\begin{description}
\sphinxlineitem{\sphinxcode{\sphinxupquote{\% Your comment text ...}}}
\sphinxAtStartPar
This is a comment. The text is not rendered by default (this can be overridden), seen in the
document viewer, or counted towards word counts. It is intended for you to make notes in your
text for your own sake, whatever that may be, that isn’t part of the story text. This is the
general format of a comment.

\sphinxlineitem{\sphinxcode{\sphinxupquote{\%Synopsis: Your synopsis text ...}}}
\sphinxAtStartPar
This is a synopsis comment. It is generally treated in the same way as a regular comment, except
that it is also captured by the indexing algorithm and displayed in the {\hyperref[\detokenize{usage_project:a-ui-outline}]{\sphinxcrossref{\DUrole{std}{\DUrole{std-ref}{The Novel Outline View}}}}}. It
can also be filtered separately when building the project to for instance generate an outline
document of the whole project.

\sphinxlineitem{\sphinxcode{\sphinxupquote{\%Short: Your short description ...}}}
\sphinxAtStartPar
This is a short description comment. It is identical to the synopsis comment (they are
interchangeable), but is intended to be used for project notes. The text shows up in the
Reference panel below the document viewer in the last column labelled \sphinxstylestrong{Short Description}.

\sphinxlineitem{\sphinxcode{\sphinxupquote{\%Footnote.\textless{}key\textgreater{}: Your footnote text ...}}}
\sphinxAtStartPar
This is a special comment assigned to a footnote marker. See {\hyperref[\detokenize{usage_format:a-fmt-footnote}]{\sphinxcrossref{\DUrole{std}{\DUrole{std-ref}{Footnotes}}}}} for how to
use them in your text.

\end{description}

\begin{sphinxadmonition}{note}{Note:}
\sphinxAtStartPar
Only one comment can be flagged as a synopsis or short comment for each heading. If multiple
comments are flagged as synopsis or short comments, the last one will be used and the rest
ignored.
\end{sphinxadmonition}


\section{Footnotes}
\label{\detokenize{usage_format:footnotes}}\label{\detokenize{usage_format:a-fmt-footnote}}
\sphinxAtStartPar
Footnotes are added with a shortcode, paired with a matching comment for the actual footnote text.
The matching is done with a key that links the two. If you insert a footnote from the \sphinxstylestrong{Insert}
menu, a unique key is generated for you.

\sphinxAtStartPar
The insert feature will add the footnote shortcode marker at the position of your cursor in the
text, and create the associated footnote comment right after the paragraph, and move the cursor
there so you can immediately start typing the footnote text.

\sphinxAtStartPar
The footnote comment can be anywhere in the document, so if you wish to move them to, say, the
bottom of the text, you are free to do so.

\sphinxAtStartPar
Footnote keys are only required to be unique within a document, so if you copy, move or merge text,
you must make sure the keys are not duplicated. If you use the automatically generated keys from
the \sphinxstylestrong{Insert} menu, they are unique among all indexed documents. They are not guaranteed to be
unique against footnotes in the Archive or Trash folder though, but the chance of accidentally
generating the same key twice in a project is relatively small in the first place (1 in 810 000).

\sphinxAtStartPar
This is what a footnote inserted into a paragraph may look like when completed:

\begin{sphinxVerbatim}[commandchars=\\\{\}]
This is a text paragraph with a footnote[footnote:fn1] in the middle.

\PYGZpc{}Footnote.fn1: This is the text of the footnote.
\end{sphinxVerbatim}

\sphinxAtStartPar
\DUrole{versionmodified}{\DUrole{added}{Added in version 2.5.}}


\section{Ignored Text}
\label{\detokenize{usage_format:ignored-text}}\label{\detokenize{usage_format:a-fmt-ignore}}
\sphinxAtStartPar
If you want to completely ignore some of the text in your documents, but are not ready to delete
it, you can add \sphinxcode{\sphinxupquote{\%\textasciitilde{}}} before the text paragraph or line. This will cause novelWriter to skip the
text entirely when generating previews or building manuscripts.

\sphinxAtStartPar
This is a better way of removing text than converting them to regular comments, as you may want to
include regular comments in your previews or draft manuscript.

\sphinxAtStartPar
You can toggle the ignored text feature on and off for a paragraph by pressing \sphinxkeyboard{\sphinxupquote{Ctrl+Shift+D}}
on your keyboard with your cursor somewhere in the paragraph.

\sphinxAtStartPar
Example:

\begin{sphinxVerbatim}[commandchars=\\\{\}]
\PYGZpc{}\PYGZti{} This text is ignored.

This text is a regular paragraph.
\end{sphinxVerbatim}


\section{Tags and References}
\label{\detokenize{usage_format:tags-and-references}}\label{\detokenize{usage_format:a-fmt-tags}}
\sphinxAtStartPar
The document editor supports a set of keywords used for setting tags, and making references between
documents based on those tags.

\sphinxAtStartPar
You must use the keyword \sphinxcode{\sphinxupquote{@tag:}} to define a tag. The tag can be set once per section defined by
a heading. Setting it multiple times under the same heading will just override the previous
setting.
\begin{description}
\sphinxlineitem{\sphinxcode{\sphinxupquote{@tag: value}}}
\sphinxAtStartPar
A tag keyword followed by the tag value, like for instance the name of a character.

\end{description}

\sphinxAtStartPar
References can be set anywhere within a section, and are collected according to their category.
References are on the form:
\begin{description}
\sphinxlineitem{\sphinxcode{\sphinxupquote{@keyword: value1, value2, ..., valueN}}}
\sphinxAtStartPar
A reference keyword followed by a value, or a comma separated list of values.

\end{description}

\sphinxAtStartPar
Tags and references are covered in detail in the {\hyperref[\detokenize{project_references:a-references}]{\sphinxcrossref{\DUrole{std}{\DUrole{std-ref}{Tags and References}}}}} chapter. The keywords can be
inserted at the cursor position in the editor via the \sphinxstylestrong{Insert} menu. If you start typing an \sphinxcode{\sphinxupquote{@}}
on a new line, and auto\sphinxhyphen{}complete menu will also pop up suggesting keywords.


\section{Paragraph Alignment and Indentation}
\label{\detokenize{usage_format:paragraph-alignment-and-indentation}}\label{\detokenize{usage_format:a-fmt-align}}
\sphinxAtStartPar
All documents have the text by default aligned to the left or justified, depending on your setting
in \sphinxstylestrong{Preferences}.

\sphinxAtStartPar
You can override the default text alignment on individual paragraphs by specifying alignment tags.
These tags are double angle brackets. Either \sphinxcode{\sphinxupquote{\textgreater{}\textgreater{}}} or \sphinxcode{\sphinxupquote{\textless{}\textless{}}}. You put them either before or after
the paragraph, and they will “push” the text towards the edge the brackets point towards. This
should be fairly intuitive.

\sphinxAtStartPar
Indentation uses a similar syntax. But here you use a single \sphinxcode{\sphinxupquote{\textgreater{}}} or \sphinxcode{\sphinxupquote{\textless{}}} to push the text away
from the edge.

\sphinxAtStartPar
Examples:


\begin{savenotes}\sphinxattablestart
\sphinxthistablewithglobalstyle
\centering
\sphinxcapstartof{table}
\sphinxthecaptionisattop
\sphinxcaption{Text Alignment and Indentation}\label{\detokenize{usage_format:id4}}
\sphinxaftertopcaption
\begin{tabular}[t]{\X{40}{100}\X{60}{100}}
\sphinxtoprule
\sphinxstyletheadfamily 
\sphinxAtStartPar
Syntax
&\sphinxstyletheadfamily 
\sphinxAtStartPar
Description
\\
\sphinxmidrule
\sphinxtableatstartofbodyhook
\sphinxAtStartPar
\sphinxcode{\sphinxupquote{\textgreater{}\textgreater{} Right aligned text}}
&
\sphinxAtStartPar
The text paragraph is right\sphinxhyphen{}aligned.
\\
\sphinxhline
\sphinxAtStartPar
\sphinxcode{\sphinxupquote{Left aligned text \textless{}\textless{}}}
&
\sphinxAtStartPar
The text paragraph is left\sphinxhyphen{}aligned.
\\
\sphinxhline
\sphinxAtStartPar
\sphinxcode{\sphinxupquote{\textgreater{}\textgreater{} Centred text \textless{}\textless{}}}
&
\sphinxAtStartPar
The text paragraph is centred.
\\
\sphinxhline
\sphinxAtStartPar
\sphinxcode{\sphinxupquote{\textgreater{} Left indented text}}
&
\sphinxAtStartPar
The text has an increased left margin.
\\
\sphinxhline
\sphinxAtStartPar
\sphinxcode{\sphinxupquote{Right indented text \textless{}}}
&
\sphinxAtStartPar
The text has an increased right margin.
\\
\sphinxhline
\sphinxAtStartPar
\sphinxcode{\sphinxupquote{\textgreater{} Left/right indented text \textless{}}}
&
\sphinxAtStartPar
The text has both margins increased.
\\
\sphinxbottomrule
\end{tabular}
\sphinxtableafterendhook\par
\sphinxattableend\end{savenotes}

\begin{sphinxadmonition}{note}{Note:}
\sphinxAtStartPar
The text editor will not show the alignment and indentation live. But the viewer will show them
when you open the document there. It will of course also be reflected in the document generated
from the manuscript build tool as long as the format supports paragraph alignment.
\end{sphinxadmonition}


\subsection{Alignment with Line Breaks}
\label{\detokenize{usage_format:alignment-with-line-breaks}}
\sphinxAtStartPar
If you have line breaks in the paragraph, like for instance when you are writing verses, the
alignment markers must be applied to the first line. Markers on the other lines are ignored. The
markers for the first line are used for all the other lines.

\sphinxAtStartPar
For the following text, all lines will be centred, not just the first:

\begin{sphinxVerbatim}[commandchars=\\\{\}]
\PYGZgt{}\PYGZgt{} I am the very model of a modern Major\PYGZhy{}General \PYGZlt{}\PYGZlt{}
I\PYGZsq{}ve information vegetable, animal, and mineral
I know the kings of England, and I quote the fights historical
From Marathon to Waterloo, in order categorical
\end{sphinxVerbatim}


\subsection{Alignment with First Line Indent}
\label{\detokenize{usage_format:alignment-with-first-line-indent}}
\sphinxAtStartPar
If you have first line indent enabled in your Manuscript build settings, you probably want to
disable it for text in verses. Adding any alignment tags will cause the first line indent to be
switched off for that paragraph.

\sphinxAtStartPar
The following text will always be aligned against the left margin:

\begin{sphinxVerbatim}[commandchars=\\\{\}]
I am the very model of a modern Major\PYGZhy{}General \PYGZlt{}\PYGZlt{}
I\PYGZsq{}ve information vegetable, animal, and mineral
I know the kings of England, and I quote the fights historical
From Marathon to Waterloo, in order categorical
\end{sphinxVerbatim}


\section{Vertical Space and Page Breaks}
\label{\detokenize{usage_format:vertical-space-and-page-breaks}}\label{\detokenize{usage_format:a-fmt-break}}
\sphinxAtStartPar
You can apply page breaks to partition, chapter and scene headings for novel documents from the
\sphinxstylestrong{Manuscript Build} tool. If you need to add a page break or additional vertical spacing in other
places, there are special codes available for this purpose.

\sphinxAtStartPar
Adding more than one line break between paragraphs will \sphinxstylestrong{not} increase the space between those
paragraphs when building the project. To add additional space between paragraphs, add the text
\sphinxcode{\sphinxupquote{{[}vspace{]}}} on a line of its own, and the build tool will insert a blank paragraph in its place.

\sphinxAtStartPar
If you need multiple blank paragraphs just add a colon and a number to the above code. For
instance, writing \sphinxcode{\sphinxupquote{{[}vspace:3{]}}} will insert three blank paragraphs.

\sphinxAtStartPar
If you need to add a page break somewhere, put the text \sphinxcode{\sphinxupquote{{[}new page{]}}} on a line by itself before
the text you wish to start on a new page.

\begin{sphinxadmonition}{note}{Note:}
\sphinxAtStartPar
The page break code is applied to the text that follows it. It adds a “page break before” mark
to the text when exporting to HTML or Open Document. This means that a \sphinxcode{\sphinxupquote{{[}new page{]}}} which has
no text following it, it will not result in a page break.
\end{sphinxadmonition}

\sphinxAtStartPar
\sphinxstylestrong{Example:}

\begin{sphinxVerbatim}[commandchars=\\\{\}]
This is a text paragraph.

[vspace:2]

This is another text paragraph, but there will be two empty paragraphs
between them.

[new page]

This text will start on a new page if the build format has pages.
\end{sphinxVerbatim}

\sphinxstepscope


\chapter{Keyboard Shortcuts}
\label{\detokenize{usage_shortcuts:keyboard-shortcuts}}\label{\detokenize{usage_shortcuts:a-kb}}\label{\detokenize{usage_shortcuts::doc}}
\sphinxAtStartPar
Most features in novelWriter are available as keyboard shortcuts. This is a reference list of those
shortcuts. Most of them are also listed in the application’s user interface.

\begin{sphinxadmonition}{note}{Note:}
\sphinxAtStartPar
On MacOS, replace \sphinxkeyboard{\sphinxupquote{Ctrl}} with \sphinxkeyboard{\sphinxupquote{Cmd}}.
\end{sphinxadmonition}


\section{Main Window Shortcuts}
\label{\detokenize{usage_shortcuts:main-window-shortcuts}}\label{\detokenize{usage_shortcuts:a-kb-main}}

\begin{savenotes}\sphinxattablestart
\sphinxthistablewithglobalstyle
\centering
\begin{tabulary}{\linewidth}[t]{TT}
\sphinxtoprule
\sphinxstyletheadfamily 
\sphinxAtStartPar
Shortcut
&\sphinxstyletheadfamily 
\sphinxAtStartPar
Description
\\
\sphinxmidrule
\sphinxtableatstartofbodyhook
\sphinxAtStartPar
\sphinxkeyboard{\sphinxupquote{F1}}
&
\sphinxAtStartPar
Open the online user manual
\\
\sphinxhline
\sphinxAtStartPar
\sphinxkeyboard{\sphinxupquote{F5}}
&
\sphinxAtStartPar
Open the \sphinxstylestrong{Build Manuscript} tool
\\
\sphinxhline
\sphinxAtStartPar
\sphinxkeyboard{\sphinxupquote{F6}}
&
\sphinxAtStartPar
Open the \sphinxstylestrong{Writing Statistics} tool
\\
\sphinxhline
\sphinxAtStartPar
\sphinxkeyboard{\sphinxupquote{F8}}
&
\sphinxAtStartPar
Toggle \sphinxstylestrong{Focus Mode}
\\
\sphinxhline
\sphinxAtStartPar
\sphinxkeyboard{\sphinxupquote{F9}}
&
\sphinxAtStartPar
Re\sphinxhyphen{}build the project’s index
\\
\sphinxhline
\sphinxAtStartPar
\sphinxkeyboard{\sphinxupquote{F11}}
&
\sphinxAtStartPar
Toggle full screen mode
\\
\sphinxhline
\sphinxAtStartPar
\sphinxkeyboard{\sphinxupquote{Ctrl+,}}
&
\sphinxAtStartPar
Open the \sphinxstylestrong{Preferences} dialog
\\
\sphinxhline
\sphinxAtStartPar
\sphinxkeyboard{\sphinxupquote{Ctrl+E}}
&
\sphinxAtStartPar
Switch or toggle focus for the editor or viewer
\\
\sphinxhline
\sphinxAtStartPar
\sphinxkeyboard{\sphinxupquote{Ctrl+T}}
&
\sphinxAtStartPar
Switch or toggle focus for the project tree or novel view
\\
\sphinxhline
\sphinxAtStartPar
\sphinxkeyboard{\sphinxupquote{Ctrl+Q}}
&
\sphinxAtStartPar
Exit novelWriter
\\
\sphinxhline
\sphinxAtStartPar
\sphinxkeyboard{\sphinxupquote{Ctrl+Shift+,}}
&
\sphinxAtStartPar
Open the \sphinxstylestrong{Project Settings} dialog
\\
\sphinxhline
\sphinxAtStartPar
\sphinxkeyboard{\sphinxupquote{Ctrl+Shift+O}}
&
\sphinxAtStartPar
Open the Welcome dialog to open or create a project
\\
\sphinxhline
\sphinxAtStartPar
\sphinxkeyboard{\sphinxupquote{Ctrl+Shift+S}}
&
\sphinxAtStartPar
Save the current project
\\
\sphinxhline
\sphinxAtStartPar
\sphinxkeyboard{\sphinxupquote{Ctrl+Shift+T}}
&
\sphinxAtStartPar
Switch focus to the outline view
\\
\sphinxhline
\sphinxAtStartPar
\sphinxkeyboard{\sphinxupquote{Ctrl+Shift+W}}
&
\sphinxAtStartPar
Close the current project
\\
\sphinxhline
\sphinxAtStartPar
\sphinxkeyboard{\sphinxupquote{Shift+F1}}
&
\sphinxAtStartPar
Open the local user manual (PDF) if it is available
\\
\sphinxhline
\sphinxAtStartPar
\sphinxkeyboard{\sphinxupquote{Shift+F6}}
&
\sphinxAtStartPar
Open the \sphinxstylestrong{Project Details} dialog
\\
\sphinxbottomrule
\end{tabulary}
\sphinxtableafterendhook\par
\sphinxattableend\end{savenotes}


\section{Project Tree Shortcuts}
\label{\detokenize{usage_shortcuts:project-tree-shortcuts}}\label{\detokenize{usage_shortcuts:a-kb-tree}}

\begin{savenotes}\sphinxattablestart
\sphinxthistablewithglobalstyle
\centering
\begin{tabulary}{\linewidth}[t]{TT}
\sphinxtoprule
\sphinxstyletheadfamily 
\sphinxAtStartPar
Shortcut
&\sphinxstyletheadfamily 
\sphinxAtStartPar
Description
\\
\sphinxmidrule
\sphinxtableatstartofbodyhook
\sphinxAtStartPar
\sphinxkeyboard{\sphinxupquote{F2}}
&
\sphinxAtStartPar
Edit the label of the selected item
\\
\sphinxhline
\sphinxAtStartPar
\sphinxkeyboard{\sphinxupquote{Return}}
&
\sphinxAtStartPar
Open the selected document in the editor
\\
\sphinxhline
\sphinxAtStartPar
\sphinxkeyboard{\sphinxupquote{Alt+Up}}
&
\sphinxAtStartPar
Jump or go to the previous item at same level in the tree
\\
\sphinxhline
\sphinxAtStartPar
\sphinxkeyboard{\sphinxupquote{Alt+Down}}
&
\sphinxAtStartPar
Jump or go to the next item at same level in the tree
\\
\sphinxhline
\sphinxAtStartPar
\sphinxkeyboard{\sphinxupquote{Alt+Left}}
&
\sphinxAtStartPar
Jump to the parent item in the tree
\\
\sphinxhline
\sphinxAtStartPar
\sphinxkeyboard{\sphinxupquote{Alt+Right}}
&
\sphinxAtStartPar
Jump to the first child item in the project tree
\\
\sphinxhline
\sphinxAtStartPar
\sphinxkeyboard{\sphinxupquote{Ctrl+.}}
&
\sphinxAtStartPar
Open the context menu on the selected item
\\
\sphinxhline
\sphinxAtStartPar
\sphinxkeyboard{\sphinxupquote{Ctrl+L}}
&
\sphinxAtStartPar
Open the \sphinxstylestrong{Quick Links} menu
\\
\sphinxhline
\sphinxAtStartPar
\sphinxkeyboard{\sphinxupquote{Ctrl+N}}
&
\sphinxAtStartPar
Open the \sphinxstylestrong{Create New Item} menu
\\
\sphinxhline
\sphinxAtStartPar
\sphinxkeyboard{\sphinxupquote{Ctrl+O}}
&
\sphinxAtStartPar
Open the selected document in the editor
\\
\sphinxhline
\sphinxAtStartPar
\sphinxkeyboard{\sphinxupquote{Ctrl+R}}
&
\sphinxAtStartPar
Open the selected document in the viewer
\\
\sphinxhline
\sphinxAtStartPar
\sphinxkeyboard{\sphinxupquote{Ctrl+Up}}
&
\sphinxAtStartPar
Move selected item one step up in the tree
\\
\sphinxhline
\sphinxAtStartPar
\sphinxkeyboard{\sphinxupquote{Ctrl+Down}}
&
\sphinxAtStartPar
Move selected item one step down in the tree
\\
\sphinxhline
\sphinxAtStartPar
\sphinxkeyboard{\sphinxupquote{Ctrl+Shift+Del}}
&
\sphinxAtStartPar
Move the selected item to Trash
\\
\sphinxbottomrule
\end{tabulary}
\sphinxtableafterendhook\par
\sphinxattableend\end{savenotes}


\section{Document Editor Shortcuts}
\label{\detokenize{usage_shortcuts:document-editor-shortcuts}}\label{\detokenize{usage_shortcuts:a-kb-editor}}

\subsection{Text Search Shortcuts}
\label{\detokenize{usage_shortcuts:text-search-shortcuts}}

\begin{savenotes}\sphinxattablestart
\sphinxthistablewithglobalstyle
\centering
\begin{tabulary}{\linewidth}[t]{TT}
\sphinxtoprule
\sphinxstyletheadfamily 
\sphinxAtStartPar
Shortcut
&\sphinxstyletheadfamily 
\sphinxAtStartPar
Description
\\
\sphinxmidrule
\sphinxtableatstartofbodyhook
\sphinxAtStartPar
\sphinxkeyboard{\sphinxupquote{F3}}
&
\sphinxAtStartPar
Find the next occurrence of the search word
\\
\sphinxhline
\sphinxAtStartPar
\sphinxkeyboard{\sphinxupquote{Ctrl+F}}
&
\sphinxAtStartPar
Open search and look for the selected word
\\
\sphinxhline
\sphinxAtStartPar
\sphinxkeyboard{\sphinxupquote{Ctrl+G}}
&
\sphinxAtStartPar
Find the next occurrence of the search word
\\
\sphinxhline
\sphinxAtStartPar
\sphinxkeyboard{\sphinxupquote{Ctrl+H}}
&
\sphinxAtStartPar
Open replace and look for the selected word (Mac \sphinxkeyboard{\sphinxupquote{Cmd+=}})
\\
\sphinxhline
\sphinxAtStartPar
\sphinxkeyboard{\sphinxupquote{Ctrl+Shift+1}}
&
\sphinxAtStartPar
Replace selected occurrence, and move to the next
\\
\sphinxhline
\sphinxAtStartPar
\sphinxkeyboard{\sphinxupquote{Ctrl+Shift+G}}
&
\sphinxAtStartPar
Find the previous occurrence of the search word
\\
\sphinxhline
\sphinxAtStartPar
\sphinxkeyboard{\sphinxupquote{Ctrl+Shift+F}}
&
\sphinxAtStartPar
Open project search and look for the selected word
\\
\sphinxhline
\sphinxAtStartPar
\sphinxkeyboard{\sphinxupquote{Shift+F3}}
&
\sphinxAtStartPar
Find the previous occurrence of the search word
\\
\sphinxbottomrule
\end{tabulary}
\sphinxtableafterendhook\par
\sphinxattableend\end{savenotes}


\subsection{Text Formatting Shortcuts}
\label{\detokenize{usage_shortcuts:text-formatting-shortcuts}}

\begin{savenotes}\sphinxattablestart
\sphinxthistablewithglobalstyle
\centering
\begin{tabulary}{\linewidth}[t]{TT}
\sphinxtoprule
\sphinxstyletheadfamily 
\sphinxAtStartPar
Shortcut
&\sphinxstyletheadfamily 
\sphinxAtStartPar
Description
\\
\sphinxmidrule
\sphinxtableatstartofbodyhook
\sphinxAtStartPar
\sphinxkeyboard{\sphinxupquote{Ctrl+\textquotesingle{}}}
&
\sphinxAtStartPar
Wrap selected text, or word under cursor, in single quotes
\\
\sphinxhline
\sphinxAtStartPar
\sphinxkeyboard{\sphinxupquote{Ctrl+"}}
&
\sphinxAtStartPar
Wrap selected text, or word under cursor, in double quotes
\\
\sphinxhline
\sphinxAtStartPar
\sphinxkeyboard{\sphinxupquote{Ctrl+/}}
&
\sphinxAtStartPar
Toggle comment format for block or selected text
\\
\sphinxhline
\sphinxAtStartPar
\sphinxkeyboard{\sphinxupquote{Ctrl+0}}
&
\sphinxAtStartPar
Remove format for block or selected text
\\
\sphinxhline
\sphinxAtStartPar
\sphinxkeyboard{\sphinxupquote{Ctrl+1}}
&
\sphinxAtStartPar
Change block format to heading level 1
\\
\sphinxhline
\sphinxAtStartPar
\sphinxkeyboard{\sphinxupquote{Ctrl+2}}
&
\sphinxAtStartPar
Change block format to heading level 2
\\
\sphinxhline
\sphinxAtStartPar
\sphinxkeyboard{\sphinxupquote{Ctrl+3}}
&
\sphinxAtStartPar
Change block format to heading level 3
\\
\sphinxhline
\sphinxAtStartPar
\sphinxkeyboard{\sphinxupquote{Ctrl+4}}
&
\sphinxAtStartPar
Change block format to heading level 4
\\
\sphinxhline
\sphinxAtStartPar
\sphinxkeyboard{\sphinxupquote{Ctrl+5}}
&
\sphinxAtStartPar
Change block alignment to left\sphinxhyphen{}aligned
\\
\sphinxhline
\sphinxAtStartPar
\sphinxkeyboard{\sphinxupquote{Ctrl+6}}
&
\sphinxAtStartPar
Change block alignment to centred
\\
\sphinxhline
\sphinxAtStartPar
\sphinxkeyboard{\sphinxupquote{Ctrl+7}}
&
\sphinxAtStartPar
Change block alignment to right\sphinxhyphen{}aligned
\\
\sphinxhline
\sphinxAtStartPar
\sphinxkeyboard{\sphinxupquote{Ctrl+8}}
&
\sphinxAtStartPar
Add a left margin to the block
\\
\sphinxhline
\sphinxAtStartPar
\sphinxkeyboard{\sphinxupquote{Ctrl+9}}
&
\sphinxAtStartPar
Add a right margin to the block
\\
\sphinxhline
\sphinxAtStartPar
\sphinxkeyboard{\sphinxupquote{Ctrl+B}}
&
\sphinxAtStartPar
Format selected text, or word under cursor, with bold
\\
\sphinxhline
\sphinxAtStartPar
\sphinxkeyboard{\sphinxupquote{Ctrl+D}}
&
\sphinxAtStartPar
Format selected text, or word under cursor, with strike through
\\
\sphinxhline
\sphinxAtStartPar
\sphinxkeyboard{\sphinxupquote{Ctrl+I}}
&
\sphinxAtStartPar
Format selected text, or word under cursor, with italic
\\
\sphinxhline
\sphinxAtStartPar
\sphinxkeyboard{\sphinxupquote{Ctrl+Shift+/}}
&
\sphinxAtStartPar
Remove format for block or selected text
\\
\sphinxhline
\sphinxAtStartPar
\sphinxkeyboard{\sphinxupquote{Ctrl+Shift+D}}
&
\sphinxAtStartPar
Toggle ignored text format for block or selected text
\\
\sphinxbottomrule
\end{tabulary}
\sphinxtableafterendhook\par
\sphinxattableend\end{savenotes}


\subsection{Other Editor Shortcuts}
\label{\detokenize{usage_shortcuts:other-editor-shortcuts}}

\begin{savenotes}\sphinxattablestart
\sphinxthistablewithglobalstyle
\centering
\begin{tabulary}{\linewidth}[t]{TT}
\sphinxtoprule
\sphinxstyletheadfamily 
\sphinxAtStartPar
Shortcut
&\sphinxstyletheadfamily 
\sphinxAtStartPar
Description
\\
\sphinxmidrule
\sphinxtableatstartofbodyhook
\sphinxAtStartPar
\sphinxkeyboard{\sphinxupquote{F7}}
&
\sphinxAtStartPar
Re\sphinxhyphen{}run the spell checker on the document
\\
\sphinxhline
\sphinxAtStartPar
\sphinxkeyboard{\sphinxupquote{Ctrl+.}}
&
\sphinxAtStartPar
Open the context menu at the current cursor location
\\
\sphinxhline
\sphinxAtStartPar
\sphinxkeyboard{\sphinxupquote{Ctrl+A}}
&
\sphinxAtStartPar
Select all text in the document
\\
\sphinxhline
\sphinxAtStartPar
\sphinxkeyboard{\sphinxupquote{Ctrl+C}}
&
\sphinxAtStartPar
Copy selected text to clipboard
\\
\sphinxhline
\sphinxAtStartPar
\sphinxkeyboard{\sphinxupquote{Ctrl+K}}
&
\sphinxAtStartPar
Activate the insert commands (see list in {\hyperref[\detokenize{usage_shortcuts:a-kb-ins}]{\sphinxcrossref{\DUrole{std}{\DUrole{std-ref}{Insert Shortcuts}}}}})
\\
\sphinxhline
\sphinxAtStartPar
\sphinxkeyboard{\sphinxupquote{Ctrl+R}}
&
\sphinxAtStartPar
Open or reload the current document in the viewer
\\
\sphinxhline
\sphinxAtStartPar
\sphinxkeyboard{\sphinxupquote{Ctrl+S}}
&
\sphinxAtStartPar
Save the current document
\\
\sphinxhline
\sphinxAtStartPar
\sphinxkeyboard{\sphinxupquote{Ctrl+V}}
&
\sphinxAtStartPar
Paste text from clipboard to cursor position
\\
\sphinxhline
\sphinxAtStartPar
\sphinxkeyboard{\sphinxupquote{Ctrl+W}}
&
\sphinxAtStartPar
Close the current document
\\
\sphinxhline
\sphinxAtStartPar
\sphinxkeyboard{\sphinxupquote{Ctrl+X}}
&
\sphinxAtStartPar
Cut selected text to clipboard
\\
\sphinxhline
\sphinxAtStartPar
\sphinxkeyboard{\sphinxupquote{Ctrl+Y}}
&
\sphinxAtStartPar
Redo latest undo
\\
\sphinxhline
\sphinxAtStartPar
\sphinxkeyboard{\sphinxupquote{Ctrl+Z}}
&
\sphinxAtStartPar
Undo latest changes
\\
\sphinxhline
\sphinxAtStartPar
\sphinxkeyboard{\sphinxupquote{Ctrl+Backspace}}
&
\sphinxAtStartPar
Delete the word before the cursor
\\
\sphinxhline
\sphinxAtStartPar
\sphinxkeyboard{\sphinxupquote{Ctrl+Del}}
&
\sphinxAtStartPar
Delete the word after the cursor
\\
\sphinxhline
\sphinxAtStartPar
\sphinxkeyboard{\sphinxupquote{Ctrl+F7}}
&
\sphinxAtStartPar
Toggle spell checking
\\
\sphinxhline
\sphinxAtStartPar
\sphinxkeyboard{\sphinxupquote{Ctrl+Return}}
&
\sphinxAtStartPar
Open the tag or reference under the cursor in the viewer
\\
\sphinxhline
\sphinxAtStartPar
\sphinxkeyboard{\sphinxupquote{Ctrl+Shift+A}}
&
\sphinxAtStartPar
Select all text in the current paragraph
\\
\sphinxbottomrule
\end{tabulary}
\sphinxtableafterendhook\par
\sphinxattableend\end{savenotes}


\subsection{Insert Shortcuts}
\label{\detokenize{usage_shortcuts:insert-shortcuts}}\label{\detokenize{usage_shortcuts:a-kb-ins}}
\sphinxAtStartPar
A set of insert features are also available through shortcuts, but they require a double
combination of key sequences. The insert feature is activated with \sphinxkeyboard{\sphinxupquote{Ctrl+K}}, followed by
a key or key combination for the inserted content.


\begin{savenotes}
\sphinxatlongtablestart
\sphinxthistablewithglobalstyle
\makeatletter
  \LTleft \@totalleftmargin plus1fill
  \LTright\dimexpr\columnwidth-\@totalleftmargin-\linewidth\relax plus1fill
\makeatother
\begin{longtable}{ll}
\sphinxtoprule
\sphinxstyletheadfamily 
\sphinxAtStartPar
Shortcut
&\sphinxstyletheadfamily 
\sphinxAtStartPar
Description
\\
\sphinxmidrule
\endfirsthead

\multicolumn{2}{c}{\sphinxnorowcolor
    \makebox[0pt]{\sphinxtablecontinued{\tablename\ \thetable{} \textendash{} continued from previous page}}%
}\\
\sphinxtoprule
\sphinxstyletheadfamily 
\sphinxAtStartPar
Shortcut
&\sphinxstyletheadfamily 
\sphinxAtStartPar
Description
\\
\sphinxmidrule
\endhead

\sphinxbottomrule
\multicolumn{2}{r}{\sphinxnorowcolor
    \makebox[0pt][r]{\sphinxtablecontinued{continues on next page}}%
}\\
\endfoot

\endlastfoot
\sphinxtableatstartofbodyhook

\sphinxAtStartPar
\sphinxkeyboard{\sphinxupquote{Ctrl+K}}, \sphinxkeyboard{\sphinxupquote{Space}}
&
\sphinxAtStartPar
Insert a non\sphinxhyphen{}breaking space
\\
\sphinxhline
\sphinxAtStartPar
\sphinxkeyboard{\sphinxupquote{Ctrl+K}}, \sphinxkeyboard{\sphinxupquote{\_}}
&
\sphinxAtStartPar
Insert a long dash (em dash)
\\
\sphinxhline
\sphinxAtStartPar
\sphinxkeyboard{\sphinxupquote{Ctrl+K}}, \sphinxkeyboard{\sphinxupquote{.}}
&
\sphinxAtStartPar
Insert an ellipsis
\\
\sphinxhline
\sphinxAtStartPar
\sphinxkeyboard{\sphinxupquote{Ctrl+K}}, \sphinxkeyboard{\sphinxupquote{\textquotesingle{}}}
&
\sphinxAtStartPar
Insert a modifier apostrophe
\\
\sphinxhline
\sphinxAtStartPar
\sphinxkeyboard{\sphinxupquote{Ctrl+K}}, \sphinxkeyboard{\sphinxupquote{*}}
&
\sphinxAtStartPar
Insert a list bullet
\\
\sphinxhline
\sphinxAtStartPar
\sphinxkeyboard{\sphinxupquote{Ctrl+K}}, \sphinxkeyboard{\sphinxupquote{\%}}
&
\sphinxAtStartPar
Insert a per mille symbol
\\
\sphinxhline
\sphinxAtStartPar
\sphinxkeyboard{\sphinxupquote{Ctrl+K}}, \sphinxkeyboard{\sphinxupquote{\textasciitilde{}}}
&
\sphinxAtStartPar
Insert a figure dash (same width as a number)
\\
\sphinxhline
\sphinxAtStartPar
\sphinxkeyboard{\sphinxupquote{Ctrl+K}}, \sphinxkeyboard{\sphinxupquote{−}}
&
\sphinxAtStartPar
Insert a short dash (en dash)
\\
\sphinxhline
\sphinxAtStartPar
\sphinxkeyboard{\sphinxupquote{Ctrl+K}}, \sphinxkeyboard{\sphinxupquote{1}}
&
\sphinxAtStartPar
Insert a left single quote
\\
\sphinxhline
\sphinxAtStartPar
\sphinxkeyboard{\sphinxupquote{Ctrl+K}}, \sphinxkeyboard{\sphinxupquote{2}}
&
\sphinxAtStartPar
Insert a right single quote
\\
\sphinxhline
\sphinxAtStartPar
\sphinxkeyboard{\sphinxupquote{Ctrl+K}}, \sphinxkeyboard{\sphinxupquote{3}}
&
\sphinxAtStartPar
Insert a left double quote
\\
\sphinxhline
\sphinxAtStartPar
\sphinxkeyboard{\sphinxupquote{Ctrl+K}}, \sphinxkeyboard{\sphinxupquote{4}}
&
\sphinxAtStartPar
Insert a right double quote
\\
\sphinxhline
\sphinxAtStartPar
\sphinxkeyboard{\sphinxupquote{Ctrl+K}}, \sphinxkeyboard{\sphinxupquote{C}}
&
\sphinxAtStartPar
Insert a \sphinxcode{\sphinxupquote{@char}} keyword
\\
\sphinxhline
\sphinxAtStartPar
\sphinxkeyboard{\sphinxupquote{Ctrl+K}}, \sphinxkeyboard{\sphinxupquote{E}}
&
\sphinxAtStartPar
Insert an \sphinxcode{\sphinxupquote{@entity}} keyword
\\
\sphinxhline
\sphinxAtStartPar
\sphinxkeyboard{\sphinxupquote{Ctrl+K}}, \sphinxkeyboard{\sphinxupquote{F}}
&
\sphinxAtStartPar
Insert a \sphinxcode{\sphinxupquote{@focus}} keyword
\\
\sphinxhline
\sphinxAtStartPar
\sphinxkeyboard{\sphinxupquote{Ctrl+K}}, \sphinxkeyboard{\sphinxupquote{G}}
&
\sphinxAtStartPar
Insert a \sphinxcode{\sphinxupquote{@tag}} keyword
\\
\sphinxhline
\sphinxAtStartPar
\sphinxkeyboard{\sphinxupquote{Ctrl+K}}, \sphinxkeyboard{\sphinxupquote{H}}
&
\sphinxAtStartPar
Insert a short description comment
\\
\sphinxhline
\sphinxAtStartPar
\sphinxkeyboard{\sphinxupquote{Ctrl+K}}, \sphinxkeyboard{\sphinxupquote{L}}
&
\sphinxAtStartPar
Insert a \sphinxcode{\sphinxupquote{@location}} keyword
\\
\sphinxhline
\sphinxAtStartPar
\sphinxkeyboard{\sphinxupquote{Ctrl+K}}, \sphinxkeyboard{\sphinxupquote{M}}
&
\sphinxAtStartPar
Insert a \sphinxcode{\sphinxupquote{@mention}} keyword
\\
\sphinxhline
\sphinxAtStartPar
\sphinxkeyboard{\sphinxupquote{Ctrl+K}}, \sphinxkeyboard{\sphinxupquote{O}}
&
\sphinxAtStartPar
Insert an \sphinxcode{\sphinxupquote{@object}} keyword
\\
\sphinxhline
\sphinxAtStartPar
\sphinxkeyboard{\sphinxupquote{Ctrl+K}}, \sphinxkeyboard{\sphinxupquote{P}}
&
\sphinxAtStartPar
Insert a \sphinxcode{\sphinxupquote{@plot}} keyword
\\
\sphinxhline
\sphinxAtStartPar
\sphinxkeyboard{\sphinxupquote{Ctrl+K}}, \sphinxkeyboard{\sphinxupquote{S}}
&
\sphinxAtStartPar
Insert a synopsis comment
\\
\sphinxhline
\sphinxAtStartPar
\sphinxkeyboard{\sphinxupquote{Ctrl+K}}, \sphinxkeyboard{\sphinxupquote{T}}
&
\sphinxAtStartPar
Insert a \sphinxcode{\sphinxupquote{@time}} keyword
\\
\sphinxhline
\sphinxAtStartPar
\sphinxkeyboard{\sphinxupquote{Ctrl+K}}, \sphinxkeyboard{\sphinxupquote{V}}
&
\sphinxAtStartPar
Insert a \sphinxcode{\sphinxupquote{@pov}} keyword
\\
\sphinxhline
\sphinxAtStartPar
\sphinxkeyboard{\sphinxupquote{Ctrl+K}}, \sphinxkeyboard{\sphinxupquote{X}}
&
\sphinxAtStartPar
Insert a \sphinxcode{\sphinxupquote{@custom}} keyword
\\
\sphinxhline
\sphinxAtStartPar
\sphinxkeyboard{\sphinxupquote{Ctrl+K}}, \sphinxkeyboard{\sphinxupquote{Ctrl+Space}}
&
\sphinxAtStartPar
Insert a thin non\sphinxhyphen{}breaking space
\\
\sphinxhline
\sphinxAtStartPar
\sphinxkeyboard{\sphinxupquote{Ctrl+K}}, \sphinxkeyboard{\sphinxupquote{Ctrl+\_}}
&
\sphinxAtStartPar
Insert a horizontal bar (quotation dash)
\\
\sphinxhline
\sphinxAtStartPar
\sphinxkeyboard{\sphinxupquote{Ctrl+K}}, \sphinxkeyboard{\sphinxupquote{Ctrl+\textquotesingle{}}}
&
\sphinxAtStartPar
Insert a prime
\\
\sphinxhline
\sphinxAtStartPar
\sphinxkeyboard{\sphinxupquote{Ctrl+K}}, \sphinxkeyboard{\sphinxupquote{Ctrl+"}}
&
\sphinxAtStartPar
Insert a double prime
\\
\sphinxhline
\sphinxAtStartPar
\sphinxkeyboard{\sphinxupquote{Ctrl+K}}, \sphinxkeyboard{\sphinxupquote{Ctrl+*}}
&
\sphinxAtStartPar
Insert a flower mark (alternative bullet)
\\
\sphinxhline
\sphinxAtStartPar
\sphinxkeyboard{\sphinxupquote{Ctrl+K}}, \sphinxkeyboard{\sphinxupquote{Ctrl+−}}
&
\sphinxAtStartPar
Insert a hyphen bullet (alternative bullet)
\\
\sphinxhline
\sphinxAtStartPar
\sphinxkeyboard{\sphinxupquote{Ctrl+K}}, \sphinxkeyboard{\sphinxupquote{Ctrl+D}}
&
\sphinxAtStartPar
Insert a division sign
\\
\sphinxhline
\sphinxAtStartPar
\sphinxkeyboard{\sphinxupquote{Ctrl+K}}, \sphinxkeyboard{\sphinxupquote{Ctrl+O}}
&
\sphinxAtStartPar
Insert a degree symbol
\\
\sphinxhline
\sphinxAtStartPar
\sphinxkeyboard{\sphinxupquote{Ctrl+K}}, \sphinxkeyboard{\sphinxupquote{Ctrl+X}}
&
\sphinxAtStartPar
Insert a times sign
\\
\sphinxhline
\sphinxAtStartPar
\sphinxkeyboard{\sphinxupquote{Ctrl+K}}, \sphinxkeyboard{\sphinxupquote{Shift+Space}}
&
\sphinxAtStartPar
Insert a thin space
\\
\sphinxbottomrule
\end{longtable}
\sphinxtableafterendhook
\sphinxatlongtableend
\end{savenotes}


\section{Document Viewer Shortcuts}
\label{\detokenize{usage_shortcuts:document-viewer-shortcuts}}\label{\detokenize{usage_shortcuts:a-kb-viewer}}

\begin{savenotes}\sphinxattablestart
\sphinxthistablewithglobalstyle
\centering
\begin{tabulary}{\linewidth}[t]{TT}
\sphinxtoprule
\sphinxstyletheadfamily 
\sphinxAtStartPar
Shortcut
&\sphinxstyletheadfamily 
\sphinxAtStartPar
Description
\\
\sphinxmidrule
\sphinxtableatstartofbodyhook
\sphinxAtStartPar
\sphinxkeyboard{\sphinxupquote{Alt+Left}}
&
\sphinxAtStartPar
Move backward in the view history
\\
\sphinxhline
\sphinxAtStartPar
\sphinxkeyboard{\sphinxupquote{Alt+Right}}
&
\sphinxAtStartPar
Move forward in the view history
\\
\sphinxhline
\sphinxAtStartPar
\sphinxkeyboard{\sphinxupquote{Ctrl+C}}
&
\sphinxAtStartPar
Copy selected text to clipboard
\\
\sphinxhline
\sphinxAtStartPar
\sphinxkeyboard{\sphinxupquote{Ctrl+Shift+A}}
&
\sphinxAtStartPar
Select all text in the current paragraph
\\
\sphinxhline
\sphinxAtStartPar
\sphinxkeyboard{\sphinxupquote{Ctrl+Shift+R}}
&
\sphinxAtStartPar
Close the document viewer
\\
\sphinxbottomrule
\end{tabulary}
\sphinxtableafterendhook\par
\sphinxattableend\end{savenotes}

\sphinxstepscope


\chapter{Typographical Notes}
\label{\detokenize{usage_typography:typographical-notes}}\label{\detokenize{usage_typography:a-typ}}\label{\detokenize{usage_typography::doc}}
\sphinxAtStartPar
novelWriter has some support for typographical symbols that are not usually easily available in
many text editors. This includes for instance the proper unicode quotation marks, dashes, ellipsis,
thin spaces, etc. All these symbols are available from the \sphinxstylestrong{Insert} menu, and via
keyboard shortcuts. See {\hyperref[\detokenize{usage_shortcuts:a-kb-ins}]{\sphinxcrossref{\DUrole{std}{\DUrole{std-ref}{Insert Shortcuts}}}}}.

\sphinxAtStartPar
This chapter provides some additional information on how novelWriter handles these symbols.


\section{Special Notes on Symbols}
\label{\detokenize{usage_typography:special-notes-on-symbols}}\label{\detokenize{usage_typography:a-typ-notes}}
\sphinxAtStartPar
This section contains additional notes on the available special symbols.


\subsection{Dashes and Ellipsis}
\label{\detokenize{usage_typography:dashes-and-ellipsis}}
\sphinxAtStartPar
With the auto\sphinxhyphen{}replace feature enabled (see {\hyperref[\detokenize{usage_writing:a-ui-edit-auto}]{\sphinxcrossref{\DUrole{std}{\DUrole{std-ref}{Auto\sphinxhyphen{}Replace as You Type}}}}}), multiple hyphens are converted
automatically to short and long dashes, and three dots to ellipsis. The last auto\sphinxhyphen{}replace can
always be reverted with the undo command \sphinxkeyboard{\sphinxupquote{Ctrl+Z}}, reverting the text to what you typed before
the automatic replacement occurred.

\sphinxAtStartPar
In addition, “Figure Dash” is available. The Figure Dash is a dash that has the same width as the
numbers of the same font, for most fonts. It helps to align numbers nicely in columns when you need
to use a dash in them.


\subsection{Single and Double Quotes}
\label{\detokenize{usage_typography:single-and-double-quotes}}
\sphinxAtStartPar
All the different quotation marks listed on the \sphinxhref{https://en.wikipedia.org/wiki/Quotation\_mark}{Quotation Mark} Wikipedia page are available, and
can be selected as auto\sphinxhyphen{}replaced symbols for straight single and double quote key strokes. The
settings can be found in \sphinxstylestrong{Preferences}.

\sphinxAtStartPar
Ordinarily, text wrapped in quotes are highlighted by the editor. This is meant as a convenience
for highlighting dialogue between characters. This feature can be disabled in
\sphinxstylestrong{Preferences} if this feature isn’t wanted.

\sphinxAtStartPar
The editor distinguishes between text wrapped in regular straight double quotes and the
user\sphinxhyphen{}selected double quote symbols. This is to help the writer recognise which parts of the text
are not using the chosen quote symbols. Two convenience functions in the \sphinxstylestrong{Format} menu
can be used to re\sphinxhyphen{}format a selected section of text with the correct quote symbols.


\subsection{Single and Double Prime}
\label{\detokenize{usage_typography:single-and-double-prime}}
\sphinxAtStartPar
Both single and double prime symbols are available in the \sphinxstylestrong{Insert} menu. These symbols
are the correct symbols to use for unit symbols for feet, inches, minutes, and seconds. The usage
of these is described in more detail on the Wikipedia \sphinxhref{https://en.wikipedia.org/wiki/Prime\_(symbol)}{Prime} page. They look very similar to single
and double straight quotes, and may be rendered similarly by the font, but they have different
codes. Using these correctly will also prevent the auto\sphinxhyphen{}replace and dialogue highlighting features
misunderstanding their meaning in the text.


\subsection{Modifier Letter Apostrophe}
\label{\detokenize{usage_typography:id1}}
\sphinxAtStartPar
The auto\sphinxhyphen{}replace feature will consider any right\sphinxhyphen{}facing single straight quote as a quote symbol,
even if it is intended as an apostrophe. This also includes the syntax highlighter, which may
assume the first following apostrophe is the closing symbol of a single quoted region of text.

\sphinxAtStartPar
To get around this, an alternative apostrophe is available. It is a special Unicode character that
is not categorised as punctuation, but as a modifier. It is usually rendered the same way as the
right single quotation marks, depending on the font. There is a Wikipedia article for the
\sphinxhref{https://en.wikipedia.org/wiki/Modifier\_letter\_apostrophe}{Modifier letter apostrophe} with more details.

\begin{sphinxadmonition}{note}{Note:}
\sphinxAtStartPar
On export with the \sphinxstylestrong{Build Manuscript} tool, these apostrophes will be replaced
automatically with the corresponding right hand single quote symbol as is generally recommended.
Therefore it doesn’t really matter if you only use them to correct syntax highlighting.
\end{sphinxadmonition}


\subsection{Special Space Symbols}
\label{\detokenize{usage_typography:special-space-symbols}}
\sphinxAtStartPar
A few variations of the regular space character is supported. The correct typographical way to
separate a number from its unit is with a \sphinxhref{https://en.wikipedia.org/wiki/Thin\_space}{thin space}. It is usually 2/3 the width of a regular
space. For numbers and units, this should in addition be a non\sphinxhyphen{}breaking space, that is, the text
wrapping should not add a line break on this particular space.

\sphinxAtStartPar
A regular space can also be made into a non\sphinxhyphen{}breaking space if needed.

\sphinxAtStartPar
All non\sphinxhyphen{}breaking spaces are highlighted with a differently coloured background to make it easier to
spot them in the text. The colour will depend on the selected colour theme.

\sphinxAtStartPar
The thin and non\sphinxhyphen{}breaking spaces are converted to their corresponding HTML codes on export to HTML
format.

\sphinxstepscope


\chapter{Managing Projects}
\label{\detokenize{project_overview:managing-projects}}\label{\detokenize{project_overview:a-proj}}\label{\detokenize{project_overview::doc}}
\sphinxAtStartPar
Your text in novelWriter is organised into projects. Each project is meant to contain one novel
and associated notes. If you have multiple novels in a series, with the same characters and shared
notes, it is also possible to keep all of them in the same project.


\section{Creating Project}
\label{\detokenize{project_overview:creating-project}}\label{\detokenize{project_overview:a-proj-new}}
\sphinxAtStartPar
New projects can be created from the \sphinxstylestrong{Project} menu by selecting \sphinxstylestrong{Create or Open Project}. This
will open the \sphinxstylestrong{Welcome} dialog, where you can select the \sphinxguilabel{New} button that will assist
you in creating a new project. This dialog is also displayed when you start novelWriter.

\sphinxAtStartPar
A novelWriter project requires a dedicated folder for storing its files on the local file system.
If you’re interested in the details, you can have a look at the chapter {\hyperref[\detokenize{tech_storage:a-storage}]{\sphinxcrossref{\DUrole{std}{\DUrole{std-ref}{How Data is Stored}}}}}.

\sphinxAtStartPar
A list of recently opened projects is maintained, and displayed in the \sphinxstylestrong{Welcome} dialog. A
project can be removed from this list by selecting it and pressing the \sphinxkeyboard{\sphinxupquote{Del}} key or by
right\sphinxhyphen{}clicking it and selecting the \sphinxstylestrong{Remove Project} option.

\begin{figure}[htbp]
\centering
\capstart

\noindent\sphinxincludegraphics{{fig_welcome}.jpg}
\caption{The project list (left) and new project form (right) of the \sphinxguilabel{Welcome} dialog.}\label{\detokenize{project_overview:id1}}\end{figure}

\sphinxAtStartPar
Project\sphinxhyphen{}specific settings are available in \sphinxstylestrong{Project Settings} in the \sphinxstylestrong{Project} menu. See
further details below in the {\hyperref[\detokenize{project_overview:a-proj-settings}]{\sphinxcrossref{\DUrole{std}{\DUrole{std-ref}{Project Settings}}}}} section.

\sphinxAtStartPar
Details about the project’s novel text, including word counts, and a table of contents with word
and page counts, is available through the \sphinxstylestrong{Novel Details} dialog. Statistics about the project
is also available in the \sphinxstylestrong{Manuscript Build} tool.


\subsection{Template Projects}
\label{\detokenize{project_overview:template-projects}}
\sphinxAtStartPar
From the Welcome dialog you can also create a new from another existing project. If you have a
specific structure you want to use for all your new projects, you can create a dedicated project to
be used as a template, and select to copy an existing project from the :guilabel:”Prefill Project”
option from the \sphinxstylestrong{New Project} form.


\section{Project Structure}
\label{\detokenize{project_overview:project-structure}}\label{\detokenize{project_overview:a-proj-roots}}
\sphinxAtStartPar
Projects are structured into a set of top level folders called “Root Folders”. They are visible in
the project tree at the left side of the main window.

\sphinxAtStartPar
The {\hyperref[\detokenize{int_glossary:term-Novel-Documents}]{\sphinxtermref{\DUrole{xref}{\DUrole{std}{\DUrole{std-term}{novel documents}}}}}} go into a root folder of type \sphinxstylestrong{Novel}. {\hyperref[\detokenize{int_glossary:term-Project-Notes}]{\sphinxtermref{\DUrole{xref}{\DUrole{std}{\DUrole{std-term}{Project notes}}}}}} go into
the other root folders. These other root folder types are intended for your notes on the various
elements of your story. Using them is of course entirely optional.

\sphinxAtStartPar
A new project may not have all of the root folders present, but you can add the ones you want from
the project tree tool bar.

\sphinxAtStartPar
Each root folder has one or more {\hyperref[\detokenize{int_glossary:term-Reference}]{\sphinxtermref{\DUrole{xref}{\DUrole{std}{\DUrole{std-term}{reference}}}}}} {\hyperref[\detokenize{int_glossary:term-Keyword}]{\sphinxtermref{\DUrole{xref}{\DUrole{std}{\DUrole{std-term}{keyword}}}}}} associated with it that is used
to reference them from other documents and notes. The intended usage of each type of root folder is
listed below. However, aside from the \sphinxstylestrong{Novel} folder, no restrictions are applied by the
application on what you put in them. You can use them however you want.

\sphinxAtStartPar
The root folder system is closely connected to how the Tags and References system works. For more
details, see the {\hyperref[\detokenize{project_references:a-references}]{\sphinxcrossref{\DUrole{std}{\DUrole{std-ref}{Tags and References}}}}} chapter.


\subsection{Root Folder Types}
\label{\detokenize{project_overview:root-folder-types}}\begin{description}
\sphinxlineitem{\sphinxstylestrong{Novel}}
\sphinxAtStartPar
This is the root folder type for text that goes into the final novel or novels. This class of
documents have other rules and features than the project notes. See {\hyperref[\detokenize{project_structure:a-struct}]{\sphinxcrossref{\DUrole{std}{\DUrole{std-ref}{Novel Structure}}}}} for more
details.

\sphinxlineitem{\sphinxstylestrong{Plot}}
\sphinxAtStartPar
This is the root folder type where main plots can be outlined. It is optional, but adding at
least brief notes can be useful in order to tag plot elements for the \sphinxstylestrong{Outline View}. Tags in
this folder can be references using the \sphinxcode{\sphinxupquote{@plot}} keyword.

\sphinxlineitem{\sphinxstylestrong{Characters}}
\sphinxAtStartPar
Character notes go in this root folder type. These are especially important if you want to use
the \sphinxstylestrong{Outline View} to see which character appears where, which part of the story is told from
a specific character’s point\sphinxhyphen{}of\sphinxhyphen{}view, or focusing on a particular character’s storyline.

\sphinxAtStartPar
The character names can also be inserted into for instance chapter titles when you create your
manuscript. Tags in this type of folder can be referenced using the \sphinxcode{\sphinxupquote{@pov}} keyword for
point\sphinxhyphen{}of\sphinxhyphen{}view characters, \sphinxcode{\sphinxupquote{@focus}} for a focus character, or the \sphinxcode{\sphinxupquote{@char}} keyword for any
other character present.

\sphinxlineitem{\sphinxstylestrong{Locations}}
\sphinxAtStartPar
The locations folder type is for various scene locations that you want to track. Tags in this
folder can be references using the \sphinxcode{\sphinxupquote{@location}} keyword.

\sphinxlineitem{\sphinxstylestrong{Timeline}}
\sphinxAtStartPar
If the story has multiple plot timelines or jumps in time within the same plot, this folder type
can be used to track this. Tags in this type of folder can be references using the \sphinxcode{\sphinxupquote{@time}}
keyword.

\sphinxlineitem{\sphinxstylestrong{Objects}}
\sphinxAtStartPar
Important objects in the story, for instance physical objects that change hands often, can be
tracked here. Tags in this type of folder can be references using the \sphinxcode{\sphinxupquote{@object}} keyword.

\sphinxlineitem{\sphinxstylestrong{Entities}}
\sphinxAtStartPar
Does your plot have many powerful organisations or companies? Or other entities that are part of
the plot? They can be organised here. Tags in this type of folder can be references using the
\sphinxcode{\sphinxupquote{@entity}} keyword.

\sphinxlineitem{\sphinxstylestrong{Custom}}
\sphinxAtStartPar
The custom root folder type can be used for tracking anything else not covered by the above
options. Tags in this folder type can be references using the \sphinxcode{\sphinxupquote{@custom}} keyword.

\end{description}

\sphinxAtStartPar
The root folders are closely tied to the tags and reference system. Each folder type corresponds to
the categories of tags that can be used to reference them. For more information about the tags
listed, see {\hyperref[\detokenize{project_references:a-references-references}]{\sphinxcrossref{\DUrole{std}{\DUrole{std-ref}{How to Use References}}}}}.

\sphinxAtStartPar
There is also a \sphinxcode{\sphinxupquote{@mention}} keyword that can be used to reference any tag.
See {\hyperref[\detokenize{project_references:a-references-references}]{\sphinxcrossref{\DUrole{std}{\DUrole{std-ref}{How to Use References}}}}} for more details.

\begin{sphinxadmonition}{note}{Note:}
\sphinxAtStartPar
You can rename root folders to whatever you want. However, this doesn’t change the reference
keyword or what they do.
\end{sphinxadmonition}

\sphinxAtStartPar
\DUrole{versionmodified}{\DUrole{added}{Added in version 2.0: }}As of version 2.0, you can make multiple root folders of each kind to split up your project.


\subsection{Deleted Documents}
\label{\detokenize{project_overview:deleted-documents}}\label{\detokenize{project_overview:a-proj-roots-del}}
\sphinxAtStartPar
Deleted documents are moved into a special \sphinxstylestrong{Trash} root folder. Documents in the trash folder can
then be deleted permanently, either individually, or by emptying the trash from the menu. Documents
in the trash folder are removed from the {\hyperref[\detokenize{int_glossary:term-Project-Index}]{\sphinxtermref{\DUrole{xref}{\DUrole{std}{\DUrole{std-term}{project index}}}}}} and cannot be referenced.

\sphinxAtStartPar
A document or a folder can be moved to trash from the \sphinxstylestrong{Project} menu, or by pressing
\sphinxkeyboard{\sphinxupquote{Ctrl+Shift+Del}}. Root folders can only be removed when they are empty.


\subsection{Archived Documents}
\label{\detokenize{project_overview:archived-documents}}\label{\detokenize{project_overview:a-proj-roots-out}}
\sphinxAtStartPar
If you don’t want to delete a document, or put it in the \sphinxstylestrong{Trash} folder where it may be deleted
accidentally, but still want it out of your main project tree, you can create an \sphinxstylestrong{Archive} root
folder instead and move it there. It has the same effect as moving it to \sphinxstylestrong{Trash}, but it is safe
from deletion.

\sphinxAtStartPar
You can drag any document to this folder and preserve its settings. The document will always be
excluded from the \sphinxstylestrong{Build Manuscript} tool. It is also removed from the {\hyperref[\detokenize{int_glossary:term-Project-Index}]{\sphinxtermref{\DUrole{xref}{\DUrole{std}{\DUrole{std-term}{project index}}}}}}, so
the tags and references defined in it will not show up anywhere else.


\subsection{Using Folders in the Project Tree}
\label{\detokenize{project_overview:using-folders-in-the-project-tree}}\label{\detokenize{project_overview:a-proj-roots-dirs}}
\sphinxAtStartPar
Regular folders, those that are not root folders, have no structural significance to the project.
When novelWriter is processing the documents in a project, like for instance when you create a
manuscript from it, these folders are ignored. Only the order of the documents themselves matter.

\sphinxAtStartPar
The folders are there purely as a way for you to organise the documents in meaningful sections and
to be able to collapse and hide them in the project tree when you’re not working on those
documents.

\sphinxAtStartPar
\DUrole{versionmodified}{\DUrole{added}{Added in version 2.0: }}As of version 2.0 it is possible to add child documents to other documents. This is particularly
useful when you create chapters and scenes. If you add separate scene documents, you should also
add separate chapter documents, even if they only contain a chapter heading. You can then add
scene documents as child items to the chapters.


\subsection{Recovered Documents}
\label{\detokenize{project_overview:recovered-documents}}\label{\detokenize{project_overview:a-proj-roots-orphaned}}
\sphinxAtStartPar
If novelWriter crashes or otherwise exits without saving the project state, or if you’re using a
file synchronisation tool that runs out of sync, there may be files in the project storage folder
that aren’t tracked in the core project file. These files, when discovered, are recovered and added
back into the project.

\sphinxAtStartPar
The discovered files are scanned for metadata that give clues as to where the document may
previously have been located in the project. The project loading routine will try to put them back
as close as possible to this location, if it still exists. Generally, it will be appended to the
end of the folder where it previously was located. If that folder doesn’t exist, it will try to add
it to the correct root folder type. If it cannot figure out which root folder is correct, the
document will be added to the \sphinxstylestrong{Novel} root folder. Finally, if a \sphinxstylestrong{Novel} does not exist, one
will be created.

\sphinxAtStartPar
If the title of the document can be recovered, the word “Recovered:” will be added as a prefix to
indicate that it may need further attention. If the title cannot be determined, the document will
be named after its internal key, which is a string of characters and numbers.


\subsection{Project Lockfile}
\label{\detokenize{project_overview:project-lockfile}}\label{\detokenize{project_overview:a-proj-roots-lock}}
\sphinxAtStartPar
To prevent lost documents caused by file conflicts when novelWriter projects are synchronised via
file synchronisation tools, a project lockfile is written to the project storage folder when a
project is open. If you try to open a project which already has such a file present, you will be
presented with a warning, and some information about where else novelWriter thinks the project is
also open. You will be given the option to ignore this warning, and continue opening the project at
your own risk.

\begin{sphinxadmonition}{note}{Note:}
\sphinxAtStartPar
If, for some reason, novelWriter or your computer crashes, the lock file may remain even if
there are no other instances keeping the project open. In such a case it is safe to ignore the
lock file warning when re\sphinxhyphen{}opening the project.
\end{sphinxadmonition}

\begin{sphinxadmonition}{warning}{Warning:}
\sphinxAtStartPar
If you choose to ignore the warning and continue opening the project, and multiple instances of
the project are in fact open, you are likely to cause inconsistencies and create diverging
project files, potentially resulting in loss of data and orphaned files. You are not likely to
lose any actual text unless both instances have the same document open in the editor, and
novelWriter will try to resolve project inconsistencies the next time you open the project.
\end{sphinxadmonition}


\section{Project Documents}
\label{\detokenize{project_overview:project-documents}}\label{\detokenize{project_overview:a-proj-files}}
\sphinxAtStartPar
New documents can be created from the toolbar in the project tree, or by pressing \sphinxkeyboard{\sphinxupquote{Ctrl+N}}.
This will open the create new item menu and let you choose between a number of pre\sphinxhyphen{}defined
documents and folders. You will be prompted for a label for the new item.

\sphinxAtStartPar
You can always rename an item by selecting \sphinxstylestrong{Rename Item} from the \sphinxstylestrong{Project} menu, or by
pressing \sphinxkeyboard{\sphinxupquote{F2}} when a document or folder is selected.

\sphinxAtStartPar
Other settings for project documents and folders are available from the context menu that you can
activate by right\sphinxhyphen{}clicking on an it in the tree. The \sphinxstylestrong{Transform} submenu includes options for
converting, splitting, or merging documents. See {\hyperref[\detokenize{usage_project:a-ui-tree-split-merge}]{\sphinxcrossref{\DUrole{std}{\DUrole{std-ref}{Splitting and Merging Documents}}}}} for more details on
the latter two.


\subsection{Document Templates}
\label{\detokenize{project_overview:document-templates}}
\sphinxAtStartPar
If you wish to create template documents to be used when creating new project documents, like for
instance a character note template, you can add a \sphinxstylestrong{Templates} root folder to your project. Any
document added to this root folder will show up in the \sphinxstylestrong{Add Item} menu in the project tree
toolbar. When selected, a new document is created with its content copied from the chosen template.

\sphinxAtStartPar
\DUrole{versionmodified}{\DUrole{added}{Added in version 2.3.}}


\subsection{Word Counts}
\label{\detokenize{project_overview:word-counts}}\label{\detokenize{project_overview:a-proj-files-counts}}
\sphinxAtStartPar
A character, word and paragraph count is maintained for each document, as well as for each section
of a document following a {\hyperref[\detokenize{int_glossary:term-Headings}]{\sphinxtermref{\DUrole{xref}{\DUrole{std}{\DUrole{std-term}{heading}}}}}}. The word count and change of words in the
current session is displayed in the footer of any document open in the editor, and all stats are
shown in the details panel below the project tree for any document selected in the project or novel
trees.

\sphinxAtStartPar
The word counts are not updated in real time, but run in the background every few seconds for as
long as the document is being actively edited.

\sphinxAtStartPar
A total project word count is displayed in the status bar. The total count depends on the sum of
the values in the project tree, which again depend on an up to date {\hyperref[\detokenize{int_glossary:term-Project-Index}]{\sphinxtermref{\DUrole{xref}{\DUrole{std}{\DUrole{std-term}{project index}}}}}}. If the
counts seem wrong, a full project word recount can be initiated by rebuilding the project’s index.
Either from the \sphinxstylestrong{Tools} menu, or by pressing \sphinxkeyboard{\sphinxupquote{F9}}.

\sphinxAtStartPar
The rules for how the counts are made is covered in more detail in {\hyperref[\detokenize{more_counting:a-counting}]{\sphinxcrossref{\DUrole{std}{\DUrole{std-ref}{Word and Text Counts}}}}}.


\section{Project Settings}
\label{\detokenize{project_overview:project-settings}}\label{\detokenize{project_overview:a-proj-settings}}
\sphinxAtStartPar
The \sphinxstylestrong{Project Settings} can be accessed from the \sphinxstylestrong{Project} menu, or by pressing
\sphinxkeyboard{\sphinxupquote{Ctrl+Shift+,}}. This will open a dialog box, with a set of tabs.


\subsection{Settings Tab}
\label{\detokenize{project_overview:settings-tab}}
\sphinxAtStartPar
The \sphinxstylestrong{Settings} tab holds the project name, author, and language settings.

\sphinxAtStartPar
The \sphinxstylestrong{Project Name} can be edited here. It is used for the main window title and for generating
backup files. So keep in mind that if you do change this setting, the backup file names will change
too.

\sphinxAtStartPar
You can also change the \sphinxstylestrong{Authors} and \sphinxstylestrong{Project Language} setting. These are only used when
building the manuscript, for some formats. The language setting is also used when inserting text
into documents in the viewer, like for instance labels for keywords and special comments.

\sphinxAtStartPar
If your project is in a different language than your main spell checking language is set to, you
can override the default setting here. The project language can also be changed from the \sphinxstylestrong{Tools}
menu.

\sphinxAtStartPar
You can also override the automatic backup setting for the project if you wish.


\subsection{Status and Importance Tabs}
\label{\detokenize{project_overview:status-and-importance-tabs}}
\sphinxAtStartPar
Each document or folder of type \sphinxstylestrong{Novel} can be given a “Status” label accompanied by a coloured
icon with an optional shape selected from a list of pre\sphinxhyphen{}defined shapes. Each document or folder of
the remaining types can be given an “Importance” label with the same customisation options.

\sphinxAtStartPar
These labels are there purely for your convenience, and you are not required to use them for any
other features to work. No other part of novelWriter accesses this information. The intention is to
use these to indicate at what stage of completion each novel document is, or how important the
content of a note is to the story. You don’t have to use them this way, that’s just what they were
intended for, but you can make them whatever you want.

\sphinxAtStartPar
See also {\hyperref[\detokenize{usage_project:a-ui-tree-status}]{\sphinxcrossref{\DUrole{std}{\DUrole{std-ref}{Document Importance and Status}}}}}.

\begin{sphinxadmonition}{note}{Note:}
\sphinxAtStartPar
The status or importance level currently in use by one or more documents cannot be deleted, but
they can be edited.
\end{sphinxadmonition}


\subsection{Auto\sphinxhyphen{}Replace Tab}
\label{\detokenize{project_overview:auto-replace-tab}}
\sphinxAtStartPar
A set of automatically replaced keywords can be added in this tab. The keywords in the left column
will be replaced by the text in the right column when documents are opened in the viewer. They will
also be applied to manuscript builds.

\sphinxAtStartPar
The auto\sphinxhyphen{}replace feature will replace text in angle brackets that is in this list. The syntax
highlighter will add an alternate colour to text matching the syntax, but it doesn’t check if the
text is in this list.

\begin{sphinxadmonition}{note}{Note:}
\sphinxAtStartPar
A keyword cannot contain spaces. The angle brackets are added by default, and when used in the
text are a part of the keyword to be replaced. This is to ensure that parts of the text aren’t
unintentionally replaced by the content of the list.
\end{sphinxadmonition}


\section{Backup}
\label{\detokenize{project_overview:backup}}\label{\detokenize{project_overview:a-proj-backup}}
\sphinxAtStartPar
An automatic backup system is built into novelWriter. In order to use it, a backup path to where
the backup files are to be stored must be provided in \sphinxstylestrong{Preferences}. The path defaults to a
folder named “Backups” in your home directory.

\sphinxAtStartPar
Backups can be run automatically when a project is closed, which also implies it is run when the
application itself is closed. Backups are date stamped zip files of the project files in the
project folder (files not strictly a part of the project are ignored). The zip archives are stored
in a subfolder of the backup path. The subfolder will have the same name as the \sphinxstylestrong{Project Name} as
defined in {\hyperref[\detokenize{project_overview:a-proj-settings}]{\sphinxcrossref{\DUrole{std}{\DUrole{std-ref}{Project Settings}}}}}.

\sphinxAtStartPar
The backup feature, when configured, can also be run manually from the \sphinxstylestrong{Tools} menu. It is also
possible to disable automated backups for a given project in \sphinxstylestrong{Project Settings}.

\begin{sphinxadmonition}{note}{Note:}
\sphinxAtStartPar
For the backup to be able to run, the \sphinxstylestrong{Project Name} must be set in \sphinxstylestrong{Project Settings}. This
value is used to generate the name and path of the backups. Without it, the backup will not run
at all, but it will produce a warning message.
\end{sphinxadmonition}


\section{Writing Statistics}
\label{\detokenize{project_overview:writing-statistics}}\label{\detokenize{project_overview:a-proj-stats}}
\sphinxAtStartPar
When you work on a project, a log file records when you opened it, when you closed it, and the
total word counts of your novel documents and notes at the end of the session, provided that the
session lasted either more than 5 minutes, or that the total word count changed. For more details
about the log file, see {\hyperref[\detokenize{tech_storage:a-storage}]{\sphinxcrossref{\DUrole{std}{\DUrole{std-ref}{How Data is Stored}}}}}.

\sphinxAtStartPar
A tool to view the content of the log file is available in the \sphinxstylestrong{Tools} menu under \sphinxstylestrong{Writing
Statistics}. You can also launch it by pressing \sphinxkeyboard{\sphinxupquote{F6}}, or find it on the sidebar.

\sphinxAtStartPar
The tool will show a list of all your sessions, and a set of filters to apply to the data. You can
also export the filtered data to a JSON file or to a CSV file that can be opened by a spreadsheet
application like for instance Libre Office Calc or Excel.

\sphinxAtStartPar
\DUrole{versionmodified}{\DUrole{added}{Added in version 1.2: }}As of version 1.2, the log file also stores how much of the session time was spent idle. The
definition of idle here is that the novelWriter main window loses focus, or the user hasn’t made
any changes to the currently open document in five minutes. The number of minutes can be altered
in \sphinxstylestrong{Preferences}.

\sphinxstepscope


\chapter{Novel Structure}
\label{\detokenize{project_structure:novel-structure}}\label{\detokenize{project_structure:a-struct}}\label{\detokenize{project_structure::doc}}
\sphinxAtStartPar
This chapter covers the structure of a novel project.

\sphinxAtStartPar
There are two different types of documents in a project, \sphinxstylestrong{Novel Documents} and \sphinxstylestrong{Project Notes}.
Active novel documents can only live in a \sphinxstylestrong{Novel} type root folder. You can also move them to
\sphinxstylestrong{Archive} and \sphinxstylestrong{Trash} of course, where they become inactive.

\sphinxAtStartPar
The project tree can distinguish between the different heading levels of the novel documents using
coloured icons, and optionally add emphasis on the label, set in \sphinxstylestrong{Preferences} for easier
identification.


\section{Importance of Headings}
\label{\detokenize{project_structure:importance-of-headings}}\label{\detokenize{project_structure:a-struct-heads}}
\sphinxAtStartPar
Subfolders under root folders have no impact on the structure of the novel itself. The structure is
instead dictated by the heading level of the headings within the documents.

\sphinxAtStartPar
Four levels of headings are supported, signified by the number of hashes (\sphinxcode{\sphinxupquote{\#}}) preceding the
title. See also the {\hyperref[\detokenize{usage_format:a-fmt}]{\sphinxcrossref{\DUrole{std}{\DUrole{std-ref}{Formatting Your Text}}}}} section for more details about the markup syntax.

\begin{sphinxadmonition}{note}{Note:}
\sphinxAtStartPar
The heading levels are not only important when generating the manuscript, they are also used by
the indexer when building the outline tree in the \sphinxstylestrong{Outline View} as well as in the \sphinxstylestrong{Novel
Tree}. Each heading also starts a new region where new Tags and References can be defined. See
{\hyperref[\detokenize{project_references:a-references}]{\sphinxcrossref{\DUrole{std}{\DUrole{std-ref}{Tags and References}}}}} for more details.
\end{sphinxadmonition}

\sphinxAtStartPar
The syntax for the four basic heading types, and the three special types, is listed in section
{\hyperref[\detokenize{usage_format:a-fmt-head}]{\sphinxcrossref{\DUrole{std}{\DUrole{std-ref}{Headings}}}}}. The meaning of the four levels for the structure of your novel is as follows:
\begin{description}
\sphinxlineitem{\sphinxstylestrong{Heading Level 1: Partition}}
\sphinxAtStartPar
This heading level signifies that the text refers to a top level heading. This is useful when
you want to split the manuscript up into books, parts, or acts. These headings are not required.
The novel title itself should use the special heading level \sphinxcode{\sphinxupquote{\#!}} covered in {\hyperref[\detokenize{usage_format:a-fmt-head}]{\sphinxcrossref{\DUrole{std}{\DUrole{std-ref}{Headings}}}}}.

\sphinxlineitem{\sphinxstylestrong{Heading Level 2: Chapter}}
\sphinxAtStartPar
This heading level signifies a chapter. Each time you want to start a new chapter, you must add
such a heading. If you choose to split your manuscript up into one document per scene, you need
a single chapter document with just the heading. You can of course also add a synopsis and
reference keywords to the chapter document. If you want to open the chapter with a quote or
other introductory text that isn’t part of a scene, this is also where you’d put that text.

\sphinxlineitem{\sphinxstylestrong{Heading Level 3: Scene}}
\sphinxAtStartPar
This heading level signifies a scene. You must provide a title text, but the title text can be
replaced with a scene separator or just skipped entirely when you build your manuscript. If you
need to distinguish between hard and soft scene breaks, there is an alternative format for
scenes you can use for this distinction. The formatting is covered in {\hyperref[\detokenize{usage_format:a-fmt-head}]{\sphinxcrossref{\DUrole{std}{\DUrole{std-ref}{Headings}}}}}. See
also {\hyperref[\detokenize{project_structure:a-struct-heads-scenes}]{\sphinxcrossref{\DUrole{std}{\DUrole{std-ref}{Hard and Soft Scene Breaks}}}}}.

\sphinxlineitem{\sphinxstylestrong{Heading Level 4: Section}}
\sphinxAtStartPar
This heading level can be used to split up a scene, usually called a “section” in the
documentation and the user interface. These can be useful if you want to change references
mid\sphinxhyphen{}scene, like if you change the point\sphinxhyphen{}of\sphinxhyphen{}view character. You are free to use sections as you
wish, and you can filter them out of the final manuscript.

\end{description}

\sphinxAtStartPar
Page breaks can be automatically added before partition, chapter and scene headings from the
\sphinxstylestrong{Manuscript Build} tool when you build your project to a format that supports page breaks. If you
want page breaks in other places, you have to specify them manually. See {\hyperref[\detokenize{usage_format:a-fmt-break}]{\sphinxcrossref{\DUrole{std}{\DUrole{std-ref}{Vertical Space and Page Breaks}}}}}.

\begin{sphinxadmonition}{tip}{Tip:}
\sphinxAtStartPar
There are multiple options of how to process novel headings when building the manuscript. For
instance, chapter numbers can be applied automatically, and so can scene numbers if you want
them in a draft manuscript. You can also insert point\sphinxhyphen{}of\sphinxhyphen{}view character names in chapter titles.
See the {\hyperref[\detokenize{project_manuscript:a-manuscript}]{\sphinxcrossref{\DUrole{std}{\DUrole{std-ref}{Building the Manuscript}}}}} page for more details.
\end{sphinxadmonition}


\subsection{Novel Title and Front Matter}
\label{\detokenize{project_structure:novel-title-and-front-matter}}\label{\detokenize{project_structure:a-struct-heads-title}}
\sphinxAtStartPar
It is recommended that you add a document at the very top of each \sphinxstylestrong{Novel} root folder with the
novel title as the first line. You should modify the level 1 heading format code with an \sphinxcode{\sphinxupquote{!}} in
order to render it as a document title that is excluded from any automatic Table of Content in a
manuscript build document, like so:

\begin{sphinxVerbatim}[commandchars=\\\{\}]
\PYG{g+gh}{\PYGZsh{}! My Novel}

\PYGZgt{}\PYGZgt{} \PYG{g+ge}{\PYGZus{}by Jane Doe\PYGZus{}} \PYGZlt{}\PYGZlt{}
\end{sphinxVerbatim}

\sphinxAtStartPar
The title is by default centred on the page. You can add more text to the page as you wish, like
for instance the author’s name and details.

\sphinxAtStartPar
If you want an additional page of text after the title page, starting on a fresh page, you can add
\sphinxcode{\sphinxupquote{{[}new page{]}}} on a line by itself, and continue the text after it. This will insert a page break
before the text. See also {\hyperref[\detokenize{usage_format:a-fmt-break}]{\sphinxcrossref{\DUrole{std}{\DUrole{std-ref}{Vertical Space and Page Breaks}}}}}.


\subsection{Unnumbered Chapter Headings}
\label{\detokenize{project_structure:unnumbered-chapter-headings}}\label{\detokenize{project_structure:a-struct-heads-unnum}}
\sphinxAtStartPar
If you use the automatic numbering feature for your chapters, but you want to keep some special
chapters separate from this, you can add an \sphinxcode{\sphinxupquote{!}} to the level 2 heading formatting code to tell
the build tool to skip these chapters when adding numbers.

\begin{sphinxVerbatim}[commandchars=\\\{\}]
\PYG{g+gu}{\PYGZsh{}\PYGZsh{}! Unnumbered Chapter Title}

Chapter Text
\end{sphinxVerbatim}

\sphinxAtStartPar
There is a separate formatting feature for such chapter titles in the \sphinxstylestrong{Manuscript Build} tool as
well. See the {\hyperref[\detokenize{project_manuscript:a-manuscript}]{\sphinxcrossref{\DUrole{std}{\DUrole{std-ref}{Building the Manuscript}}}}} page for more details. When building a document of a format that
supports page breaks, also unnumbered chapters can have a page break added just like for normal
chapters.


\subsection{Hard and Soft Scene Breaks}
\label{\detokenize{project_structure:hard-and-soft-scene-breaks}}\label{\detokenize{project_structure:a-struct-heads-scenes}}
\sphinxAtStartPar
If you need two different ways to style scenes in your manuscript, like if you want to insert
different scene separators for soft and hard scene breaks, there is an alternative scene format
available for scene headings with a \sphinxcode{\sphinxupquote{!}} added to the formatting code.

\begin{sphinxVerbatim}[commandchars=\\\{\}]
\PYG{g+gu}{\PYGZsh{}\PYGZsh{}\PYGZsh{} Soft Scene Transition}

A soft scene break.

\PYG{g+gu}{\PYGZsh{}\PYGZsh{}\PYGZsh{}! Hard Scene Transition}

A hard scene break.
\end{sphinxVerbatim}

\sphinxAtStartPar
There is a separate formatting feature for these titles in the \sphinxstylestrong{Manuscript Build} tool.

\sphinxAtStartPar
\DUrole{versionmodified}{\DUrole{added}{Added in version 2.4.}}

\sphinxstepscope


\chapter{Tags and References}
\label{\detokenize{project_references:tags-and-references}}\label{\detokenize{project_references:a-references}}\label{\detokenize{project_references::doc}}
\sphinxAtStartPar
In novelWriter there are no forms or tables to fill in to define characters, locations or other
elements of your story. Instead, you create {\hyperref[\detokenize{int_glossary:term-Project-Notes}]{\sphinxtermref{\DUrole{xref}{\DUrole{std}{\DUrole{std-term}{project notes}}}}}} which you can mark as
representing these story elements by creating a {\hyperref[\detokenize{int_glossary:term-Tag}]{\sphinxtermref{\DUrole{xref}{\DUrole{std}{\DUrole{std-term}{tag}}}}}}. Whenever you want to link a piece of
your story to a note defining a story element, like a character, you create a {\hyperref[\detokenize{int_glossary:term-Reference}]{\sphinxtermref{\DUrole{xref}{\DUrole{std}{\DUrole{std-term}{reference}}}}}}
back to that tag. You can also cross\sphinxhyphen{}link your project notes in the same way.

\sphinxAtStartPar
This is perhaps one of the features that makes novelWriter different from other, similar
applications. It is therefore not always obvious to new users how this is supposed to work, so
this chapter hopes to explain in more detail how to use this tags and references system.

\begin{sphinxadmonition}{tip}{Tip:}
\sphinxAtStartPar
If you find the Tags and Reference system difficult to follow just from reading this chapter,
you can create a new project in the \sphinxstylestrong{Welcome} dialog’s New project form and select  “Create an
example project” from the “Pre\sphinxhyphen{}fill project” option. The example project contains several
examples of tags and references.
\end{sphinxadmonition}


\section{Metadata in novelWriter}
\label{\detokenize{project_references:metadata-in-novelwriter}}\label{\detokenize{project_references:a-references-metadata}}
\sphinxAtStartPar
The structure of your novelWriter project is inferred from the {\hyperref[\detokenize{int_glossary:term-Headings}]{\sphinxtermref{\DUrole{xref}{\DUrole{std}{\DUrole{std-term}{headings}}}}}} within the
documents, not the documents themselves. See {\hyperref[\detokenize{project_structure:a-struct-heads}]{\sphinxcrossref{\DUrole{std}{\DUrole{std-ref}{Importance of Headings}}}}} for more details. Therefore,
metadata is also associated with headings, and not the documents directly.

\sphinxAtStartPar
If you split your project into separate documents for each scene, this distinction may not matter.
However, there are several benefits to using documents at a larger structural scale when first
starting your project. For instance, it may make more sense to define all your scenes, and even
chapters, in a single document at first, or perhaps a document per act. You can later split these
documents up using the document split feature. See {\hyperref[\detokenize{usage_project:a-ui-tree-split-merge}]{\sphinxcrossref{\DUrole{std}{\DUrole{std-ref}{Splitting and Merging Documents}}}}} for more details.

\sphinxAtStartPar
You can do the same with your notes. You can treat each heading as an independent element of your
notes that can be referenced somewhere else. That way you can collect all your minor or background
characters in a single note file, and still be able to reference them individually by separating
them with headings and assigning each a tag.


\section{How to Use Tags}
\label{\detokenize{project_references:how-to-use-tags}}\label{\detokenize{project_references:a-references-tags}}
\sphinxAtStartPar
A “tag” in novelWriter is a word or phrase that you define as belonging to a heading. Tags are set
by using the \sphinxcode{\sphinxupquote{@tag}} {\hyperref[\detokenize{int_glossary:term-Keyword}]{\sphinxtermref{\DUrole{xref}{\DUrole{std}{\DUrole{std-term}{keyword}}}}}}.

\sphinxAtStartPar
The basic format of a tag is \sphinxcode{\sphinxupquote{@tag: tagName}}.

\sphinxAtStartPar
An alternative format of a tag is \sphinxcode{\sphinxupquote{@tag: tagName | displayName}}.
\begin{description}
\sphinxlineitem{\sphinxcode{\sphinxupquote{tagName}} (Required)}
\sphinxAtStartPar
This is a unique identifier of your choosing. It is the value you use later for making
references back to this document, or section of the document.

\sphinxlineitem{\sphinxcode{\sphinxupquote{displayName}} (Optional)}
\sphinxAtStartPar
This is an optional display name used for the tag. When you build your manuscript, you can for
instance insert the point of view character name directly into chapter headings. By default, the
\sphinxcode{\sphinxupquote{tagName}} value is used in such headings, but if you use a shortened format internally in your
project, you can use this to specify a more suitable format for your manuscript headings.

\end{description}

\sphinxAtStartPar
You can only set \sphinxstylestrong{one} tag per heading, and the tag has to be unique across \sphinxstylestrong{all} documents in
the project.

\sphinxAtStartPar
After a tag has been defined, it can be referenced in novel documents, or cross\sphinxhyphen{}referenced in other
notes. Tags will also show up in the \sphinxstylestrong{Outline View} and in the references panel under the
document viewer when a document is open in the viewer.

\sphinxAtStartPar
The syntax highlighter will indicate to you that the keyword is correctly used and that the tag is
allowed, that is, the tag is unique. Duplicate tags should be detected as long as the index is up
to date. An invalid tag should have a green wiggly line under it, and will not receive the colour
that valid tags do.

\sphinxAtStartPar
The tag is the only part of these notes that novelWriter uses. The rest of the document content is
there for you to use in whatever way you wish. Of course, the content of the documents can be added
to the manuscript, or an outline document. If you want to compile a single document of all your
notes, you can do this from the \sphinxstylestrong{Manuscript Build} tool.

\sphinxAtStartPar
\DUrole{versionmodified}{\DUrole{added}{Added in version 2.2: }}Tags are no longer case sensitive. The tags are by default displayed with the capitalisation you
use when defining the tag, but you don’t have to use the same capitalisation when referencing
it later.

\sphinxAtStartPar
\DUrole{versionmodified}{\DUrole{added}{Added in version 2.3: }}Tags can have an optional display name for manuscript builds.

\sphinxAtStartPar
Example of a heading with a tag for a character of the story:

\begin{sphinxVerbatim}[commandchars=\\\{\}]
\PYG{g+gh}{\PYGZsh{} Character: Jane Doe}

\PYG{n+ni}{@tag:} Jane | Jane Doe

Some information about the character Jane Doe.
\end{sphinxVerbatim}

\sphinxAtStartPar
When this is done in a document in a {\hyperref[\detokenize{int_glossary:term-Root-Folder}]{\sphinxtermref{\DUrole{xref}{\DUrole{std}{\DUrole{std-term}{Root Folder}}}}}} of type “Characters”, the tag is
automatically treated as an available character in your project with the value “Jane”, and you will
be able to reference it in any of your other documents using the reference keywords for characters.
It will also show up in the Character tab in the Reference panel below the document viewer, and in
the reference auto\sphinxhyphen{}completer menu in the editor when you fill in references. See {\hyperref[\detokenize{usage_writing:a-ui-view}]{\sphinxcrossref{\DUrole{std}{\DUrole{std-ref}{Viewing a Document}}}}}
and {\hyperref[\detokenize{project_references:a-references-completer}]{\sphinxcrossref{\DUrole{std}{\DUrole{std-ref}{The References Auto\sphinxhyphen{}Completer}}}}}.

\sphinxAtStartPar
It is the root folder type that defines what category of story elements the tag is indexed under.
See the {\hyperref[\detokenize{project_overview:a-proj-roots}]{\sphinxcrossref{\DUrole{std}{\DUrole{std-ref}{Project Structure}}}}} section for an overview of available root folder types. They are also
covered in the next section.


\section{How to Use References}
\label{\detokenize{project_references:how-to-use-references}}\label{\detokenize{project_references:a-references-references}}
\sphinxAtStartPar
Each heading of any level in your project can contain references to tags set in project notes. The
references are gathered by the indexer and used to generate the \sphinxstylestrong{Outline View}, among other
things.

\sphinxAtStartPar
References are set as a {\hyperref[\detokenize{int_glossary:term-Keyword}]{\sphinxtermref{\DUrole{xref}{\DUrole{std}{\DUrole{std-term}{keyword}}}}}} and a list of corresponding tags. The valid keywords are
listed below. The format of a reference line is \sphinxcode{\sphinxupquote{@keyword: value1, {[}value2{]} ... {[}valueN{]}}}. All
reference keywords allow multiple values.
\begin{description}
\sphinxlineitem{\sphinxcode{\sphinxupquote{@pov}}}
\sphinxAtStartPar
The point\sphinxhyphen{}of\sphinxhyphen{}view character for the current section. The target must be a note tag in a
\sphinxstylestrong{Character} type root folder.

\sphinxlineitem{\sphinxcode{\sphinxupquote{@focus}}}
\sphinxAtStartPar
The character that has the focus for the current section. This can be used in cases where the
focus is not a point\sphinxhyphen{}of\sphinxhyphen{}view character. The target must be a note tag in a \sphinxstylestrong{Character} type
root folder.

\sphinxlineitem{\sphinxcode{\sphinxupquote{@char}}}
\sphinxAtStartPar
Other characters in the current section. The target must be a note tag in a \sphinxstylestrong{Character} type
root folder. This should not include the point\sphinxhyphen{}of\sphinxhyphen{}view or focus character if those references
are used.

\sphinxlineitem{\sphinxcode{\sphinxupquote{@plot}}}
\sphinxAtStartPar
The plot or subplot advanced in the current section. The target must be a note tag in a \sphinxstylestrong{Plot}
type root folder.

\sphinxlineitem{\sphinxcode{\sphinxupquote{@time}}}
\sphinxAtStartPar
The timelines touched by the current section. The target must be a note tag in a \sphinxstylestrong{Timeline}
type root folder.

\sphinxlineitem{\sphinxcode{\sphinxupquote{@location}}}
\sphinxAtStartPar
The location the current section takes place in. The target must be a note tag in a
\sphinxstylestrong{Locations} type root folder.

\sphinxlineitem{\sphinxcode{\sphinxupquote{@object}}}
\sphinxAtStartPar
Objects present in the current section. The target must be a note tag in a \sphinxstylestrong{Object} type root
folder.

\sphinxlineitem{\sphinxcode{\sphinxupquote{@entity}}}
\sphinxAtStartPar
Entities present in the current section. The target must be a note tag in an \sphinxstylestrong{Entities} type
root folder.

\sphinxlineitem{\sphinxcode{\sphinxupquote{@custom}}}
\sphinxAtStartPar
Custom references in the current section. The target must be a note tag in a \sphinxstylestrong{Custom} type
root folder. The custom folder are for any other category of notes you may want to use.

\sphinxlineitem{\sphinxcode{\sphinxupquote{@mention}}}
\sphinxAtStartPar
Anything mentioned, but not present in the current section. It is intended for those cases where
you reveal details about a character or place in a scene without it being otherwise a part of
it. This can be useful when checking for consistency later. Any tag in any root note folder can
be listed under mentions.

\end{description}

\sphinxAtStartPar
The syntax highlighter will alert the user that the tags and references are used correctly, and
that the tags referenced exist.

\begin{sphinxadmonition}{note}{Note:}
\sphinxAtStartPar
The highlighter may be mistaken if the index of defined tags is out of date. If so, press
\sphinxkeyboard{\sphinxupquote{F9}} to regenerate it, or select \sphinxstylestrong{Rebuild Index} from the \sphinxstylestrong{Tools} menu. In general, the
index for a document is regenerated when it is saved, so this shouldn’t normally be necessary.
\end{sphinxadmonition}

\begin{sphinxadmonition}{tip}{Tip:}
\sphinxAtStartPar
If you add a reference in the editor to a tag that doesn’t yet exist, you can right\sphinxhyphen{}click it and
select \sphinxstylestrong{Create Note for Tag}. This will generate a new project note automatically with the new
tag defined. In order for this to be possible, a root folder for that category of references
must already exist.
\end{sphinxadmonition}

\sphinxAtStartPar
One note can also reference another note in the same way novel documents do. When the note is
opened in the document viewer, the references become clickable links, making it easier to follow
connections in the plot. You can follow links in the document editor by clicking them with the
mouse while holding down the \sphinxkeyboard{\sphinxupquote{Ctrl}} key. Clicked links are always opened in the view panel.

\sphinxAtStartPar
Project notes don’t show up in the \sphinxstylestrong{Outline View}, so referencing between notes is only
meaningful if you want to be able to click\sphinxhyphen{}navigate between them, or of course if you just want to
highlight that two notes are related.

\begin{sphinxadmonition}{tip}{Tip:}
\sphinxAtStartPar
If you cross\sphinxhyphen{}reference between notes and export your project as an HTML document using the
\sphinxstylestrong{Manuscript Build} tool, the cross\sphinxhyphen{}references become clickable links in the exported HTML
document as well.
\end{sphinxadmonition}

\sphinxAtStartPar
Example of a novel document with references to characters and plots:

\begin{sphinxVerbatim}[commandchars=\\\{\}]
\PYG{g+gu}{\PYGZsh{}\PYGZsh{} Chapter 1}

\PYG{n+ni}{@pov:} Jane

\PYG{g+gu}{\PYGZsh{}\PYGZsh{}\PYGZsh{} Scene 1}

\PYG{n+ni}{@char:} John, Sam
\PYG{n+ni}{@plot:} Main

Once upon a time ...
\end{sphinxVerbatim}


\subsection{The References Auto\sphinxhyphen{}Completer}
\label{\detokenize{project_references:the-references-auto-completer}}\label{\detokenize{project_references:a-references-completer}}
\sphinxAtStartPar
An auto\sphinxhyphen{}completer context menu will show up automatically in the document editor when you type the
character \sphinxcode{\sphinxupquote{@}} on a new line. It will first suggest tag or reference keywords for you to add, and
after the \sphinxcode{\sphinxupquote{:}} has been added, suggest references from the list of tags you have already defined.

\sphinxAtStartPar
You can use the auto\sphinxhyphen{}completer to add multiple references with a \sphinxcode{\sphinxupquote{,}} between them, and even type
new ones. New references can be created by right\sphinxhyphen{}clicking on them and selecting \sphinxstylestrong{Create Note for
Tag} from the menu.

\sphinxAtStartPar
\DUrole{versionmodified}{\DUrole{added}{Added in version 2.2.}}

\sphinxstepscope


\chapter{Building the Manuscript}
\label{\detokenize{project_manuscript:building-the-manuscript}}\label{\detokenize{project_manuscript:a-manuscript}}\label{\detokenize{project_manuscript::doc}}
\sphinxAtStartPar
You can at any time build a manuscript, an outline of your notes, or any other type of document
from the text in your project. All of this is handled by the \sphinxstylestrong{Manuscript Build} tool. You can
activate it from the sidebar, the \sphinxstylestrong{Tools} menu, or by pressing \sphinxkeyboard{\sphinxupquote{F5}}.

\sphinxAtStartPar
\DUrole{versionmodified}{\DUrole{added}{Added in version 2.1: }}This tool is new for version 2.1. A simpler tool was used for earlier versions. The simpler tool
only allows you to define a single set of options for the build, but otherwise had much the same
functionality.


\section{The Manuscript Build Tool}
\label{\detokenize{project_manuscript:the-manuscript-build-tool}}\label{\detokenize{project_manuscript:a-manuscript-main}}
\begin{figure}[htbp]
\centering
\capstart

\noindent\sphinxincludegraphics[width=0.800\linewidth]{{fig_manuscript_build}.png}
\caption{The \sphinxstylestrong{Manuscript Build} tool main window.}\label{\detokenize{project_manuscript:id1}}\end{figure}

\sphinxAtStartPar
The main window of the \sphinxstylestrong{Manuscript Build} tool contains a list of all the builds you have
defined, a selection of settings, and a few buttons to generate preview, open the print dialog, or
run the build to create a manuscript document.


\subsection{Outline and Word Counts}
\label{\detokenize{project_manuscript:outline-and-word-counts}}
\begin{figure}[htbp]
\centering
\capstart

\noindent\sphinxincludegraphics[width=0.800\linewidth]{{fig_manuscript_build_outline}.png}
\caption{The \sphinxstylestrong{Manuscript Build} tool main window with the \sphinxstylestrong{Outline} visible.}\label{\detokenize{project_manuscript:id2}}\end{figure}

\sphinxAtStartPar
The \sphinxstylestrong{Outline} tab on the left lets you navigate the headings in the preview document. It will
show up to scene level headings for novel documents, and level 2 headings for notes.

\sphinxAtStartPar
A collapsible panel of word and character counts are also available below the preview document.
These are calculated from the text you have included in the document, and are more accurate counts
than what’s available in the project tree since they are counted \sphinxstyleemphasis{after formatting}.

\sphinxAtStartPar
For a detailed description on how they are counted, see {\hyperref[\detokenize{more_counting:a-counting}]{\sphinxcrossref{\DUrole{std}{\DUrole{std-ref}{Word and Text Counts}}}}}.


\section{Build Settings}
\label{\detokenize{project_manuscript:build-settings}}\label{\detokenize{project_manuscript:a-manuscript-settings}}
\sphinxAtStartPar
Each build definition can be edited by opening it in the \sphinxstylestrong{Manuscript Build Settings} dialog,
either by double\sphinxhyphen{}clicking or by selecting it and pressing the edit button in the toolbar.

\begin{sphinxadmonition}{tip}{Tip:}
\sphinxAtStartPar
You can keep the \sphinxstylestrong{Manuscript Build Settings} dialog open while testing the different options,
and just hit the \sphinxguilabel{Apply} button. You can test the result of your settings by pressing
the \sphinxguilabel{Preview} button in the main \sphinxstylestrong{Manuscript Build} window. When you’re happy with
the result, you can close the settings.
\end{sphinxadmonition}


\subsection{Document Selection}
\label{\detokenize{project_manuscript:document-selection}}
\begin{figure}[htbp]
\centering
\capstart

\noindent\sphinxincludegraphics[width=0.800\linewidth]{{fig_build_settings_selections}.png}
\caption{The \sphinxstylestrong{Selections} page of the \sphinxstylestrong{Manuscript Build Settings} dialog.}\label{\detokenize{project_manuscript:id3}}\end{figure}

\sphinxAtStartPar
The \sphinxstylestrong{Selections} page of the \sphinxstylestrong{Manuscript Build Settings} dialog allows you to fine tune which
documents are included in the build. They are indicated by a green arrow icon in the last column.
On the right you have some filter options for selecting content of a specific type, and a set of
switches for which root folders to include.

\sphinxAtStartPar
You can override the result of these filters by marking one or more documents and selecting to
explicitly include or exclude them by using the buttons below the tree view. The last button can be
used to reset the override and return control to the filter settings.

\sphinxAtStartPar
In the figure, the green arrow icon and the blue pin icon indicates which documents are included,
and the red forbidden icon indicates that a document is explicitly excluded.


\subsection{Formatting Headings}
\label{\detokenize{project_manuscript:formatting-headings}}\label{\detokenize{project_manuscript:a-manuscript-settings-head}}
\begin{figure}[htbp]
\centering
\capstart

\noindent\sphinxincludegraphics[width=0.800\linewidth]{{fig_build_settings_headings}.png}
\caption{The \sphinxstylestrong{Headings} page of the \sphinxstylestrong{Manuscript Build Settings} dialog.}\label{\detokenize{project_manuscript:id4}}\end{figure}

\sphinxAtStartPar
The \sphinxstylestrong{Headings} page of the \sphinxstylestrong{Manuscript Build Settings} dialog allows you to set how the
headings in your {\hyperref[\detokenize{int_glossary:term-Novel-Documents}]{\sphinxtermref{\DUrole{xref}{\DUrole{std}{\DUrole{std-term}{Novel Documents}}}}}} are formatted. By default, the title is just copied as\sphinxhyphen{}is,
indicated by the \sphinxcode{\sphinxupquote{\{Title\}}} format. You can change this to for instance add chapter numbers and
scene numbers, or insert character names, like shown in the figure above.

\sphinxAtStartPar
Clicking the edit button next to a format will copy the formatting string into the edit box where
it can be modified, and where a syntax highlighter will help indicate which parts are automatically
generated by the build tool. The \sphinxguilabel{Insert} button is a dropdown list of these formats, and
selecting one will insert it at the position of the cursor.

\sphinxAtStartPar
Any text you add that isn’t highlighted in colours will remain in your formatted titles.
\sphinxcode{\sphinxupquote{\{Title\}}} will always be replaced by the text in the heading from your documents.

\sphinxAtStartPar
You can preview the result of these format strings by clicking \sphinxguilabel{Apply}, and then clicking
\sphinxguilabel{Preview} in the \sphinxstylestrong{Manuscript Build} tool main window.


\subsubsection{Scene Separators}
\label{\detokenize{project_manuscript:scene-separators}}
\sphinxAtStartPar
If you don’t want any titles for your scenes (or for your sections if you have them), you can leave
the formatting boxes empty. If so, an empty paragraph will be inserted between the scenes or
sections instead, resulting in a gap in the text. You can also switch on the \sphinxguilabel{Hide}
setting, which will ignore them completely. That is, there won’t even be an extra gap inserted.

\sphinxAtStartPar
Alternatively, if you want a separator text between them, like the common \sphinxcode{\sphinxupquote{* * *}}, you can enter
the desired separator text as the format. If the format is any piece of static text, it will always
be treated as a separator.


\subsubsection{Hard and Soft Scenes}
\label{\detokenize{project_manuscript:hard-and-soft-scenes}}\label{\detokenize{project_manuscript:a-manuscript-settings-head-hard}}
\sphinxAtStartPar
If you wish to distinguish between so\sphinxhyphen{}called soft and hard scene breaks, you can use the
alternative scene heading format in your text. You can then give these headings a different
formatting in the \sphinxstylestrong{Headings} settings.

\sphinxAtStartPar
See {\hyperref[\detokenize{usage_format:a-fmt-head}]{\sphinxcrossref{\DUrole{std}{\DUrole{std-ref}{Headings}}}}} for more info on how to format headings in your text.


\subsection{Output Settings}
\label{\detokenize{project_manuscript:output-settings}}
\sphinxAtStartPar
The \sphinxstylestrong{Content}, \sphinxstylestrong{Format} and \sphinxstylestrong{Output} pages of the \sphinxstylestrong{Manuscript Build Settings} dialog
control a number of other settings for the output. Some of these only apply to specific output
formats, which is indicated by the section headings on the settings pages.


\section{Building Manuscript Documents}
\label{\detokenize{project_manuscript:building-manuscript-documents}}\label{\detokenize{project_manuscript:a-manuscript-build}}
\begin{figure}[htbp]
\centering
\capstart

\noindent\sphinxincludegraphics[width=0.800\linewidth]{{fig_build_build}.png}
\caption{The \sphinxstylestrong{Manuscript Build} dialog used for writing the actual manuscript documents.}\label{\detokenize{project_manuscript:id5}}\end{figure}

\sphinxAtStartPar
When you press the \sphinxguilabel{Build} button on the \sphinxstylestrong{Build Manuscript} tool main window, a special
file dialog opens up. This is where you pick your desired output format and where to write the
file.

\sphinxAtStartPar
On the left side of the dialog is a list of all the available file formats, and on the right, a
list of the documents which are included based on the build definition you selected. You can choose
an output path, and set a base file name as well. The file extension will be added automatically.

\sphinxAtStartPar
To generate the manuscript document, press the \sphinxguilabel{Build} button. A small progress bar will
show the build progress, but for small projects it may pass very fast.


\subsection{File Formats}
\label{\detokenize{project_manuscript:file-formats}}
\sphinxAtStartPar
Currently, four document formats are supported.
\begin{description}
\sphinxlineitem{Open Document Format}
\sphinxAtStartPar
The Build tool can produce either an \sphinxcode{\sphinxupquote{.odt}} file, or an \sphinxcode{\sphinxupquote{.fodt}} file. The latter is just a
flat version of the document format as a single XML file. Most rich text editors support the
former, and only a few the latter.

\sphinxlineitem{novelWriter HTML}
\sphinxAtStartPar
The HTML format writes a single \sphinxcode{\sphinxupquote{.htm}} file with minimal style formatting. The HTML document
is suitable for further processing by document conversion tools like \sphinxhref{https://pandoc.org/}{Pandoc}, for importing in
word processors, or for printing from browser.

\sphinxlineitem{novelWriter Markup}
\sphinxAtStartPar
This is simply a concatenation of the project documents selected by the filters into a \sphinxcode{\sphinxupquote{.txt}}
file. The documents are stacked together in the order they appear in the project tree, with
comments, tags, etc. included if they are selected. This is a useful format for exporting the
project for later import back into novelWriter.

\sphinxlineitem{Standard/Extended Markdown}
\sphinxAtStartPar
The Markdown format comes in both Standard and Extended flavour. The \sphinxstyleemphasis{only} difference in terms
of novelWriter functionality is the support for strike through text, which is not supported by
the Standard flavour.

\end{description}


\subsection{Additional Formats}
\label{\detokenize{project_manuscript:additional-formats}}
\sphinxAtStartPar
In addition to the above document formats, the novelWriter HTML and Markup formats can also be
wrapped in a JSON file. These files will have a meta data entry and a body entry. For HTML, also
the accompanying CSS styles used by the preview are included.

\sphinxAtStartPar
The text body is saved in a two\sphinxhyphen{}level list. The outer list contains one entry per document, in the
order they appear in the project tree. Each document is then split up into a list as well, with one
entry per paragraph it contains.

\sphinxAtStartPar
These files are mainly intended for scripted post\sphinxhyphen{}processing for those who want that option. A JSON
file can be imported directly into a Python dict object or a PHP array, to mentions a few options.


\section{Print and PDF}
\label{\detokenize{project_manuscript:print-and-pdf}}\label{\detokenize{project_manuscript:a-manuscript-print}}
\sphinxAtStartPar
The \sphinxguilabel{Print} button allows you to print the content in the preview window. You can either
print to one of your system’s printers, or select PDF as your output format from the printer icon
on the print dialog.

\begin{sphinxadmonition}{note}{Note:}
\sphinxAtStartPar
The paper format should in all cases default to whatever your system default is. If you want to
change it, you have to select it from the \sphinxstylestrong{Print Preview} dialog.
\end{sphinxadmonition}

\sphinxstepscope


\chapter{Customisations}
\label{\detokenize{more_customise:customisations}}\label{\detokenize{more_customise:a-custom}}\label{\detokenize{more_customise::doc}}
\sphinxAtStartPar
There are a few ways you can customise novelWriter yourself. Currently, you can add new GUI themes,
your own syntax themes, and install additional dictionaries.


\section{Spell Check Dictionaries}
\label{\detokenize{more_customise:spell-check-dictionaries}}\label{\detokenize{more_customise:a-custom-dict}}
\sphinxAtStartPar
novelWriter uses \sphinxhref{https://rrthomas.github.io/enchant/}{Enchant} as the spell checking tool. Depending on your operating system, it may or
may not load all installed spell check dictionaries automatically.


\subsection{Linux and MacOS}
\label{\detokenize{more_customise:linux-and-macos}}
\sphinxAtStartPar
On Linux and MacOS, you generally only have to install hunspell, aspell or myspell dictionaries on
your system like you do for other applications. See your distro or OS documentation for how to do
this. These dictionaries should show up as available spell check languages in novelWriter.


\subsection{Windows}
\label{\detokenize{more_customise:windows}}
\sphinxAtStartPar
For Windows, English is included with the installation. For other languages you have to download
and add dictionaries yourself.

\sphinxAtStartPar
\sphinxstylestrong{Install Tool}

\sphinxAtStartPar
A small tool to assist with this can be found under \sphinxstylestrong{Tools \textgreater{} Add Dictionaries}. It will import
spell checking dictionaries from Free Office or Libre Office extensions. The dictionaries are then
installed in the install location for the Enchant library and should thus work for any application
that uses Enchant for spell checking.

\sphinxAtStartPar
\sphinxstylestrong{Manual Install}

\sphinxAtStartPar
If you prefer to do this manually or want to use a different source than the ones mentioned above,
You need to get compatible dictionary files for your language. You need two files files ending with
\sphinxcode{\sphinxupquote{.aff}} and \sphinxcode{\sphinxupquote{.dic}}. These files must then be copied to the following location:

\sphinxAtStartPar
\sphinxcode{\sphinxupquote{C:\textbackslash{}Users\textbackslash{}\textless{}USER\textgreater{}\textbackslash{}AppData\textbackslash{}Local\textbackslash{}enchant\textbackslash{}hunspell}}

\sphinxAtStartPar
This assumes your user profile is stored at \sphinxcode{\sphinxupquote{C:\textbackslash{}Users\textbackslash{}\textless{}USER\textgreater{}}}. The last one or two folders may
not exist, so you may need to create them.

\sphinxAtStartPar
You can find the various dictionaries on the \sphinxhref{https://cgit.freedesktop.org/libreoffice/dictionaries/tree/}{Free Desktop} website.

\begin{sphinxadmonition}{note}{Note:}
\sphinxAtStartPar
The Free Desktop link points to a repository, and what may look like file links inside the
dictionary folder are actually links to web pages. If you right\sphinxhyphen{}click and download those, you
get HTML files, not dictionaries!

\sphinxAtStartPar
In order to download the actual dictionary files, right\sphinxhyphen{}click the “plain” label at the end of
each line and download that.
\end{sphinxadmonition}


\section{Syntax and GUI Themes}
\label{\detokenize{more_customise:syntax-and-gui-themes}}\label{\detokenize{more_customise:a-custom-theme}}
\sphinxAtStartPar
Adding your own GUI and syntax themes is relatively easy, although it requires that you manually
edit config files with colour values. The themes are defined by simple plain text config files with
meta data and colour settings.

\sphinxAtStartPar
In order to make your own versions, first copy one of the existing files to your local computer and
modify it as you like.
\begin{itemize}
\item {} 
\sphinxAtStartPar
The existing syntax themes are stored in
\sphinxhref{https://github.com/vkbo/novelWriter/tree/main/novelwriter/assets/syntax}{novelwriter/assets/syntax}.

\item {} 
\sphinxAtStartPar
The existing GUI themes are stored in
\sphinxhref{https://github.com/vkbo/novelWriter/tree/main/novelwriter/assets/themes}{novelwriter/assets/themes}.

\item {} 
\sphinxAtStartPar
The existing icon themes are stored in
\sphinxhref{https://github.com/vkbo/novelWriter/tree/main/novelwriter/assets/icons}{novelwriter/assets/icons}.

\end{itemize}

\sphinxAtStartPar
Remember to also change the name of your theme by modifying the \sphinxcode{\sphinxupquote{name}} setting at the top of the
file, otherwise you may not be able to distinguish them in \sphinxstylestrong{Preferences}.

\sphinxAtStartPar
For novelWriter to be able to locate the custom theme files, you must copy them to the
{\hyperref[\detokenize{tech_locations:a-locations-data}]{\sphinxcrossref{\DUrole{std}{\DUrole{std-ref}{Application Data}}}}} location in your home or user area. There should be a folder there named
\sphinxcode{\sphinxupquote{syntax}} for syntax themes, just \sphinxcode{\sphinxupquote{themes}} for GUI themes, and \sphinxcode{\sphinxupquote{icons}} for icon themes. These
folders are created the first time you start novelWriter.

\sphinxAtStartPar
Once the files are copied there, they should show up in \sphinxstylestrong{Preferences} with the label you
set as \sphinxcode{\sphinxupquote{name}} inside the file.

\sphinxAtStartPar
\DUrole{versionmodified}{\DUrole{added}{Added in version 2.0: }}The \sphinxcode{\sphinxupquote{icontheme}} value was added to GUI themes. Make sure you set this value in existing custom
themes. Otherwise, novelWriter will try to guess your icon theme, and may not pick the most
suitable one.


\subsection{Custom GUI and Icons Theme}
\label{\detokenize{more_customise:custom-gui-and-icons-theme}}
\sphinxAtStartPar
A GUI theme \sphinxcode{\sphinxupquote{.conf}} file consists of the following settings:

\begin{sphinxVerbatim}[commandchars=\\\{\}]
\PYG{k}{[Main]}
\PYG{n+na}{name}\PYG{+w}{        }\PYG{o}{=}\PYG{+w}{ }\PYG{l+s}{My Custom Theme}
\PYG{n+na}{description}\PYG{+w}{ }\PYG{o}{=}\PYG{+w}{ }\PYG{l+s}{A description of my custom theme}
\PYG{n+na}{author}\PYG{+w}{      }\PYG{o}{=}\PYG{+w}{ }\PYG{l+s}{Jane Doe}
\PYG{n+na}{credit}\PYG{+w}{      }\PYG{o}{=}\PYG{+w}{ }\PYG{l+s}{John Doe}
\PYG{n+na}{url}\PYG{+w}{         }\PYG{o}{=}\PYG{+w}{ }\PYG{l+s}{https://example.com}
\PYG{n+na}{license}\PYG{+w}{     }\PYG{o}{=}\PYG{+w}{ }\PYG{l+s}{CC BY\PYGZhy{}SA 4.0}
\PYG{n+na}{licenseurl}\PYG{+w}{  }\PYG{o}{=}\PYG{+w}{ }\PYG{l+s}{https://creativecommons.org/licenses/by\PYGZhy{}sa/4.0/}
\PYG{n+na}{icontheme}\PYG{+w}{   }\PYG{o}{=}\PYG{+w}{ }\PYG{l+s}{typicons\PYGZus{}light}

\PYG{k}{[Palette]}
\PYG{n+na}{window}\PYG{+w}{          }\PYG{o}{=}\PYG{+w}{ }\PYG{l+s}{100, 100, 100}
\PYG{n+na}{windowtext}\PYG{+w}{      }\PYG{o}{=}\PYG{+w}{ }\PYG{l+s}{100, 100, 100}
\PYG{n+na}{base}\PYG{+w}{            }\PYG{o}{=}\PYG{+w}{ }\PYG{l+s}{100, 100, 100}
\PYG{n+na}{alternatebase}\PYG{+w}{   }\PYG{o}{=}\PYG{+w}{ }\PYG{l+s}{100, 100, 100}
\PYG{n+na}{text}\PYG{+w}{            }\PYG{o}{=}\PYG{+w}{ }\PYG{l+s}{100, 100, 100}
\PYG{n+na}{tooltipbase}\PYG{+w}{     }\PYG{o}{=}\PYG{+w}{ }\PYG{l+s}{100, 100, 100}
\PYG{n+na}{tooltiptext}\PYG{+w}{     }\PYG{o}{=}\PYG{+w}{ }\PYG{l+s}{100, 100, 100}
\PYG{n+na}{button}\PYG{+w}{          }\PYG{o}{=}\PYG{+w}{ }\PYG{l+s}{100, 100, 100}
\PYG{n+na}{buttontext}\PYG{+w}{      }\PYG{o}{=}\PYG{+w}{ }\PYG{l+s}{100, 100, 100}
\PYG{n+na}{brighttext}\PYG{+w}{      }\PYG{o}{=}\PYG{+w}{ }\PYG{l+s}{100, 100, 100}
\PYG{n+na}{highlight}\PYG{+w}{       }\PYG{o}{=}\PYG{+w}{ }\PYG{l+s}{100, 100, 100}
\PYG{n+na}{highlightedtext}\PYG{+w}{ }\PYG{o}{=}\PYG{+w}{ }\PYG{l+s}{100, 100, 100}
\PYG{n+na}{link}\PYG{+w}{            }\PYG{o}{=}\PYG{+w}{ }\PYG{l+s}{100, 100, 100}
\PYG{n+na}{linkvisited}\PYG{+w}{     }\PYG{o}{=}\PYG{+w}{ }\PYG{l+s}{100, 100, 100}

\PYG{k}{[GUI]}
\PYG{n+na}{helptext}\PYG{+w}{        }\PYG{o}{=}\PYG{+w}{   }\PYG{l+s}{0,   0,   0}
\PYG{n+na}{fadedtext}\PYG{+w}{       }\PYG{o}{=}\PYG{+w}{ }\PYG{l+s}{128, 128, 128}
\PYG{n+na}{errortext}\PYG{+w}{       }\PYG{o}{=}\PYG{+w}{ }\PYG{l+s}{255,   0,   0}
\PYG{n+na}{statusnone}\PYG{+w}{      }\PYG{o}{=}\PYG{+w}{ }\PYG{l+s}{120, 120, 120}
\PYG{n+na}{statussaved}\PYG{+w}{     }\PYG{o}{=}\PYG{+w}{   }\PYG{l+s}{2, 133,  37}
\PYG{n+na}{statusunsaved}\PYG{+w}{   }\PYG{o}{=}\PYG{+w}{ }\PYG{l+s}{200,  15,  39}
\end{sphinxVerbatim}

\sphinxAtStartPar
In the Main section you must at least define the \sphinxcode{\sphinxupquote{name}} and \sphinxcode{\sphinxupquote{icontheme}} settings. The
\sphinxcode{\sphinxupquote{icontheme}} settings should correspond to one of the internal icon themes, either
\sphinxcode{\sphinxupquote{typicons\_light}} or \sphinxcode{\sphinxupquote{typicons\_dark}}, or to an icon theme in your custom icons directory. The
setting must match the icon theme’s folder name.

\sphinxAtStartPar
The Palette values correspond to the Qt enum values for \sphinxcode{\sphinxupquote{QPalette::ColorRole}}, see the
\sphinxhref{https://doc.qt.io/qt-5.15/qpalette.html\#ColorRole-enum}{Qt documentation} for more details. The
colour values are RGB numbers on the format \sphinxcode{\sphinxupquote{r, g, b}} where each is an integer from \sphinxcode{\sphinxupquote{0}} to
\sphinxcode{\sphinxupquote{255}}. Omitted values are not loaded and will use default values. If the \sphinxcode{\sphinxupquote{helptext}} colour is
not defined, it is computed as a colour between the \sphinxcode{\sphinxupquote{window}} and \sphinxcode{\sphinxupquote{windowtext}} colour.

\sphinxAtStartPar
\DUrole{versionmodified}{\DUrole{added}{Added in version 2.5: }}The \sphinxcode{\sphinxupquote{fadedtext}} and \sphinxcode{\sphinxupquote{errortext}} theme colour entries were added.


\subsection{Custom Syntax Theme}
\label{\detokenize{more_customise:custom-syntax-theme}}
\sphinxAtStartPar
A syntax theme \sphinxcode{\sphinxupquote{.conf}} file consists of the following settings:

\begin{sphinxVerbatim}[commandchars=\\\{\}]
\PYG{k}{[Main]}
\PYG{n+na}{name}\PYG{+w}{       }\PYG{o}{=}\PYG{+w}{ }\PYG{l+s}{My Syntax Theme}
\PYG{n+na}{author}\PYG{+w}{     }\PYG{o}{=}\PYG{+w}{ }\PYG{l+s}{Jane Doe}
\PYG{n+na}{credit}\PYG{+w}{     }\PYG{o}{=}\PYG{+w}{ }\PYG{l+s}{John Doe}
\PYG{n+na}{url}\PYG{+w}{        }\PYG{o}{=}\PYG{+w}{ }\PYG{l+s}{https://example.com}
\PYG{n+na}{license}\PYG{+w}{    }\PYG{o}{=}\PYG{+w}{ }\PYG{l+s}{CC BY\PYGZhy{}SA 4.0}
\PYG{n+na}{licenseurl}\PYG{+w}{ }\PYG{o}{=}\PYG{+w}{ }\PYG{l+s}{https://creativecommons.org/licenses/by\PYGZhy{}sa/4.0/}

\PYG{k}{[Syntax]}
\PYG{n+na}{background}\PYG{+w}{     }\PYG{o}{=}\PYG{+w}{ }\PYG{l+s}{255, 255, 255}
\PYG{n+na}{text}\PYG{+w}{           }\PYG{o}{=}\PYG{+w}{   }\PYG{l+s}{0,   0,   0}
\PYG{n+na}{link}\PYG{+w}{           }\PYG{o}{=}\PYG{+w}{   }\PYG{l+s}{0,   0,   0}
\PYG{n+na}{headertext}\PYG{+w}{     }\PYG{o}{=}\PYG{+w}{   }\PYG{l+s}{0,   0,   0}
\PYG{n+na}{headertag}\PYG{+w}{      }\PYG{o}{=}\PYG{+w}{   }\PYG{l+s}{0,   0,   0}
\PYG{n+na}{emphasis}\PYG{+w}{       }\PYG{o}{=}\PYG{+w}{   }\PYG{l+s}{0,   0,   0}
\PYG{n+na}{dialog}\PYG{+w}{         }\PYG{o}{=}\PYG{+w}{   }\PYG{l+s}{0,   0,   0}
\PYG{n+na}{altdialog}\PYG{+w}{      }\PYG{o}{=}\PYG{+w}{   }\PYG{l+s}{0,   0,   0}
\PYG{n+na}{note}\PYG{+w}{           }\PYG{o}{=}\PYG{+w}{   }\PYG{l+s}{0,   0,   0}
\PYG{n+na}{hidden}\PYG{+w}{         }\PYG{o}{=}\PYG{+w}{   }\PYG{l+s}{0,   0,   0}
\PYG{n+na}{shortcode}\PYG{+w}{      }\PYG{o}{=}\PYG{+w}{   }\PYG{l+s}{0,   0,   0}
\PYG{n+na}{keyword}\PYG{+w}{        }\PYG{o}{=}\PYG{+w}{   }\PYG{l+s}{0,   0,   0}
\PYG{n+na}{tag}\PYG{+w}{            }\PYG{o}{=}\PYG{+w}{   }\PYG{l+s}{0,   0,   0}
\PYG{n+na}{value}\PYG{+w}{          }\PYG{o}{=}\PYG{+w}{   }\PYG{l+s}{0,   0,   0}
\PYG{n+na}{optional}\PYG{+w}{       }\PYG{o}{=}\PYG{+w}{   }\PYG{l+s}{0,   0,   0}
\PYG{n+na}{spellcheckline}\PYG{+w}{ }\PYG{o}{=}\PYG{+w}{   }\PYG{l+s}{0,   0,   0}
\PYG{n+na}{errorline}\PYG{+w}{      }\PYG{o}{=}\PYG{+w}{   }\PYG{l+s}{0,   0,   0}
\PYG{n+na}{replacetag}\PYG{+w}{     }\PYG{o}{=}\PYG{+w}{   }\PYG{l+s}{0,   0,   0}
\PYG{n+na}{modifier}\PYG{+w}{       }\PYG{o}{=}\PYG{+w}{   }\PYG{l+s}{0,   0,   0}
\PYG{n+na}{texthighlight}\PYG{+w}{  }\PYG{o}{=}\PYG{+w}{ }\PYG{l+s}{255, 255, 255, 128}
\end{sphinxVerbatim}

\sphinxAtStartPar
In the Main section, you must define at least the \sphinxcode{\sphinxupquote{name}} setting. The Syntax colour values are
RGB(A) numbers of the format \sphinxcode{\sphinxupquote{r, g, b, a}} where each is an integer from \sphinxcode{\sphinxupquote{0}} to \sphinxcode{\sphinxupquote{255}}. The
fourth value is the alpha channel, which can be omitted.

\sphinxAtStartPar
Omitted syntax colours default to black, except \sphinxcode{\sphinxupquote{background}} which defaults to white, and
\sphinxcode{\sphinxupquote{texthighlight}} which defaults to white with half transparency.

\sphinxAtStartPar
\DUrole{versionmodified}{\DUrole{added}{Added in version 2.2: }}The \sphinxcode{\sphinxupquote{shortcode}} syntax colour entry was added.

\sphinxAtStartPar
\DUrole{versionmodified}{\DUrole{added}{Added in version 2.3: }}The \sphinxcode{\sphinxupquote{optional}} syntax colour entry was added.

\sphinxAtStartPar
\DUrole{versionmodified}{\DUrole{added}{Added in version 2.4: }}The \sphinxcode{\sphinxupquote{texthighlight}} syntax colour entry was added.

\sphinxAtStartPar
\DUrole{versionmodified}{\DUrole{added}{Added in version 2.5: }}The \sphinxcode{\sphinxupquote{dialog}}, \sphinxcode{\sphinxupquote{altdialog}}, \sphinxcode{\sphinxupquote{note}} and \sphinxcode{\sphinxupquote{tag}} syntax colour entries were added.
\sphinxcode{\sphinxupquote{straightquotes}}, \sphinxcode{\sphinxupquote{doublequotes}} and \sphinxcode{\sphinxupquote{singlequotes}} were removed.

\sphinxstepscope


\chapter{Project Format Changes}
\label{\detokenize{more_projectformat:project-format-changes}}\label{\detokenize{more_projectformat:a-prjfmt}}\label{\detokenize{more_projectformat::doc}}
\sphinxAtStartPar
Most of the changes to the file formats over the history of novelWriter have no impact on the
user side of things. The project files are generally updated automatically. However, some of the
changes require minor actions from the user.

\sphinxAtStartPar
The key changes in the formats are listed in this chapter, as well as the user actions required,
where applicable.

\sphinxAtStartPar
A full project file format specification is available in the online \sphinxhref{https://docs.novelwriter.io/}{documentation}.

\begin{sphinxadmonition}{caution}{Caution:}
\sphinxAtStartPar
When you update a project from one format version to the next, the project can no longer be
opened by a version of novelWriter prior to the version where the new file format was
introduced. You will get a notification about any updates to your project file format and will
have the option to decline the upgrade.
\end{sphinxadmonition}


\section{Format 1.5 Changes}
\label{\detokenize{more_projectformat:format-1-5-changes}}\label{\detokenize{more_projectformat:a-prjfmt-1-5}}
\sphinxAtStartPar
This project format was introduced in novelWriter version 2.0 RC 2.

\sphinxAtStartPar
This is a modification of the 1.4 format. It makes the XML more consistent in that meta data have
been moved to their respective section nodes as attributes, and key/value settings now have a
consistent format. Logical flags are saved as yes/no instead of Python True/False, and the main
heading of the document is now saved to the item rather than in the index. The conversion is done
automatically the first time a project is loaded. No user action is required.


\section{Format 1.4 Changes}
\label{\detokenize{more_projectformat:format-1-4-changes}}\label{\detokenize{more_projectformat:a-prjfmt-1-4}}
\sphinxAtStartPar
This project format was introduced in novelWriter version 2.0 RC 1. Since this was a release
candidate, it is unlikely that your project uses it, but it may be the case if you’ve installed a
pre\sphinxhyphen{}release.

\sphinxAtStartPar
This format changes the way project items (folders, documents and notes) are stored. It is a more
compact format that is simpler and faster to parse, and easier to extend. The conversion is done
automatically the first time a project is loaded. No user action is required.


\section{Format 1.3 Changes}
\label{\detokenize{more_projectformat:format-1-3-changes}}\label{\detokenize{more_projectformat:a-prjfmt-1-3}}
\sphinxAtStartPar
This project format was introduced in novelWriter version 1.5.

\sphinxAtStartPar
With this format, the number of document layouts was reduced from eight to two. The conversion of
document layouts is performed automatically when the project is opened.

\sphinxAtStartPar
Due to the reduction of layouts, some features that were previously controlled by these layouts
will be lost. These features are instead now controlled by syntax codes, so to recover these
features, some minor modification must be made to select documents by the user.

\sphinxAtStartPar
The manual changes the user must make should be very few as they apply to document layouts that
should be used only a few places in any given project. These are as follows:

\sphinxAtStartPar
\sphinxstylestrong{Title Pages}
\begin{itemize}
\item {} 
\sphinxAtStartPar
The formatting of the level one title on the title page must be changed from \sphinxcode{\sphinxupquote{\# Title Text}} to
\sphinxcode{\sphinxupquote{\#! Title Text}} in order to retain the previous functionality. See {\hyperref[\detokenize{usage_format:a-fmt-head}]{\sphinxcrossref{\DUrole{std}{\DUrole{std-ref}{Headings}}}}}.

\item {} 
\sphinxAtStartPar
Any text that was previously centred on the page must be manually centred using the text
alignment feature. See {\hyperref[\detokenize{usage_format:a-fmt-align}]{\sphinxcrossref{\DUrole{std}{\DUrole{std-ref}{Paragraph Alignment and Indentation}}}}}.

\end{itemize}

\sphinxAtStartPar
\sphinxstylestrong{Unnumbered Chapters}
\begin{itemize}
\item {} 
\sphinxAtStartPar
Since the specific layout for unnumbered chapters has been dropped, such chapters must all use
the \sphinxcode{\sphinxupquote{\#\#! Chapter Name}} formatting code instead of \sphinxcode{\sphinxupquote{\#\# Chapter Name}}. This also includes
chapters marked by an asterisk: \sphinxcode{\sphinxupquote{\#\# *Chapter Name}}, as this feature has also been dropped.
See {\hyperref[\detokenize{usage_format:a-fmt-head}]{\sphinxcrossref{\DUrole{std}{\DUrole{std-ref}{Headings}}}}}.

\end{itemize}

\sphinxAtStartPar
\sphinxstylestrong{Plain Pages}
\begin{itemize}
\item {} 
\sphinxAtStartPar
The layout named “Plain Page” has also been removed. The only feature of this layout was that it
ensured that the content always started on a fresh page. In the new format, fresh pages can be
set anywhere in the text with the \sphinxcode{\sphinxupquote{{[}NEW PAGE{]}}} code. See {\hyperref[\detokenize{usage_format:a-fmt-break}]{\sphinxcrossref{\DUrole{std}{\DUrole{std-ref}{Vertical Space and Page Breaks}}}}}.

\end{itemize}


\section{Format 1.2 Changes}
\label{\detokenize{more_projectformat:format-1-2-changes}}\label{\detokenize{more_projectformat:a-prjfmt-1-2}}
\sphinxAtStartPar
This project format was introduced in novelWriter version 0.10.

\sphinxAtStartPar
With this format, the way auto\sphinxhyphen{}replace entries were stored in the main project XML file changed.


\section{Format 1.1 Changes}
\label{\detokenize{more_projectformat:format-1-1-changes}}\label{\detokenize{more_projectformat:a-prjfmt-1-1}}
\sphinxAtStartPar
This project format was introduced in novelWriter version 0.7.

\sphinxAtStartPar
With this format, the \sphinxcode{\sphinxupquote{content}} folder was introduced in the project storage. Previously, all
novelWriter documents were saved in a series of folders numbered from \sphinxcode{\sphinxupquote{data\_0}} to \sphinxcode{\sphinxupquote{data\_f}}.

\sphinxAtStartPar
It also reduces the number of meta data and cache files. These files are automatically deleted if
an old project is opened. This was also when the Table of Contents file was introduced.


\section{Format 1.0 Changes}
\label{\detokenize{more_projectformat:format-1-0-changes}}\label{\detokenize{more_projectformat:a-prjfmt-1-0}}
\sphinxAtStartPar
This is the original file format and project structure. It was in use up to version 0.6.3.

\sphinxstepscope


\chapter{Word and Text Counts}
\label{\detokenize{more_counting:word-and-text-counts}}\label{\detokenize{more_counting:a-counting}}\label{\detokenize{more_counting::doc}}
\sphinxAtStartPar
This is an overview of how words and other counts of your text are performed. The counting rules
should be relatively standard, and are compared to Libre Office Writer rules.

\sphinxAtStartPar
The counts provided in the app on the raw text is meant to be approximate. For more accurate
counts, you need to build your manuscript in the \sphinxstylestrong{Manuscript Tool} and check the counts on the
generated preview.


\section{Text Word Counts and Stats}
\label{\detokenize{more_counting:text-word-counts-and-stats}}
\sphinxAtStartPar
These are the rules for the main counts available for for each document in a project.

\sphinxAtStartPar
For all counts, the following rules apply.
\begin{enumerate}
\sphinxsetlistlabels{\arabic}{enumi}{enumii}{}{.}%
\item {} 
\sphinxAtStartPar
Short (\textendash{}) and long (—) dashes are considered word separators.

\item {} 
\sphinxAtStartPar
Any line starting with \sphinxcode{\sphinxupquote{\%}} or \sphinxcode{\sphinxupquote{@}} is ignored.

\item {} 
\sphinxAtStartPar
Trailing white spaces are ignored, including line breaks.

\item {} 
\sphinxAtStartPar
Leading \sphinxcode{\sphinxupquote{\textgreater{}}} and trailing \sphinxcode{\sphinxupquote{\textless{}}} are ignored with any spaces next to them.

\item {} 
\sphinxAtStartPar
Valid shortcodes and other commands wrapped in brackets \sphinxcode{\sphinxupquote{{[}{]}}} are ignored.

\item {} 
\sphinxAtStartPar
In\sphinxhyphen{}line Markdown syntax in text paragraphs is treated as part of the text.

\end{enumerate}

\sphinxAtStartPar
After the above preparation of the text, the following counts are available.
\begin{description}
\sphinxlineitem{\sphinxstylestrong{Character Count}}
\sphinxAtStartPar
The character count is the sum of characters per line, including leading and in\sphinxhyphen{}text white space
characters, but excluding trailing white space characters. Shortcodes in the text are not
included, but Markdown codes are. Only headings and text are counted.

\sphinxlineitem{\sphinxstylestrong{Word Count}}
\sphinxAtStartPar
The words count is the sum of blocks of continuous character per line separated by any number of
white space characters or dashes. Only headings and text are counted.

\sphinxlineitem{\sphinxstylestrong{Paragraph Count}}
\sphinxAtStartPar
The paragraph count is the number of text blocks separated by one or more empty line. A line
consisting only of white spaces is considered empty.

\end{description}


\section{Manuscript Counts}
\label{\detokenize{more_counting:manuscript-counts}}
\sphinxAtStartPar
These are the rules for the counts available for a manuscript in the \sphinxstylestrong{Manuscript Tool}. The rules
have been tuned to agree with LibreOffice Writer, but will vary slightly depending on the content
of your text. LibreOffice Writer also counts the text in the page header, which the \sphinxstylestrong{Manuscript
Tool} does not.

\sphinxAtStartPar
The content of each line is counted after all formatting has been processed, so the result will be
more accurate than the counts for text documents elsewhere in the app. The following rules apply:
\begin{enumerate}
\sphinxsetlistlabels{\arabic}{enumi}{enumii}{}{.}%
\item {} 
\sphinxAtStartPar
Short (\textendash{}) and long (—) dashes are considered word separators.

\item {} 
\sphinxAtStartPar
Leading and trailing white spaces are generally included, but paragraph breaks are not.

\item {} 
\sphinxAtStartPar
Hard line breaks within paragraph are considered white space characters.

\item {} 
\sphinxAtStartPar
All formatting codes are ignored, including shortcodes, commands and Markdown.

\item {} 
\sphinxAtStartPar
Scene and section separators are counted.

\item {} 
\sphinxAtStartPar
Comments and meta data lines are counted after they are formatted.

\item {} 
\sphinxAtStartPar
Headers are counted after they are formatted with custom formats.

\end{enumerate}

\sphinxAtStartPar
The following counts are available:
\begin{description}
\sphinxlineitem{\sphinxstylestrong{Headings}}
\sphinxAtStartPar
The number of headings in the manuscript.

\sphinxlineitem{\sphinxstylestrong{Paragraphs}}
\sphinxAtStartPar
The number of body text paragraphs in the manuscript.

\sphinxlineitem{\sphinxstylestrong{Words}}
\sphinxAtStartPar
The number of words in the manuscript, including any comments and meta data text.

\sphinxlineitem{\sphinxstylestrong{Words in Text}}
\sphinxAtStartPar
The number of words in body text paragraphs, excluding all other text.

\sphinxlineitem{\sphinxstylestrong{Words in Headings}}
\sphinxAtStartPar
The number of words in headings, including inserted formatting like chapter numbers, etc.

\sphinxlineitem{\sphinxstylestrong{Characters}}
\sphinxAtStartPar
The number of characters in all lines, including any comments and meta data text. Paragraph
breaks are not counted, but in\sphinxhyphen{}paragraph hard line breaks are.

\sphinxlineitem{\sphinxstylestrong{Character in Text}}
\sphinxAtStartPar
The number of characters in body text paragraphs. Paragraph breaks are not counted, but
in\sphinxhyphen{}paragraph hard line breaks are.

\sphinxlineitem{\sphinxstylestrong{Characters in Headings}}
\sphinxAtStartPar
The number of characters in headings.

\sphinxlineitem{\sphinxstylestrong{Character in Text, No Spaces}}
\sphinxAtStartPar
The number of characters in body text paragraphs considered part of a word or punctuation. That
is, white space characters are not counted.

\sphinxlineitem{\sphinxstylestrong{Character in Headings, No Spaces}}
\sphinxAtStartPar
The number of characters in headings considered part of a word or punctuation. That is, white
space characters are not counted.

\end{description}

\sphinxstepscope


\chapter{File Locations}
\label{\detokenize{tech_locations:file-locations}}\label{\detokenize{tech_locations:a-locations}}\label{\detokenize{tech_locations::doc}}
\sphinxAtStartPar
novelWriter will create a few files on your system outside of the application folder itself. These
file locations are described in this chapter.


\section{Configuration}
\label{\detokenize{tech_locations:configuration}}\label{\detokenize{tech_locations:a-locations-conf}}
\sphinxAtStartPar
The general configuration of novelWriter, including everything that is in \sphinxstylestrong{Preferences}, is saved
in one central configuration file. The location of this file depends on your operating system. The
system paths are provided by the Qt \sphinxhref{https://doc.qt.io/qt-5/qstandardpaths.html}{QStandardPaths} class and its \sphinxcode{\sphinxupquote{ConfigLocation}} value.

\sphinxAtStartPar
The standard paths are:
\begin{itemize}
\item {} 
\sphinxAtStartPar
Linux: \sphinxcode{\sphinxupquote{\textasciitilde{}/.config/novelwriter/novelwriter.conf}}

\item {} 
\sphinxAtStartPar
MacOS: \sphinxcode{\sphinxupquote{\textasciitilde{}/Library/Preferences/novelwriter/novelwriter.conf}}

\item {} 
\sphinxAtStartPar
Windows: \sphinxcode{\sphinxupquote{C:\textbackslash{}Users\textbackslash{}\textless{}USER\textgreater{}\textbackslash{}AppData\textbackslash{}Local\textbackslash{}novelwriter\textbackslash{}novelwriter.conf}}

\end{itemize}

\sphinxAtStartPar
Here, \sphinxcode{\sphinxupquote{\textasciitilde{}}} corresponds to the user’s home directory on Linux and MacOS, and \sphinxcode{\sphinxupquote{\textless{}USER\textgreater{}}} is the
user’s username on Windows.

\begin{sphinxadmonition}{note}{Note:}
\sphinxAtStartPar
These are the standard operating system defined locations. If your system has been set up in a
different way, these locations may also be different.
\end{sphinxadmonition}


\section{Application Data}
\label{\detokenize{tech_locations:application-data}}\label{\detokenize{tech_locations:a-locations-data}}
\sphinxAtStartPar
novelWriter also stores a bit of data that is generated by the user’s actions. This includes the
list of recent projects form the \sphinxstylestrong{Welcome} dialog. Custom themes should also be saved here. The
system paths are provided by the Qt \sphinxhref{https://doc.qt.io/qt-5/qstandardpaths.html}{QStandardPaths} class and its \sphinxcode{\sphinxupquote{AppDataLocation}} value.

\sphinxAtStartPar
The standard paths are:
\begin{itemize}
\item {} 
\sphinxAtStartPar
Linux: \sphinxcode{\sphinxupquote{\textasciitilde{}/.local/share/novelwriter/}}

\item {} 
\sphinxAtStartPar
MacOS: \sphinxcode{\sphinxupquote{\textasciitilde{}/Library/Application Support/novelwriter/}}

\item {} 
\sphinxAtStartPar
Windows: \sphinxcode{\sphinxupquote{C:\textbackslash{}Users\textbackslash{}\textless{}USER\textgreater{}\textbackslash{}AppData\textbackslash{}Roaming\textbackslash{}novelwriter\textbackslash{}}}

\end{itemize}

\sphinxAtStartPar
Here, \sphinxcode{\sphinxupquote{\textasciitilde{}}} corresponds to the user’s home directory on Linux and MacOS, and \sphinxcode{\sphinxupquote{\textless{}USER\textgreater{}}} is the
user’s username on Windows.

\begin{sphinxadmonition}{note}{Note:}
\sphinxAtStartPar
These are the standard operating system defined locations. If your system has been set up in a
different way, these locations may also be different.
\end{sphinxadmonition}

\sphinxAtStartPar
The Application Data location also holds several folders:
\begin{description}
\sphinxlineitem{\sphinxcode{\sphinxupquote{cache}}}
\sphinxAtStartPar
This folder is used to save the preview data for the \sphinxstylestrong{Manuscript Build} tool.

\sphinxlineitem{\sphinxcode{\sphinxupquote{icons}}, \sphinxcode{\sphinxupquote{syntax}} and \sphinxcode{\sphinxupquote{themes}}}
\sphinxAtStartPar
These folders are empty by default, but this is where the user can store custom theme files.
See {\hyperref[\detokenize{more_customise:a-custom}]{\sphinxcrossref{\DUrole{std}{\DUrole{std-ref}{Customisations}}}}} for more details.

\end{description}

\sphinxstepscope


\chapter{How Data is Stored}
\label{\detokenize{tech_storage:how-data-is-stored}}\label{\detokenize{tech_storage:a-storage}}\label{\detokenize{tech_storage::doc}}
\sphinxAtStartPar
This chapter contains details of how novelWriter stores and handles the project data.


\section{Project Structure}
\label{\detokenize{tech_storage:project-structure}}
\sphinxAtStartPar
All novelWriter files are written with utf\sphinxhyphen{}8 encoding. Since Python automatically converts Unix
line endings to Windows line endings on Windows systems, novelWriter does not make any adaptations
to the formatting on Windows systems. This is handled entirely by the Python standard library.
Python also handles this when working on the same files on both Windows and Unix\sphinxhyphen{}based operating
systems.


\subsection{Main Project File}
\label{\detokenize{tech_storage:main-project-file}}
\sphinxAtStartPar
The project itself requires a dedicated folder for storing its files, where novelWriter will create
its own “file system” where the project’s folder and file hierarchy is described in a project XML
file. This is the main project file in the project’s root folder with the name \sphinxcode{\sphinxupquote{nwProject.nwx}}.
This file also contains all the meta data required for the project (except the index data), and a
number of related project settings.

\sphinxAtStartPar
If this file is lost or corrupted, the structure of the project is lost, although not the text
itself. It is important to keep this file backed up, either through the built\sphinxhyphen{}in backup tool, or
your own backup solution.

\sphinxAtStartPar
The project XML file is indent\sphinxhyphen{}formatted, and is suitable for diff tools and version control since
most of the file will stay static, although a timestamp is set in the meta section on line 2, and
various meta data entries incremented, on each save.

\sphinxAtStartPar
A full project file format specification is available in the online \sphinxhref{https://docs.novelwriter.io/}{documentation}.


\section{Project Documents}
\label{\detokenize{tech_storage:project-documents}}
\sphinxAtStartPar
All the project documents are saved in a subfolder of the main project folder named \sphinxcode{\sphinxupquote{content}}.
Each document has a file handle based on a 52 bit random number, represented as a hexadecimal
string. The documents are saved with a filename assembled from this handle and the file extension
\sphinxcode{\sphinxupquote{.nwd}}.

\sphinxAtStartPar
If you wish to find the file system location of a document in the project, you can either look it
up in the project XML file, select \sphinxstylestrong{Show File Details} from the \sphinxstylestrong{Document} menu when having the
document open in the editor, or look in the \sphinxcode{\sphinxupquote{ToC.txt}} file in the root of the project folder. The
\sphinxcode{\sphinxupquote{ToC.txt}} file has a list of all documents in the project, referenced by their label, and where
they are saved.

\sphinxAtStartPar
The reason for this cryptic file naming is to avoid issues with file naming conventions and
restrictions on different operating systems, and also to have a file name that does not depend on
what you name the document within the project, or changes it to. This is particularly useful when
using a versioning system.

\sphinxAtStartPar
Each document file contains a plain text version of the text from the editor. The file can in
principle be edited in any text editor, and is suitable for diffing and version control if so
desired. Just make sure the file remains in utf\sphinxhyphen{}8 encoding, otherwise unicode characters may
become mangled when the file is opened in novelWriter again.

\sphinxAtStartPar
Editing these files is generally not recommended. The reason for this is that the index will not be
automatically updated when doing so, which means novelWriter doesn’t know you’ve altered the file.
If you \sphinxstyleemphasis{do} edit a file in this manner, you should rebuild the index when you next open the project
in novelWriter.

\sphinxAtStartPar
The first lines of the file may contain some meta data starting with the characters \sphinxcode{\sphinxupquote{\%\%\textasciitilde{}}}. These
lines are mainly there to restore some information if the file is lost from the main project file,
and the information may be helpful if you do open the file in an external editor as it contains the
document label and the document class and layout. The lines can be deleted without any consequences
to the rest of the content of the file, and will be added back the next time the document is saved
in novelWriter.


\subsection{The File Saving Process}
\label{\detokenize{tech_storage:the-file-saving-process}}
\sphinxAtStartPar
When saving the project file, or any of the documents, the data is first saved to a temporary file.
If successful, the old data file is then removed, and the temporary file replaces it. This ensures
that the previously saved data is only replaced when the new data has been successfully saved to
the storage medium.


\section{Project Meta Data}
\label{\detokenize{tech_storage:project-meta-data}}
\sphinxAtStartPar
The project folder contains a subfolder named \sphinxcode{\sphinxupquote{meta}}, containing a number of files. The meta
folder contains semi\sphinxhyphen{}important files. That is, they can be lost with only minor impact to the
project. All files in this folder are JSON or JSON Lines files, although some other files may
remain from earlier versions of novelWriter as they haven’t all been JSON files in the past.

\sphinxAtStartPar
If you use version control software on your project, you can exclude this folder, although you may
want to track the session log file and the custom words list.


\subsection{The Project Index}
\label{\detokenize{tech_storage:the-project-index}}
\sphinxAtStartPar
Between writing sessions, the project index is saved in a JSON file in \sphinxcode{\sphinxupquote{meta/index.json}}.
This file is not critical. If it is lost, it can be completely rebuilt from within novelWriter from
the \sphinxstylestrong{Tools} menu.

\sphinxAtStartPar
The index is maintained and updated whenever a document or note is saved in the editor. It contains
all references and tags in documents and notes, as well as the location of all headers in the
project, and the word counts within each header section.

\sphinxAtStartPar
The integrity of the index is checked when the file is loaded. It is possible to corrupt the index
if the file is manually edited and manipulated, so the check is important to avoid sudden crashes
of novelWriter. If the file contains errors, novelWriter will automatically build it anew. If the
check somehow fails and novelWriter keeps crashing, you can delete the file manually and rebuild
the index. If this too fails, you have likely encountered a bug.


\subsection{Build Definitions}
\label{\detokenize{tech_storage:build-definitions}}
\sphinxAtStartPar
The build definitions from the \sphinxstylestrong{Manuscript Build} tool are kept in the \sphinxcode{\sphinxupquote{meta/builds.json}} file.
If this file is lost, all custom build definitions are lost too.


\subsection{Cached GUI Options}
\label{\detokenize{tech_storage:cached-gui-options}}
\sphinxAtStartPar
A file named \sphinxcode{\sphinxupquote{meta/options.json}} contains the latest state of various GUI buttons, switches,
dialog window sizes, column sizes, etc, from the GUI. These are the GUI settings that are specific
to the project. Global GUI settings are stored in the main config file.

\sphinxAtStartPar
The file is not critical, but if it is lost, all such GUI options will revert back to their default
settings.


\subsection{Custom Word List}
\label{\detokenize{tech_storage:custom-word-list}}
\sphinxAtStartPar
A file named \sphinxcode{\sphinxupquote{meta/userdict.json}} contains all the custom words you’ve added to the project for
spell checking purposes. The content of the file can be edited from the \sphinxstylestrong{Tools} menu. If you lose
this file, all your custom spell check words will be lost too.


\subsection{Session Stats}
\label{\detokenize{tech_storage:session-stats}}
\sphinxAtStartPar
The writing progress is saved in the \sphinxcode{\sphinxupquote{meta/sessions.jsonl}} file. This file records the length
and word counts of each writing session on the given project. The file is used by the \sphinxstylestrong{Writing
Statistics} tool. If this file is lost, the history it contains is also lost, but it has otherwise
no impact on the project.

\sphinxAtStartPar
Each session is recorded as a JSON object on a single line of the file. Each session record is
appended tot he file.

\sphinxstepscope


\chapter{Running from Source}
\label{\detokenize{tech_source:running-from-source}}\label{\detokenize{tech_source:a-source}}\label{\detokenize{tech_source::doc}}
\sphinxAtStartPar
This chapter describes various ways of running novelWriter directly from the source code, and how
to build the various components like the translation files and documentation.

\begin{sphinxadmonition}{note}{Note:}
\sphinxAtStartPar
The text below assumes the command \sphinxcode{\sphinxupquote{python}} corresponds to a Python 3 executable. Python 2 is
now deprecated, but on many systems the command \sphinxcode{\sphinxupquote{python3}} may be needed instead. Likewise,
\sphinxcode{\sphinxupquote{pip}} may need to be replaced with \sphinxcode{\sphinxupquote{pip3}}.
\end{sphinxadmonition}

\sphinxAtStartPar
Most of the custom commands for building packages of novelWriter, or building assets, are contained
in the \sphinxcode{\sphinxupquote{pkgutils.py}} script in the root of the source code. You can list the available commands
by running:

\begin{sphinxVerbatim}[commandchars=\\\{\}]
python\PYG{+w}{ }pkgutils.py\PYG{+w}{ }\PYG{n+nb}{help}
\end{sphinxVerbatim}


\section{Dependencies}
\label{\detokenize{tech_source:dependencies}}\label{\detokenize{tech_source:a-source-depend}}
\sphinxAtStartPar
novelWriter has been designed to rely on as few dependencies as possible. Only the Python wrapper
for the Qt GUI libraries is required. The package for spell checking is optional, but recommended.
Everything else is handled with standard Python libraries.

\sphinxAtStartPar
The following Python packages are needed to run all features of novelWriter:
\begin{itemize}
\item {} 
\sphinxAtStartPar
\sphinxcode{\sphinxupquote{PyQt5}} \textendash{} needed for connecting with the Qt5 libraries.

\item {} 
\sphinxAtStartPar
\sphinxcode{\sphinxupquote{PyEnchant}} \textendash{} needed for spell checking (optional).

\end{itemize}

\sphinxAtStartPar
PyQt/Qt must be at least 5.15.0. If you want spell checking, you must install the \sphinxcode{\sphinxupquote{PyEnchant}}
package. The spell check library must be at least 3.0 to work with Windows. On Linux, 2.0 also
works fine.

\sphinxAtStartPar
If you install from PyPi, these dependencies should be installed automatically. If you install from
source, dependencies can still be installed from PyPi with:

\begin{sphinxVerbatim}[commandchars=\\\{\}]
pip\PYG{+w}{ }install\PYG{+w}{ }\PYGZhy{}r\PYG{+w}{ }requirements.txt
\end{sphinxVerbatim}

\begin{sphinxadmonition}{note}{Note:}
\sphinxAtStartPar
On Linux distros, the Qt library is usually split up into multiple packages. In some cases,
secondary dependencies may not be installed automatically. For novelWriter, the library files
for rendering the SVG icons may be left out and needs to be installed manually. This is the
case on for instance Arch Linux.
\end{sphinxadmonition}


\section{Build and Install from Source}
\label{\detokenize{tech_source:build-and-install-from-source}}\label{\detokenize{tech_source:a-source-install}}
\sphinxAtStartPar
If you want to install novelWriter directly from the source available on \sphinxhref{https://github.com/vkbo/novelWriter/releases}{GitHub}, you must first
build the package using the Python Packaging Authority’s build tool. It can be installed with:

\begin{sphinxVerbatim}[commandchars=\\\{\}]
pip\PYG{+w}{ }install\PYG{+w}{ }build
\end{sphinxVerbatim}

\sphinxAtStartPar
On Debian\sphinxhyphen{}based systems the tool can also be installed with:

\begin{sphinxVerbatim}[commandchars=\\\{\}]
sudo\PYG{+w}{ }apt\PYG{+w}{ }install\PYG{+w}{ }python3\PYGZhy{}build
\end{sphinxVerbatim}

\sphinxAtStartPar
With the tool installed, run the following command from the root of the novelWriter source code:

\begin{sphinxVerbatim}[commandchars=\\\{\}]
python\PYG{+w}{ }\PYGZhy{}m\PYG{+w}{ }build\PYG{+w}{ }\PYGZhy{}\PYGZhy{}wheel
\end{sphinxVerbatim}

\sphinxAtStartPar
This should generate a \sphinxcode{\sphinxupquote{.whl}} file in the \sphinxcode{\sphinxupquote{dist/}} folder at your current location. The wheel
file can then be installed on your system. Here with example version number 2.0.7, but yours may be
different:

\begin{sphinxVerbatim}[commandchars=\\\{\}]
pip\PYG{+w}{ }install\PYG{+w}{ }\PYGZhy{}\PYGZhy{}user\PYG{+w}{ }dist/novelWriter\PYGZhy{}2.0.7\PYGZhy{}py3\PYGZhy{}none\PYGZhy{}any.whl
\end{sphinxVerbatim}


\section{Building the Translation Files}
\label{\detokenize{tech_source:building-the-translation-files}}\label{\detokenize{tech_source:a-source-i18n}}
\sphinxAtStartPar
If you installed novelWriter from a package, the translation files should be pre\sphinxhyphen{}built and
included. If you’re running novelWriter from the source code, you will need to generate the files
yourself. The files you need will be written to the \sphinxcode{\sphinxupquote{novelwriter/assets/i18n}} folder, and will
have the \sphinxcode{\sphinxupquote{.qm}} file extension.

\sphinxAtStartPar
You can build the \sphinxcode{\sphinxupquote{.qm}} files with:

\begin{sphinxVerbatim}[commandchars=\\\{\}]
python\PYG{+w}{ }pkgutils.py\PYG{+w}{ }qtlrelease
\end{sphinxVerbatim}

\sphinxAtStartPar
This requires that the Qt Linguist tool is installed on your system. On Ubuntu and Debian, the
needed package is called \sphinxcode{\sphinxupquote{qttools5\sphinxhyphen{}dev\sphinxhyphen{}tools}}.

\begin{sphinxadmonition}{note}{Note:}
\sphinxAtStartPar
If you want to improve novelWriter with translation files for another language, or update an
existing translation, instructions for how to contribute can be found in the \sphinxcode{\sphinxupquote{README.md}} file
in the \sphinxcode{\sphinxupquote{i18n}} folder of the source code.
\end{sphinxadmonition}


\section{Building the Example Project}
\label{\detokenize{tech_source:building-the-example-project}}\label{\detokenize{tech_source:a-source-sample}}
\sphinxAtStartPar
In order to be able to create new projects from example files, you need a \sphinxcode{\sphinxupquote{sample.zip}} file in
the \sphinxcode{\sphinxupquote{assets}} folder of the source. This file can be built from the \sphinxcode{\sphinxupquote{pkgutils.py}} script by
running:

\begin{sphinxVerbatim}[commandchars=\\\{\}]
python\PYG{+w}{ }pkgutils.py\PYG{+w}{ }sample
\end{sphinxVerbatim}


\section{Building the Documentation}
\label{\detokenize{tech_source:building-the-documentation}}\label{\detokenize{tech_source:a-source-docs}}
\sphinxAtStartPar
A local copy of this documentation can be generated as HTML. This requires installing some Python
packages from PyPi:

\begin{sphinxVerbatim}[commandchars=\\\{\}]
pip\PYG{+w}{ }install\PYG{+w}{ }\PYGZhy{}r\PYG{+w}{ }docs/source/requirements.txt
\end{sphinxVerbatim}

\sphinxAtStartPar
The documentation can then be built from the root folder in the source code by running:

\begin{sphinxVerbatim}[commandchars=\\\{\}]
make\PYG{+w}{ }\PYGZhy{}C\PYG{+w}{ }docs\PYG{+w}{ }html
\end{sphinxVerbatim}

\sphinxAtStartPar
If successful, the documentation should be available in the \sphinxcode{\sphinxupquote{docs/build/html}} folder and you can
open the \sphinxcode{\sphinxupquote{index.html}} file in your browser.

\sphinxAtStartPar
You can also build a PDF manual from the documentation using the \sphinxcode{\sphinxupquote{pkgutils.py}} script:

\begin{sphinxVerbatim}[commandchars=\\\{\}]
python\PYG{+w}{ }pkgutils.py\PYG{+w}{ }manual
\end{sphinxVerbatim}

\sphinxAtStartPar
This will build the documentation as a PDF using LaTeX. The file will then be copied into the
assets folder and made available in the \sphinxstylestrong{Help} menu in novelWriter. The Sphinx build system has a
few extra dependencies when building the PDF. Please check the \sphinxhref{https://www.sphinx-doc.org/}{Sphinx Docs} for more details.

\sphinxstepscope


\chapter{Running Tests}
\label{\detokenize{tech_tests:running-tests}}\label{\detokenize{tech_tests:a-pytest}}\label{\detokenize{tech_tests::doc}}
\sphinxAtStartPar
The novelWriter source code is well covered by tests. The test framework used for the development
is \sphinxcode{\sphinxupquote{pytest}} with the use of an extension for Qt.


\section{Dependencies}
\label{\detokenize{tech_tests:dependencies}}
\sphinxAtStartPar
The dependencies for running the tests can be installed with:

\begin{sphinxVerbatim}[commandchars=\\\{\}]
pip\PYG{+w}{ }install\PYG{+w}{ }\PYGZhy{}r\PYG{+w}{ }tests/requirements.txt
\end{sphinxVerbatim}

\sphinxAtStartPar
This will install a couple of extra packages for coverage and test management. The minimum
requirement is \sphinxcode{\sphinxupquote{pytest}} and \sphinxcode{\sphinxupquote{pytest\sphinxhyphen{}qt}}.


\section{Simple Test Run}
\label{\detokenize{tech_tests:simple-test-run}}
\sphinxAtStartPar
To run the tests, you simply need to execute the following from the root of the source folder:

\begin{sphinxVerbatim}[commandchars=\\\{\}]
pytest
\end{sphinxVerbatim}

\sphinxAtStartPar
Since several of the tests involve opening up the novelWriter GUI, you may want to disable the GUI
for the duration of the test run. Moving your mouse while the tests are running may otherwise
interfere with the execution of some tests.

\sphinxAtStartPar
You can disable the renderring of the GUI by setting the flag \sphinxcode{\sphinxupquote{QT\_QPA\_PLATFORM=offscreen}}:

\begin{sphinxVerbatim}[commandchars=\\\{\}]
\PYG{n+nb}{export}\PYG{+w}{ }\PYG{n+nv}{QT\PYGZus{}QPA\PYGZus{}PLATFORM}\PYG{o}{=}offscreen\PYG{+w}{ }pytest
\end{sphinxVerbatim}


\section{Advanced Options}
\label{\detokenize{tech_tests:advanced-options}}
\sphinxAtStartPar
Adding the flag \sphinxcode{\sphinxupquote{\sphinxhyphen{}v}} to the \sphinxcode{\sphinxupquote{pytest}} command will increase verbosity of the test execution.

\sphinxAtStartPar
You can also add coverage report generation. For instance to HTML:

\begin{sphinxVerbatim}[commandchars=\\\{\}]
\PYG{n+nb}{export}\PYG{+w}{ }\PYG{n+nv}{QT\PYGZus{}QPA\PYGZus{}PLATFORM}\PYG{o}{=}offscreen\PYG{+w}{ }pytest\PYG{+w}{ }\PYGZhy{}v\PYG{+w}{ }\PYGZhy{}\PYGZhy{}cov\PYG{o}{=}novelwriter\PYG{+w}{ }\PYGZhy{}\PYGZhy{}cov\PYGZhy{}report\PYG{o}{=}html
\end{sphinxVerbatim}

\sphinxAtStartPar
Other useful report formats are \sphinxcode{\sphinxupquote{xml}}, and \sphinxcode{\sphinxupquote{term}} for terminal output.

\sphinxAtStartPar
You can also run tests per subpackage of novelWriter with the \sphinxcode{\sphinxupquote{\sphinxhyphen{}m}} command. The available
subpackage groups are \sphinxcode{\sphinxupquote{base}}, \sphinxcode{\sphinxupquote{core}}, and \sphinxcode{\sphinxupquote{gui}}. Consider for instance:

\begin{sphinxVerbatim}[commandchars=\\\{\}]
\PYG{n+nb}{export}\PYG{+w}{ }\PYG{n+nv}{QT\PYGZus{}QPA\PYGZus{}PLATFORM}\PYG{o}{=}offscreen\PYG{+w}{ }pytest\PYG{+w}{ }\PYGZhy{}v\PYG{+w}{ }\PYGZhy{}\PYGZhy{}cov\PYG{o}{=}novelwriter\PYG{+w}{ }\PYGZhy{}\PYGZhy{}cov\PYGZhy{}report\PYG{o}{=}html\PYG{+w}{ }\PYGZhy{}m\PYG{+w}{ }core
\end{sphinxVerbatim}

\sphinxAtStartPar
This will only run the tests of the “core” package, that is, all the classes that deal with the
project data of a novelWriter project. The “gui” tests, likewise, will run the tests for the GUI
components, and the “base” tests cover the bits in\sphinxhyphen{}between.

\sphinxAtStartPar
You can also filter the tests with the \sphinxcode{\sphinxupquote{\sphinxhyphen{}k}} switch. The following will do the same as
\sphinxcode{\sphinxupquote{\sphinxhyphen{}m core}}:

\begin{sphinxVerbatim}[commandchars=\\\{\}]
\PYG{n+nb}{export}\PYG{+w}{ }\PYG{n+nv}{QT\PYGZus{}QPA\PYGZus{}PLATFORM}\PYG{o}{=}offscreen\PYG{+w}{ }pytest\PYG{+w}{ }\PYGZhy{}v\PYG{+w}{ }\PYGZhy{}\PYGZhy{}cov\PYG{o}{=}novelwriter\PYG{+w}{ }\PYGZhy{}\PYGZhy{}cov\PYGZhy{}report\PYG{o}{=}html\PYG{+w}{ }\PYGZhy{}k\PYG{+w}{ }testCore
\end{sphinxVerbatim}

\sphinxAtStartPar
All tests are named in such a way that you can filter them by adding more bits of the test names.
They all start with the word “test”. Then comes the group: “Core”, “Base”, “Dlg”, “Tool”, or “Gui”.
Finally comes the name of the class or module, which generally corresponds to a single source code
file. For instance, running the following will run all tests for the document editor:

\begin{sphinxVerbatim}[commandchars=\\\{\}]
\PYG{n+nb}{export}\PYG{+w}{ }\PYG{n+nv}{QT\PYGZus{}QPA\PYGZus{}PLATFORM}\PYG{o}{=}offscreen\PYG{+w}{ }pytest\PYG{+w}{ }\PYGZhy{}v\PYG{+w}{ }\PYGZhy{}\PYGZhy{}cov\PYG{o}{=}novelwriter\PYG{+w}{ }\PYGZhy{}\PYGZhy{}cov\PYGZhy{}report\PYG{o}{=}html\PYG{+w}{ }\PYGZhy{}k\PYG{+w}{ }testGuiEditor
\end{sphinxVerbatim}

\sphinxAtStartPar
To run a single test, simply add the full test name to the \sphinxcode{\sphinxupquote{\sphinxhyphen{}k}} switch.



\renewcommand{\indexname}{Index}
\printindex
\end{document}